\documentclass[11pt]{article}
\usepackage[margin=1in]{geometry}
\usepackage{amssymb,amsmath,amsthm,amscd,url,hyperref}



\newcommand{\Z}{\mathbb{Z}}
\newcommand{\Q}{\mathbb{Q}}
\newcommand{\R}{\mathbb{R}}
\newcommand{\F}{\mathbb{F}}
\newcommand{\C}{\mathbb{C}}

\DeclareMathOperator{\End}{End}

\begin{document}
\begin{center}
\Large 2015 Algebra Prelim\\
\normalsize September 14, 2015
\end{center}
\vspace{1em}

INSTRUCTIONS: Do as many of the eight problems as you can. Four completely
correct solutions will be a pass; a few complete solutions will count more than many
partial solutions. Always carefully justify your answers. If you skip a step or omit
some details in a proof, point out the gap and, if possible, indicate what would be
required to fill it in\\
\vspace{1em}


1. (a) Find an irreducible polynomial of degree 5 over the field $\Z_2$ of two elements
and use it to construct a field of order 32 as a quotient of the polynomial ring $\Z_2[x]$.

(b) Using the polynomial you found in part (a), find a $5\times 5$ matrix $M$ over $\Z_2$
of order $31$, so that $M^{31} = I$ but $M \neq I$.\\

2. Find the minimal polynomial of $\sqrt{2}+\sqrt{3}$ over $\Q$. Justify your answer\\

3. (a) Let $R$ be a commutative ring with no nonzero nilpotent elements. Show
that the only units in the polynomial ring $R[x]$ are the units of $R$, regarded as
constant polynomials.

(b) Find all units in the polynomial ring $\Z_4[x]$.\\

4. Let $p$ and $q$ be two distinct primes. Prove that there is at most one nonabelian
group of order $pq$ (up to isomorphisms) and describe the pairs $(p, q)$ such
that there is no non-abelian group of order $pq$.\\

5. (a) Let $L$ be a Galois extension of a field $K$ of degree 4. What is the minimum
number of subfields there could be strictly between $K$ and $L$? What is the maximum
number of such subfields? Give examples where these bounds are attained.

(b) How do these numbers change if we assume only that $L$ is separable (but not
necessarily Galois) over $K$?\\

6. (a) Let $R$ be a commutative algebra over $\C$. A derivation of $R$ is a $\C$-linear map $D:R\to R$ such that (i) $D(1) = 0$, and (ii) $D(ab) = D(a)b + aD(b)$ for all $a,b\in R$.

(a) Describe all derivations of the polynomial ring $\C[x]$.

(b) Let $A$ be the subring (or $\C$-subalgebra) of $\End_\C(\C[x])$ generated by all derivations of $\C[x]$ and the left multiplications by $x$. Prove that $\C[x]$ is a simple left $A$-module. Note that the inclusion $A\to \End_\C(\C[x])$ defines a natural left $A$-module structure on $\C[x]$. \\


7. Let $G$ be a non-abelian group of order $p^3$ with $p$ a prime.

(a) Determine the order of the center $Z$ of $G$.

(b) Determine the number of inequivalent complex 1-dimensional representations of $G$. 

(c) Compute the dimensions of all the inequivalent irreducible representations of $G$ and verify that the number of such representations equals the number of conjugacy classes of $G$. \\

8.  Prove that every finitely generated projective module over a commutative
noetherian local ring is free.
\end{document}