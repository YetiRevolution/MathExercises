\documentclass[11pt]{article}
\usepackage[margin=1in]{geometry}
\usepackage{amssymb,amsmath,amsthm,amscd,url,hyperref}



\newcommand{\Z}{\mathbb{Z}}
\newcommand{\Q}{\mathbb{Q}}
\newcommand{\R}{\mathbb{R}}
\newcommand{\F}{\mathbb{F}}
\newcommand{\C}{\mathbb{C}}

\DeclareMathOperator{\End}{End}

\begin{document}
\begin{center}
\Large 2015 Algebra Prelim\\
\normalsize September 14, 2015
\end{center}
\vspace{1em}


1. (a) Find an irreducible polynomial of degree 5 over the field $\Z_2$ of two elements
and use it to construct a field of order 32 as a quotient of the polynomial ring $\Z_2[x]$.

(b) Using the polynomial you found in part (a), find a $5\times 5$ matrix $M$ over $\Z_2$
of order $31$, so that $M^{31} = I$ but $M \neq I$.\\\\
\textbf{Solution:} \\
(a)\\
To prove that a degree five polynomial is irreducible it suffices to show that it has no roots in $\Z_2$ and no quadratic factors (factors of degree three or four imply quadratic factors and roots respectively). Among all 32 degree five polynomials in $\Z_2[x]$ we can search for one with no linear or quadratic factors by brute force. We find quickly that $f(x) = x^5+x^3+1$ has no roots (and hence no linear factors) and furthermore we can check that it is not a multiple of any of the four quadratic polynomials in $\Z_2[x]$:\begin{itemize}
\item $f(x)$ is not a multiple of $x^2$ or $x^2+x$ since it has a nonzero constant term.
\item $f(x)$ is not a multiple of $x^2+1$ since $x^2+1$ has a root in $\Z_2$ while $f(x)$ does not.
\item $f(x)$ is not a multiple of $x^2+x+1$ because by the Euclidean algorithm we have $f(x) = (x^2+x+1)(x^3+x^2+x) + (x+1)$ and so $f(x)$ has nonzero remainder when divided by $x^2+x+1$. 
\end{itemize}
We conclude that $f(x)$ has no linear or quadratic factors in $\Z_2[x]$ and so is irreducible. Since it is irreducible we know that $\Z_2[x]/\langle f(x)\rangle$ is a field, and it will have order $2^5 = 32$ since $f(x)$ has degree five. In particular this field is a 5-dimensional vector space over $\Z_2$.\\\\
(b)\\
 To find a matrix of order 31 we consider $\F$ as a 5-dimensional vector space over $\Z_2$, and associate each $p(x)\in \F$ to the linear transformation corresponding to multiplication by $p(x)$. This yields an embedding of $\F$ into the ring of $5\times 5$ matrices over $\Z_2$. To compute the specific matrix associated to each $p(x)$ we need to specify a basis for $\F$ over $\Z_2$. A simple one is given by $\{1,x,x^2,x^3,x^4\}$. 

The group of units of $\F$ has order 31, a prime, and so any nonzero nonidentity element of $\F$ generates it. We choose $x$ as our generator and note that $x$ has multiplicative order 31. To associate $x$ to a matrix we consider its action on the basis previously described. Under this basis the action of $x$ is described by the matrix \[
\begin{bmatrix}
0&0&0&0&1\\
1&0&0&0&0\\
0&1&0&0&1\\
0&0&1&0&0\\
0&0&0&1&0
\end{bmatrix}
\]
where the last column arises from the relation $x^5 = x^3+1$ in $\F$. Since the embedding of $\F$ into the ring of $5\times 5$ matrices preserves order we conclude that the matrix above has the same order as $x$, namely 31. \\\\

2. Find the minimal polynomial of $\sqrt{2}+\sqrt{3}$ over $\Q$. Justify your answer.\\\\
\textbf{Solution:}\\
Let $\alpha = \sqrt{2}+\sqrt{3}$ and $F = \Q(\sqrt{2},\sqrt{3})$. Note that $F$ is Galois over $\Q$ and contains $\alpha$, and so to determine the other roots of $\min_\alpha(\Q)$ we need only determine the possible images of $\alpha$ under the elements of $\mbox{Gal}(F/\Q)$. There are four elements of $\mbox{Gal}(F/\Q)$: the identity, the map which replaces $\sqrt{2}$ by its negative, the map which replaces $\sqrt{3}$ by its negative, and the map which replaces both $\sqrt{2}$ and $\sqrt{3}$ by their negatives. From this we see quickly that the other roots of $\min_\alpha(\Q)$ are $-\sqrt{2}+\sqrt{3}$, $\sqrt{2}-\sqrt{3}$, and $-\sqrt{2}-\sqrt{3}$. Thus we have \begin{align*}
\min_\alpha(\Q) &= (x-\sqrt{2}-\sqrt{3})(x+\sqrt{2}+\sqrt{3})(x-\sqrt{2}+\sqrt{3})(x+\sqrt{2}-\sqrt{3})\\
& = (x^2 - 5 - 2\sqrt{6})(x^2 -5+2\sqrt{6})\\
& = \fbox{$x^4 -10x +1$}.
\end{align*}
\newpage

3. (a) Let $R$ be a commutative ring with no nonzero nilpotent elements. Show
that the only units in the polynomial ring $R[x]$ are the units of $R$, regarded as
constant polynomials.

(b) Find all units in the polynomial ring $\Z_4[x]$.\\\\
\textbf{Solution:}
(a)\\
In the case that $R$ is an integral domain the result is clear: since there are no zero divisors the product of a nonconstant polynomial with another polynomial is always nonconstant, and in particular not equal to 1. In the general case, let $f(x) = a_nx^n + \cdots + a_1 x + a_0$ be a unit in $R[x]$. This implies that the image of $f(x)$ in $R/I[x]$ is also a unit for any prime ideal $I$ of $R$. But when $I$ is prime $R/I$ is an integral domain, and so we see that $a_n,a_{n-1},\ldots, a_1$ must be zero in $R/I$ for all prime ideals $I\subseteq R$. Thus $a_n,a_{n-1},\ldots, a_1$ are contained in every prime ideal of $R$. But the nilradical is the intersection of all prime ideals in $R$, and so these $a_i$ are all in the nilradical of $R$. Under the assumptions of the problem $R$ has trivial nilradical, and so $f(x) = a_0$. The only constant polynomials which are units are clearly the units of $R$, and so the result follows.\\\\
(b)\\
We saw in part (a) that if $f(x) = a_nx^n + \cdots + a_1 x + a_0$ is a unit then $a_n,\ldots, a_1$ must be contained in the nilradical of $R$, and it is also clear that $a_0$ must be a unit in $R$. We will show that these conditions on the $a_i$ are sufficient for $f(x)$ to be a unit. Recall that the sum of a nilpotent element and a unit is again a unit in any commutative ring, and notice that all $a_ix^i$ are nilpotent for $1\le i \le n$ as long as each $a_i$ is nilpotent in $R$. In fact $f(x)-a_0$ is nilpotent, since it is the sum of finitely many nilpotent elements. Then we can write $f(x)$ as the sum of a nilpotent element and a unit: $f(x) = (f(x)-a_0) + a_0$. We conclude that the units in $R[x]$ are exactly \[
(R[x])^\times = \{a_nx^n + \cdots + a_1 x + a_0 \mid a_0\text{ is a unit in $R$, and $a_i$ is nilpotent in $R$ for $1\le i \le n$}\}
\]
The ring $\Z_4$ has nilradical $\{0,2\}$ and its units are $\{1,3\}$. Thus the units in $\Z_4[x]$ are those such that the constant coefficient is odd and all other coefficients are even. 



\newpage
4. Let $p$ and $q$ be two distinct primes. Prove that there is at most one nonabelian
group of order $pq$ (up to isomorphisms) and describe the pairs $(p, q)$ such
that there is no non-abelian group of order $pq$.\\
\newpage
5. (a) Let $L$ be a Galois extension of a field $K$ of degree 4. What is the minimum
number of subfields there could be strictly between $K$ and $L$? What is the maximum
number of such subfields? Give examples where these bounds are attained.

(b) How do these numbers change if we assume only that $L$ is separable (but not
necessarily Galois) over $K$?\\\\
\textbf{Solution:}\\
(a)\\
If $L$ is Galois over $K$ of degree four, then we know $\mbox{Gal}(L/K)$ has four elements. The number of nontrivial proper subgroups of $\mbox{Gal}(L/K)$ is exactly the number of intermediate fields strictly between $L$ and $K$ by the Galois correspondence. There are only two groups of order four: $\Z_4$ and $\Z_2\times \Z_2$. The former has a single intermediate subgroup generated by 2. The latter has three subgroups of order 2, generated by $(1,0)$, $(0,1)$ and $(1,1)$. Thus we see that the smallest number of intermediate fields is 1, while the largest is 3 (and in fact we can never have exactly 2). 

An extension in which there is a single intermediate field is $\Q(\zeta)$ where $\zeta$ is a primitive 5th root of unity. This extension is Galois since it is the splitting field of $x^4+x^3+x^2+1$ over  $\Q$. The Galois group of this extension cyclically permutes the set $\{\zeta, \zeta^2,\zeta^3,\zeta^4\}$ (in this order), and the single intermediate field is $\Q(\zeta+\zeta^3)$ which is equal to $\Q(\zeta^2+\zeta^4)$. An extension with three intermediate fields is $\Q(\sqrt{2},\sqrt{3})$, the splitting field of $(x^2-2)(x^2-3)$ over $\Q$. The intermediate fields in this case are the quadratic extensions $\Q(\sqrt{2})$, $\Q(\sqrt{3})$ and $\Q(\sqrt{6})$.\\\\
(b)\\
For $L$ to be separable but not Galois, it must be the case that $L$ is not normal. Thus we seek an extension which is separable but which contains an element whose minimal polynomial over $K$ does not split in $L$. 
\newpage
6. Let $R$ be a commutative algebra over $\C$. A derivation of $R$ is a $\C$-linear map $D:R\to R$ such that (i) $D(1) = 0$, and (ii) $D(ab) = D(a)b + aD(b)$ for all $a,b\in R$.

(a) Describe all derivations of the polynomial ring $\C[x]$.

(b) Let $A$ be the subring (or $\C$-subalgebra) of $\End_\C(\C[x])$ generated by all derivations of $\C[x]$ and the left multiplications by $x$. Prove that $\C[x]$ is a simple left $A$-module. Note that the inclusion $A\to \End_\C(\C[x])$ defines a natural left $A$-module structure on $\C[x]$. \\\\
\textbf{Solution:}\\
(a)\\
We first claim that $D(x^n) = nx^{n-1} D(x)$. When $n=0$ this is clear since we have $D(x^0) = D(1) = 0$. For general $n$ we proceed by induction. Applying (ii) when $n\ge 1$ we have \begin{align*}
D(x^n) &= xD(x^{n-1}) + x^{n-1} D(x)\\
& = x((n-1)x^{n-2} D(x)) + x^{n-1} D(x) &&\text{(By inductive hypothesis)}\\
& = n x^{n-1} D(x)
\end{align*}
as desired. Since $D$ is a $\C$-linear map this rule is sufficient to specify the action of $D$ on all elements of $\C[x]$. Thus we see that a derivation $D$ is uniquely determined by the value of $D(x)$, on which there is no restriction. That is, every derivation is obtained by specifying $D(x) = f(x)$ and extending the action of $D$ to all of $\C[x]$ via $\C$-linearity and the identity $D(x^n) = nx^{n-1} D(x)$. \\\\
(b)\\
Let $M\subseteq \C[x]$ be a nonzero submodule of the $A$-module $\C[x]$. To prove $\C[x]$ is simple it suffices to show that $M = \C[x]$. Our approach will be to first show $\C\subseteq M$ and then use multiplication by $x$ to generate all of $\C[x]$.

 Let $f\in M$ be nonzero, and if necessary multiply $f$ by $x$ so that it is nonconstant. The result is of course still in $M$ since $M$ is invariant under the action of $A$. We may then write \[
f = \sum_{i=0}^n a_i x^i
\]
where $n\ge 1$ and $a_n\neq 0$. Letting $D\in A$ be the usual polynomial derivative, we can applying $D$ a total of $n-1$ times to $f$ to obtain a nonzero polynomial of degree exactly one which is again in $M$. Let $g = b_0 + b_1x$ denote this polynomial. Then for any $c\in \C$, let $D_c$ denote the derivation defined by $D(x) = c/b_1$ and observe that \[
D_c(g) = D_c(b_0) + b_1 D_c(x) = 0 + b_1 (c/b_1) = c
\]
is an element of $M$. Hence $\C\subseteq M$. Since $M$ is invariant under multiplication by $x$ we also have that $cx^n\in M$ for any $n\ge 0$ and $c\in \C$. Closure of $M$ under addition then gives us that $M = \C[x]$. Thus $\C[x]$ is a simple $A$-module, as desired. 

\newpage
7. Let $G$ be a non-abelian group of order $p^3$ with $p$ a prime.

(a) Determine the order of the center $Z$ of $G$.

(b) Determine the number of inequivalent complex 1-dimensional representations of $G$. 

(c) Compute the dimensions of all the inequivalent irreducible representations of $G$ and verify that the number of such representations equals the number of conjugacy classes of $G$. \\\\
\textbf{Solution:} (a) By Langrange's Theorem there are four candidates for the order of $Z$: $1,p,p^2,$ and $p^3$. Since $G$ is nonabelian we can rule out the last possibility. Groups of order $p^n$ always have nontrivial center, so we can also rule out 1. This leaves $p$ and $p^2$. Recall that the center of a group is always normal. If $|Z| = p^2$, then $G/Z$ has $p$ elements and is cyclic. But the quotient by the center being cyclic implies that $G$ is abelian, a contradiction. Hence the only possible order for $Z$ is \fbox{$p$}.\\\\
(b)
There are exactly $|G/[G,G]|$ complex 1-dimensional representations. To see why this is the case, observe that a complex 1-dimensional representation of $G$ is a group homomorphism $\rho:G\to \C^\times$. Since $\C^\times$ is abelian, the commutator $[G,G]$ must be in the kernel of $\phi$ (otherwise the image of $\rho(G)$ would not be abelian). Hence $\rho$ is in essence a representation of the abelian group $G/[G,G]$, in the sense that any 1-dimensional representation of $G/[G,G]$ can be uniquely extended to a representation of $G$. The number of complex representations of an abelian group is simply the number of elements in the group and all representations are automatically 1-dimensional, so it follows that there are $|G/[G,G]|$ 1-dimensional representations of $G$.

Thus we seek to compute $|G/[G,G]|$. Recall that the commutator is the smallest normal subgroup $H$ so that $G/H$ is abelian. Note that $G/Z$ has order $p^2$ and is abelian, so we have $[G,G]\le Z$. But since $G$ is nonabelian we know $[G,G]$ is nontrivial and it follows that $Z = [G,G]$ since $|Z|$ is prime. We then have that $|G/[G,G]| = p^2$, and there are \fbox{$p^2$} irreducible complex 1-dimensional representations of $G$.\\\\
(c) 
From part (b) we have $p^2$ total 1-dimensional representations. Recall that the dimension of a representation always divides the order of $G$, and so the remaining representations have dimension $p$, $p^2$ or $p^3$. Moreover, the sum of squares of the dimensions of all irreducible representations equals $|G| = p^3$. The 1-dimensional representations account for a total of $p^2$ in this sum, and so if $d_1,\ldots, d_k$ are the degrees of the higher dimensional irreducible representations we must have $
p^3 = p^2 + d_1^2+ \cdots + d_k^2$
or equivalently \[
p^2(p-1) = d_1^2+\cdots + d_k^2.
\]
Since each $d_i$ is a multiple of $p$ we see that this is only possible if $d_i = p$ and $k= p-1$. Hence there are $p-1$ irreducible representations of dimension greater than one. In total we obtain \fbox{$p^2+p-1$} irreducible representations. 

To verify that this is the number of conjugacy classes in $G$ we use the class equation. If $C_1,\ldots, C_l$ are the conjugacy classes of size greater than one in $G$ we have that $
p^3 = p + \sum_{i=1}^l |C_i|
$
or equivalently \[
p(p^2-1) = \sum_{i-1}^l |C_i|.
\]
Moreover each $|C_i|$ is a multiple of $p$, since it must divide $p^3$ and is not equal to 1. We see immediately that $|C_i| = p$ for all $i$, and $l = p^2-1$. The total number of conjugacy classes is then $p^2-1+p = p^2+p-1$, since the only other conjugacy classes are of size one, arising from elements of $Z$. This concludes the proof. 

\newpage
8.  Prove that every finitely generated projective module over a commutative
noetherian local ring is free.
\end{document}