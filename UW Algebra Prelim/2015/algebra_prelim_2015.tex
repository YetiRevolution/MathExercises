\documentclass[11pt]{article}
\usepackage[margin=1in]{geometry}
\usepackage{amssymb,amsmath,amsthm,amscd,url,hyperref}



\newcommand{\Z}{\mathbb{Z}}
\newcommand{\Q}{\mathbb{Q}}
\newcommand{\R}{\mathbb{R}}
\newcommand{\F}{\mathbb{F}}
\newcommand{\C}{\mathbb{C}}

\DeclareMathOperator{\End}{End}

\begin{document}
\begin{center}
\Large 2015 Algebra Prelim\\
\normalsize September 14, 2015
\end{center}
\vspace{1em}


1. (a) Find an irreducible polynomial of degree 5 over the field $\Z_2$ of two elements
and use it to construct a field of order 32 as a quotient of the polynomial ring $\Z_2[x]$.

(b) Using the polynomial you found in part (a), find a $5\times 5$ matrix $M$ over $\Z_2$
of order $31$, so that $M^{31} = I$ but $M \neq I$.\\\\
\textbf{Solution:} \\
(a)\\
To prove that a degree five polynomial is irreducible it suffices to show that it has no roots in $\Z_2$ and no quadratic factors (factors of degree three or four imply quadratic factors and roots respectively). Among all 32 degree five polynomials in $\Z_2[x]$ we can search for one with no linear or quadratic factors by brute force. We find quickly that $f(x) = x^5+x^3+1$ has no roots (and hence no linear factors) and furthermore we can check that it is not a multiple of any of the four quadratic polynomials in $\Z_2[x]$:\begin{itemize}
\item $f(x)$ is not a multiple of $x^2$ or $x^2+x$ since it has a nonzero constant term.
\item $f(x)$ is not a multiple of $x^2+1$ since $x^2+1$ has a root in $\Z_2$ while $f(x)$ does not.
\item $f(x)$ is not a multiple of $x^2+x+1$ because by the Euclidean algorithm we have $f(x) = (x^2+x+1)(x^3+x^2+x) + (x+1)$ and so $f(x)$ has nonzero remainder when divided by $x^2+x+1$. 
\end{itemize}
We conclude that $f(x)$ has no linear or quadratic factors in $\Z_2[x]$ and so is irreducible. Since it is irreducible we know that $\Z_2[x]/\langle f(x)\rangle$ is a field, and it will have order $2^5 = 32$ since $f(x)$ has degree five. In particular this field is a 5-dimensional vector space over $\Z_2$.\\\\
(b)\\
 To find a matrix of order 31 we consider $\F$ as a 5-dimensional vector space over $\Z_2$, and associate each $p(x)\in \F$ to the linear transformation corresponding to multiplication by $p(x)$. This yields an embedding of $\F$ into the ring of $5\times 5$ matrices over $\Z_2$. To compute the specific matrix associated to each $p(x)$ we need to specify a basis for $\F$ over $\Z_2$. A simple one is given by $\{1,x,x^2,x^3,x^4\}$. 

The group of units of $\F$ has order 31, a prime, and so any nonzero nonidentity element of $\F$ generates it. We choose $x$ as our generator and note that $x$ has multiplicative order 31. To associate $x$ to a matrix we consider its action on the basis previously described. Under this basis the action of $x$ is described by the matrix \[
\begin{bmatrix}
0&0&0&0&1\\
1&0&0&0&0\\
0&1&0&0&1\\
0&0&1&0&0\\
0&0&0&1&0
\end{bmatrix}
\]
where the last column arises from the relation $x^5 = x^3+1$ in $\F$. Since the embedding of $\F$ into the ring of $5\times 5$ matrices preserves order we conclude that the matrix above has the same order as $x$, namely 31. \\\\

2. Find the minimal polynomial of $\sqrt{2}+\sqrt{3}$ over $\Q$. Justify your answer.\\\\
\textbf{Solution:}\\
Let $\alpha = \sqrt{2}+\sqrt{3}$ and $F = \Q(\sqrt{2},\sqrt{3})$. Note that $F$ is Galois over $\Q$ and contains $\alpha$, and so to determine the other roots of $\min_\alpha(\Q)$ we need only determine the possible images of $\alpha$ under the elements of $\mbox{Gal}(F/\Q)$. There are four elements of $\mbox{Gal}(F/\Q)$: the identity, the map which replaces $\sqrt{2}$ by its negative, the map which replaces $\sqrt{3}$ by its negative, and the map which replaces both $\sqrt{2}$ and $\sqrt{3}$ by their negatives. From this we see quickly that the other roots of $\min_\alpha(\Q)$ are $-\sqrt{2}+\sqrt{3}$, $\sqrt{2}-\sqrt{3}$, and $-\sqrt{2}-\sqrt{3}$. Thus we have \begin{align*}
\min_\alpha(\Q) &= (x-\sqrt{2}-\sqrt{3})(x+\sqrt{2}+\sqrt{3})(x-\sqrt{2}+\sqrt{3})(x+\sqrt{2}-\sqrt{3})\\
& = (x^2 - 5 - 2\sqrt{6})(x^2 -5+2\sqrt{6})\\
& = \fbox{$x^4 -10x +1$}.
\end{align*}

3. (a) Let $R$ be a commutative ring with no nonzero nilpotent elements. Show
that the only units in the polynomial ring $R[x]$ are the units of $R$, regarded as
constant polynomials.

(b) Find all units in the polynomial ring $\Z_4[x]$.\\

4. Let $p$ and $q$ be two distinct primes. Prove that there is at most one nonabelian
group of order $pq$ (up to isomorphisms) and describe the pairs $(p, q)$ such
that there is no non-abelian group of order $pq$.\\

5. (a) Let $L$ be a Galois extension of a field $K$ of degree 4. What is the minimum
number of subfields there could be strictly between $K$ and $L$? What is the maximum
number of such subfields? Give examples where these bounds are attained.

(b) How do these numbers change if we assume only that $L$ is separable (but not
necessarily Galois) over $K$?\\

6. (a) Let $R$ be a commutative algebra over $\C$. A derivation of $R$ is a $\C$-linear map $D:R\to R$ such that (i) $D(1) = 0$, and (ii) $D(ab) = D(a)b + aD(b)$ for all $a,b\in R$.

(a) Describe all derivations of the polynomial ring $\C[x]$.

(b) Let $A$ be the subring (or $\C$-subalgebra) of $\End_\C(\C[x])$ generated by all derivations of $\C[x]$ and the left multiplications by $x$. Prove that $\C[x]$ is a simple left $A$-module. Note that the inclusion $A\to \End_\C(\C[x])$ defines a natural left $A$-module structure on $\C[x]$. \\


7. Let $G$ be a non-abelian group of order $p^3$ with $p$ a prime.

(a) Determine the order of the center $Z$ of $G$.

(b) Determine the number of inequivalent complex 1-dimensional representations of $G$. 

(c) Compute the dimensions of all the inequivalent irreducible representations of $G$ and verify that the number of such representations equals the number of conjugacy classes of $G$. \\

8.  Prove that every finitely generated projective module over a commutative
noetherian local ring is free.
\end{document}