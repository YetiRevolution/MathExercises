\documentclass[11pt]{article}
\usepackage[margin=1in]{geometry}
\usepackage{amssymb,amsmath,amsthm,amscd,url,hyperref}



\newcommand{\Z}{\mathbb{Z}}
\newcommand{\Q}{\mathbb{Q}}
\newcommand{\R}{\mathbb{R}}
\newcommand{\F}{\mathbb{F}}
\newcommand{\C}{\mathbb{C}}

\DeclareMathOperator{\End}{End}

\begin{document}
\begin{center}
\Large 2015 Algebra Prelim\\
\normalsize September 14, 2015
\end{center}
\vspace{1em}

INSTRUCTIONS: Do as many of the eight problems as you can. Four completely
correct solutions will be a pass; a few complete solutions will count more than many
partial solutions. Always carefully justify your answers. If you skip a step or omit
some details in a proof, point out the gap and, if possible, indicate what would be
required to fill it in\\
\vspace{1em}

1. Classify all groups of order 2012 up to isomorphism. (Hint: 503 is prime.)\\


2. For any positive integer $n$, let $G_n$ be the group generated by $a$ and $b$ subject to the following three
relations:\[
a^2=1,\quad b^2=1,\quad\text{and}\quad (ab)^n=1.
\]

(a) Find the order of the group $G_n$.

(b) Classify all irreducible complex representations of $G_4$ up to isomorphism.\\

3. Let $R$ be a (commutative) principal ideal domain, let $M$ and $N$ be finitely generated free $R$-modules,
and let $\phi : M \to N$ be an $R$-module homomorphism.

(a) Let $K$ be the kernel of $\phi$. Prove that $K$ is a direct summand of $M$.

(b) Let $C$ be the image of $\phi$. Show by example (specifying $R$, $M$, $N$ and $\phi$) that $C$ need not be a direct summand of $N$.\\

4. Let $G$ be an abelian group. Prove that the group ring $\Z[G]$ is noetherian if and only if $G$ is finitely
generated.\\

5. Let $M_3(\R)$ be the $3\times 3$-matrix algebra over the real numbers $\R$. For any $b\in \R$ let $B\in M_3(\R)$ be the matrix $\begin{pmatrix}1&b&0\\b&1&b\\0&b&1\end{pmatrix}$. Find the set of numbers $b$ so that the matrix equation $X^2 = B$ has at least one, and only finitely many, solutions in $M_3(\R)$. \\

6. Determine the Galois groups of the following polynomials over $\Q$.

(a) $f(x) = x^4+4x^2+1$

(b) $f(x) = x^4 +4x^2 -5$\\

7. Prove that if $A$ is a finite abelian group, then $\mbox{Hom}_\Z(A,\Q/\Z)\cong \mbox{Ext}_\Z^1(A,\Z) \cong A$. (Here $\mbox{Ext}_\Z^1(-,-)$ is also sometimes written as $\mbox{Ext}(-,-)$. \\

8. Let $A$ be the $\C$-algebra $\C[x,y]$, and define algebra automorphisms $\sigma$ and $\tau$ of $A$ by \[
\sigma(x)= y, \quad \sigma(x) = y
\]
and \[
\tau(x) = x,\quad \tau(y) = \zeta y,
\]
where $\zeta\in \C$ is a primitive third root of unity (namely, $\zeta \neq $ and $\zeta^3 = 1$). Let $G$ be the group of algebra automorphisms of $A$ generated by $\sigma$ and $\tau$. Define \[
A^G = \{f\in A \mid \phi(f) = f\text{ for all } \phi\in G\}.
\]
Then $A^G$ is a subalgebra of $A$ -- you do not need to prove this. Describe the algebra $A^G$ by finding a set of generators and a set of relations. 


\end{document}