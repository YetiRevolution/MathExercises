\documentclass[11pt]{article}
\usepackage[margin=1in]{geometry}
\usepackage{amssymb,amsmath,amsthm,amscd,url,hyperref}



\newcommand{\Z}{\mathbb{Z}}
\newcommand{\Q}{\mathbb{Q}}
\newcommand{\R}{\mathbb{R}}
\newcommand{\F}{\mathbb{F}}
\newcommand{\C}{\mathbb{C}}

\DeclareMathOperator{\End}{End}

\begin{document}
\begin{center}
\Large 2015 Algebra Prelim\\
\normalsize September 14, 2015
\end{center}
\vspace{1em}

INSTRUCTIONS: Do as many of the eight problems as you can. Four completely
correct solutions will be a pass; a few complete solutions will count more than many
partial solutions. Always carefully justify your answers. If you skip a step or omit
some details in a proof, point out the gap and, if possible, indicate what would be
required to fill it in\\
\vspace{1em}

1. For any group $G$ we define $\Omega(G)$ to be the image of the group homomorphism
$\rho : G \to \mbox{Aut}(G)$ where $\rho$ maps $g\in G$ to the conjugation automorphism $x \mapsto gxg^{-1}$.
Starting with a group $G_0$, we define $G_1 = \Omega(G_0)$ and $G_{i+1} = \Omega(G_i)$ for all $i \ge 1$. If $G_0$
is of order $p^e$ for a prime $p$ and integer $e \ge 2$, prove that $G_{e−1}$ is the trivial group.\\

2. Let $\F_2$ be the field with 2 elements.

(a) What is the order of $GL_3(\F_2)$?

(b) Use the fact that $GL_3(\F_2)$ is a simple group (which you should not prove) to find the number of elements of order 7 in $GL_3(\F_2)$. \\


3. Let $G$ be a finite abelian group. Let $f : \Z^m \to G$ be a surjection of abelian groups.
We may think of $f$ as a homomorphism of $\Z$-modules. Let $K$ be the kernel of $f$.

(a) Prove that $K$ is isomorphic to $\Z^m$.

(b) We can therefore write an inclusion map $K\to \Z^m$ as $\Z^m\to \Z^m$ and represent it by an $m\times m$ integer matrix $A$. Prove that $|\mbox{det} A| = |G|$. \\

4. Let $R = C([0,1])$ be the ring of all continuous real-valued functions on the closed
interval $[0,1]$, and for each $c \in [0,1]$, denote by $M_c$ the set of all functions $f\in R$ such
that $f(c) = 0$.

(a) Prove that $g\in R$ is a unit if and only if $g(c) \neq 0$ for all $c\in [0,1]$. 

(b) Prove that for each $c\in [0,1]$, $M_c$ is a maximal ideal of $R$.

(c) Prove that if $M$ is a maximal ideal of $R$, then $M = M_c$ for some $c\in [0,1]$. (Hint: The compactness of $[0,1]$ may be relevant.) \\

5. Let $R$ and $S$ be commutative rings, and $f : R \to S$ a ring homomorphism.

(a) Show that if $I$ is a prime ideal of $S$, then\[
f^{-1}(I) = \{r \in R : f(r) \in I\}
\]
is a prime ideal of $R$.

(b) Let $N$ be the set of nilpotent elements of $R$:\[
N = \{r\in R : r^m = 0 \text{ for some } m \ge 1\}.\]
$N$ is called the \emph{nilradical} of $R$. Prove that it is an ideal which is contained in
every prime ideal.

(c) Part (a) lets us define a function\begin{align*}
f^* : \{\text{prime ideals of $S$}\} &\rightarrow \{\text{prime ideals of $R$}\}\\
I&\mapsto f^{-1}(I)
\end{align*}
Let $N$ be the nilradical of $R$. Show that if $S = R/N$ and $f : R \to R/N$ is the
quotient map, then $f^*$ is a bijection.\\


6. Let $F$ be a finite field of characteristic $p$. Let $A$ be an $n\times n$ matrix over $F$. Suppose that $A^p$ is the identity matrix. Show that for every polynomial $f(x)$, the characteristic polynomial of the matrix $f(A)$ is equal to $(t-c)^n$ for some $c$. \\

7. Consider the polynomial $f(x) = x^{10} + x^5 +1 \in\Q[x]$ with splitting field $K$ over $\Q$.

(a) Determine whether $f(x)$ is irreducible over $\Q$ and find $[K:\Q]$. 

(b) Detrmine the structure of the Galois group $\mbox{Gal}(K/\Q)$. \\

8. For each prime number $p$ and each positive integer $n$, how many elements $\alpha$ are there in $\F_{p^n}$ such that $\F_p(\alpha) = \F_{p^6}$? 


\end{document}