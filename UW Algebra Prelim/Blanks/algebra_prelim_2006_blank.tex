\documentclass[11pt]{article}
\usepackage[margin=1in]{geometry}
\usepackage{amssymb,amsmath,amsthm,amscd,url,hyperref}



\newcommand{\Z}{\mathbb{Z}}
\newcommand{\Q}{\mathbb{Q}}
\newcommand{\R}{\mathbb{R}}
\newcommand{\F}{\mathbb{F}}
\newcommand{\C}{\mathbb{C}}

\DeclareMathOperator{\End}{End}

\begin{document}
\begin{center}
\Large 2015 Algebra Prelim\\
\normalsize September 14, 2015
\end{center}
\vspace{1em}

INSTRUCTIONS: Do as many of the eight problems as you can. Four completely
correct solutions will be a pass; a few complete solutions will count more than many
partial solutions. Always carefully justify your answers. If you skip a step or omit
some details in a proof, point out the gap and, if possible, indicate what would be
required to fill it in\\
\vspace{1em}

1. Let $\Z$ be the ring of integers and let $\{0\}$ be the trivial $\Z$-module. Let $G$ be a finitely generated $\Z$-module,
i.e., a finitely generated Abelian group. Show that if $G\otimes_\Z F = \{0\}$ for all fields $F$, then $G = \{0\}$.\\

2. Let $c$ be an automorphism of order 1 or 2 of a field $F$. Suppose that $c$ has the property that for any finite
set $\{a_j\}_{j\in J}$ of nonzero elements of $F$, $\sum_{j\in J} a_j c(a_j)$ is nonzero. For any $n \times n$-matrix $A = (a_{ij})$ with entries
in $F$, let $A^c = (c(a_{ij}))$. The $n\times n$ identity matrix is denoted by $I_n$.

Show that if $A$ has the property that $A(A^c)^T = I_n$, then every eigenvalue $\lambda\in F$ for $A$ satisfies the equation
$\lambda c(\lambda) = 1.$

(Remark: This result implies the familiar facts that the eigenvalues of an orthogonal matrix over the field
of real numbers are in the set $\{\pm1\}$ and that the eigenvalues of a unitary matrix over the complex numbers
have absolute value 1. One takes $c$ to be the identity map in the first case, complex conjugation in the
second.)\\

3. (a) Let $p<q<r$ be prime integers. Show that a group of order $pqr$ cannot be simple.

(b) Consider groups of orders $2^2\cdot 3\cdot  p$ where $p$ has the values 5,7 and 11. For each of those values of $p$,
either display a simple group of order $2^2 \cdot 3 \cdot p$, or show that there cannot be a simple group of that order.

4. Let $K/F$ be a finite Galois extension and let $n = [K : F]$. There is a theorem (often referred to as the
“normal basis theorem”) which states that there exists an irreducible polynomial $f(x) \in F[x]$  whose roots
form a basis for $K$ as a vector space over $F$. You may assume that theorem in this problem.

(a) Let $G = \mbox{Gal}(K/F)$. The action of $G$ on $K$ makes $K$ into a finite-dimensional representation space for
$G$ over $F$. Prove that $K$ is isomorphic to the regular representation for $G$ over $F$.
(The regular representation is defined by letting $G$ act on the group algebra $F[G]$ by multiplication on the
left.)
	
(b) Suppose that the Galois group $G$ is cyclic and that $F$ contains a primitive $n$-th root of unity. Show that
there exists an injective homomorphism $\chi  : G \to F^\times$.

(c) Show that K contains a nonzero element a with the following property:
\[g(a) = \chi(g) \cdot a\]
for all $g\in G$.

(d) If $a$ has the property stated in (c), show that $K = F(a)$ and that $a^n \in F^\times$.\\

5. Let G be the group of matrices of the form $\begin{pmatrix} 1&a&b\\0&a&c\\0&0&1\end{pmatrix}$ with entries in the finite field $\F_p$ of $p$ elements, where $p$ is a prime.

(a) Prove that $G$ is nonabelian.

(b) Suppose that $p$ is odd. Prove that $g^p  = I_3$ for all $g\in G$. 

(c) Suppose that $p=2$. It is known that there are exactly two nonabelian groups of order 8, up to isomorphism: the dihedral group $D_4$ and the quaternionic group. Assuming this fact without proof, determine which of these groups $G$ is isomorphic to. \\

6. Let $R = \Z[x]$, the polynomial ring in a variable $x$ with coefficients in the integers.

(a)  Let $M$ be a maximal ideal of $R$. Show that $R/M$ is a finite field.


(b)  Suppose that $k$ is any finite field. Prove that there exists at least one and no more than a finite number
of maximal ideals $M$ such that $R/M \cong k$.

(c)  Prove that no maximal ideal of $R$ is principal, but that all nonmaximal prime ideals of $R$ are principal.\\

7. There are five nonisomorphic groups of order 8. For each of those groups $G$, find the smallest positive
integer $n$ such that there is an injective homomorphism $\phi : G \to S_n$.\\

8. Let K be the field $\Q(z)$ of rational functions in a variable $z$ with coeffients in the rational field $Q$. Let $n$
be a positive integer. Consider the polynomial $x^n - z\in  K[x]$.

(a) Show that the polynomial $x^n-z$ is irreducible over $K$.

(b) Describe the splitting field of $x^n- z$ over $K$.

(c) Determine the Galois group of the splitting field of $x^5 - z$ over the field $K$.


\end{document}