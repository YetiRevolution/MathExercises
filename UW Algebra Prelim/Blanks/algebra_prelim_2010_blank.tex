\documentclass[11pt]{article}
\usepackage[margin=1in]{geometry}
\usepackage{amssymb,amsmath,amsthm,amscd,url,hyperref}



\newcommand{\Z}{\mathbb{Z}}
\newcommand{\Q}{\mathbb{Q}}
\newcommand{\R}{\mathbb{R}}
\newcommand{\F}{\mathbb{F}}
\newcommand{\C}{\mathbb{C}}

\DeclareMathOperator{\End}{End}

\begin{document}
\begin{center}
\Large 2015 Algebra Prelim\\
\normalsize September 14, 2015
\end{center}
\vspace{1em}

INSTRUCTIONS: Do as many of the eight problems as you can. Four completely
correct solutions will be a pass; a few complete solutions will count more than many
partial solutions. Always carefully justify your answers. If you skip a step or omit
some details in a proof, point out the gap and, if possible, indicate what would be
required to fill it in\\
\vspace{1em}

1. Let $p$ be a positive prime number, $\F_p$ the field with $p$ elements, and let $G = GL_2(\F_p)$.

(a) Compute the order of $G$, $|G|$.

(b) Write down an explicit isomorphism from $\Z/p\Z$ to\[
U = \left\{ \begin{pmatrix}1&a\\0&1\end{pmatrix}\,\middle| \, a\in \F_p\right\}.
\]
(c) How many subgroups of order $p$ does $G$ have?
Hint: compute $gug^{−1}$
for $g \in G$ and $u\in U$; use this to find the size of the
normalizer of $U$ in $G$.\\

2. (a) Give definitions of the following terms: (i) a finite length (left) module, (ii) a
composition series for a module, and (iii) the length of a module.


(b) Let $l(M)$ denote the length of a module $M$. Prove that if\[
0 \to M_1\to M_2\to \cdots \to M_n\to 0
\]
is an exact sequence of modules of finite length, then\[
\sum_{i=1}^n(-1)^il(M_i) = 0.
\]

3. Let $\F$ be a field of characteristic $p$, and $G$ a group of order $p^n$. Let $R = \F[G]$ be the
group ring (group algebra) of $G$ over $\F$, and let $u := \sum_{x\in G} x$ (so $u$ is an element of $R$).

(a) Prove that $u$ lies in the center of $R$.

(b) Verify that $Ru$ is a 2-sided ideal of $R$.

(c) Show there exists a positive integer $k$ such that $u^k = 0$. Conclude that for such a $k$, $(Ru)^k = 0$.


(d) Show that $R$ is \textbf{not} a semi-simple ring. (\textbf{Warning}: Please use the definition
of a semisimple ring; do \emph{not} use the result that a finite length ring fails to be
semisimple if and only if it has a non-zero nilpotent ideal.)\\

4. Let $f(x) = a_nx^n + a_{n-1}x^{n-1}+\cdots+a_0\in \Z[x]$ (where $a_n\neq 0$) and let $R = \Z[x]/(f)$. Prove that $R$ is a finitely-generated module over $\Z$ if and only if $a_n = \pm 1$. \\

5. Consider the ring \[
S = C[0,1] = \{f:[0,1]\to \R \mid \text{$f$ is continuous}\}
\]
with the usual operations of addition and multiplication of functions.

(a) What are the invertible elements of $S$?

(b) For $a\in [0,1]$, define $I_a = \{f\in S\mid f(a) = 0\}$. Show that $I_a$ is a maximal ideal of $S$. 

(c) Show that the elements of any proper ideal of $S$ have a common zero, i.e., if $I$ is
a proper ideal of $S$, then there exists $a\in [0, 1]$ such that $f(a) = 0$ for all $f \in I$.
Conclude that every maximal ideal of $S$ is of the form $I_a$ for some $a \in[0, 1]$.
Hint: as $[0, 1]$ is compact, every open cover of $[0, 1]$ contains a finite subcover.\\

6. (a) Let $L/F$ be a field extension that is finite and Galois. Show that if the Galois
group $\mbox{Gal}(L/F)$ is abelian then for every intermediate field $F\subseteq  K \subseteq L$, $K/F$ is
also a Galois extension.

(b) Let $K = \Q(\sqrt{1+\sqrt{2}}) \subset \R$. Show that $K/\Q$ is an extension of degree 4 that is \emph{not} Galois.

(c) Let $L$ be the smallest Galois extension of $\Q$ that contains $K = \Q(\sqrt{1+\sqrt{2}})$. Compute the group $\mbox{Gal}(L/\Q)$. \\

7. Let $F$ be a field of characteristic zero, and let $K$ be an algebraic extension of $F$ that
possesses the following property: every polynomial $f \in F[x]$ has a root in $K$. Show
that $K$ is algebraically closed.

\textbf{Hint:} if $K(\theta)/K$ is algebraic, consider $F(\theta)/F$ and its normal closure; primitive elements
might be of help.\\

8. Let $G$ be the unique non-abelian group of order 21.

(a) Describe all 1-dimensional complex representations of $G$.

(b) How many (non-isomorphic) irreducible complex representations does $G$ have and
what are their dimensions?

(c) Determine the character table of $G$.






\end{document}