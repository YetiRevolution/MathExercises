\documentclass[11pt]{article}
\usepackage[margin=1in]{geometry}
\usepackage{amssymb,amsmath,amsthm,amscd,url,hyperref}



\newcommand{\Z}{\mathbb{Z}}
\newcommand{\Q}{\mathbb{Q}}
\newcommand{\R}{\mathbb{R}}
\newcommand{\F}{\mathbb{F}}
\newcommand{\C}{\mathbb{C}}

\DeclareMathOperator{\End}{End}

\begin{document}
\begin{center}
\Large 2015 Algebra Prelim\\
\normalsize September 14, 2015
\end{center}
\vspace{1em}

INSTRUCTIONS: Do as many of the eight problems as you can. Four completely
correct solutions will be a pass; a few complete solutions will count more than many
partial solutions. Always carefully justify your answers. If you skip a step or omit
some details in a proof, point out the gap and, if possible, indicate what would be
required to fill it in\\
\vspace{1em}

1. Let $f(x)$ be an irreducible polynomial of degree 5 over the field $\Q$ of rational
numbers with exactly 3 real roots.

(a) Show that $f(x)$ is not solvable by radicals.

(b) Let $E$ be the splitting field of $f$ over $\Q$. Construct a Galois extension $K$ of
degree 2 over $\Q$ lying in $E$ such that no field $F$ strictly between $K$ and $E$ is
Galois over $\Q$.\\

2. Let $F$ be a finite field. Show for any positive integer $n$ that there are irreducible
polynomials of degree $n$ in $F[x]$.\\

3. Show that the order of the group $GL_n(\F_q)$ of invertible $n\times n$ matrices over the field $\F_q$ of $q$ elements is given by $(q^n -1)(q^n-q)\cdots (q^n-q^{n-1})$. \\

4. By looking at degrees of polynomials, show that any $\C$-subalgebra of the ring
$\C[x]$ of polynomials in one variable over the complex field $\C$ is finitely generated.\\

5. (a) Let $R$ be a commutative principal ideal domain. Show that any $R$-module $M$
generated by two elements takes the form $R/(a)\oplus R/(b)$ for some $a, b\in R$.
What more can you say about $a$ and $b$?

(b) Give a necessary and sufficient condition for two direct sums as in part (a) to
be isomorphic as $R$-modules.\\

6. Let $G$ be the subgroup of $GL_3(\C)$ generated by the three matrices\[
A = \begin{pmatrix}
0&0&1\\0&1&0\\1&0&0
\end{pmatrix}, \quad B= \begin{pmatrix}
0&0&1\\1&0&0\\0&1&0
\end{pmatrix}, \quad C = \begin{pmatrix}
i&0&0\\0&1&0\\0&0&1
\end{pmatrix}
\]
where  $i^2=-1$. Here $\C$ denotes the complex field. 

(a) Compute the order of $G$.

(b) Find a matrix in $G$ of largest possible order (as an element of $G$) and compute this order.

(c) Compute the number of elements in $G$ with this largest order. \\


7. (a) Let $G$ be a group of (finite) order $n$. Show that any irreducible left module
over the group algebra $\C G$ has complex dimension at most $\sqrt{n}$.

(b) Give an example of a group $G$ of order $n\ge 5$ and an irreducible left module over $\C G$ of complex dimension $\lfloor \sqrt{n}\rfloor$, the greatest integer to $\sqrt{n}$. \\

8. Use the rational canonical form to show that any square matrix $M$ over a field
$k$ is similar to its transpose $M^t$, recalling that $p(M) = 0$ for some $p\in k[t]$ if and only if $p(M^t) = 0$. 










\end{document}