\documentclass[11pt]{article}
\usepackage[margin=1in]{geometry}
\usepackage{amssymb,amsmath,amsthm,amscd,url,hyperref}



\newcommand{\Z}{\mathbb{Z}}
\newcommand{\Q}{\mathbb{Q}}
\newcommand{\R}{\mathbb{R}}
\newcommand{\F}{\mathbb{F}}
\newcommand{\C}{\mathbb{C}}

\DeclareMathOperator{\End}{End}

\begin{document}
\begin{center}
\Large 2015 Algebra Prelim\\
\normalsize September 14, 2015
\end{center}
\vspace{1em}

INSTRUCTIONS: Do as many of the eight problems as you can. Four completely
correct solutions will be a pass; a few complete solutions will count more than many
partial solutions. Always carefully justify your answers. If you skip a step or omit
some details in a proof, point out the gap and, if possible, indicate what would be
required to fill it in\\
\vspace{1em}

The letters $k$ and $K$ always denote fields.

1. Let $K$ be a field of characteristic zero and $L$ a Galois extension of $K$.
Let $f$ be an irreducible polynomial in $K[x]$ of degree 7 and suppose $f$
has no zeroes in $L$. Show that f is irreducible in $L[x]$.\\

2. Let $K$ be a field of characteristic zero and $f \in K[x]$ an irreducible
polynomial of degree $n$. Let $L$ be a splitting field for $f$. Let $G$ be the
group of automorphisms of $L$ which act trivially on $K$.

(a) Show that $G$ embeds in the symmetric group $S_n$.

(b) For each $n$, give an example of a field $K$ and polynomial $f$ such
that $G = S_n$.

(c) What are the possible groups $G$ when $n = 3$? Justify your answer.\\

3. Show there are exactly two groups of order 21 up to isomorphism.\\

4. (a) Show that the ring $\Z[i]$ of Gaussian integers is a unique factorisation
domain (UFD).

(b) Is $\Z[\sqrt{-5}]$ a UFD? Justify your answer.\\

5. Let $A$ be a domain and $K$ its field of fractions. Recall that we say
$f \in K$ is \emph{integral over} $A$ if it satisfies an equation\[
f^n + a_{n-1}f^{n-1} +\cdots + a_1f + a_0 = 0
\]
where $a_{n-1},\ldots,a_0\in A$.  The integral closure $\tilde A \subset K$  of $A$ is the set of
$f\in K$ which are integral over $A$, and we say $A$ is integrally closed if
$\tilde A = A$.

(a) Show that a UFD is integrally closed. (Hint: write $f$ as a fraction.)

(b) Compute the integral closure of $k[x,y]/(x^2-y^3)$. (Remember that
a polynomial ring is a UFD and therefore integrally closed.)

(c) Compute the integral closure of $k[x,y,z]/(x^2-y^2z)$. (Hint: there
is an obvious integral element.)\\

6. Let $V$ be a finite dimensional vector space over $\Q$ and $A: V \to V$ a
linear map such that $A^7 = \mbox{id}$, the identity map. Suppose that 1 is not
an eigenvalue of $A$. Prove that $\dim V$ is divisible by 6.\\

7. Let $V$ be a vector space over a field $k$ that is not of characteristic two.
Let $\omega:V\times V \to k$ be a non-degenerate skew-symmetric bilinear form,
i.e., $\omega(x, y) = -\omega (y, x)$ for all $x, y \in V$ , and if $x\neq 0$ there is a $y$ such
that $\omega(x, y) \neq  0$.

(a) Show that there exists a basis $e_1, . . . , e_n, f_1, . . . , f_n$ of $V$ such that $\omega(e_i,e_j) = \omega(f_i,f_j) = 0$ and $\omega(e_i,f_j) = \delta_{ij}$ for all $i,j.$ (In particular, $\dim V = 2n$ is even). 

(b) We say a subspace $W\subset V$ is isotropic if $\omega(w_1,w_2) = 0$ for all $w_1,w_2\in W$. Show that the dimension of an isotropic subspace is
at $\frac{1}{2}\dim V$.\\

8. Let $\mathbb H$ be the ring of quaternions with standard basis $1, i, j, k$ and identify
$\C$ with the subring $\R + \R i$ of  $\mathbb H$.

(a) Use the action of $\mathbb H$ on itself by left multiplication to explain why
there is a ring homomorphism $\phi: \mathbb H \to M_2(\C)$, where $M_2(\C)$
denotes the ring of $2\times 2$ matrices. (Warning: there are two ways to
view $\mathbb H$ as a $\C$-vector space, through right and left multiplication
by elements in the subring $\R + \R i$.)

(b) Say why $\phi$  is injective.

(c) The special unitary group $SU(2)$ consists of all $ 2\times 2$ complex
matrices $u$ such that $\det(u) = 1$ and $uu^* = u^*u = 1$ where $u^*$
is the conjugate transpose, i.e., the transpose of the matrix whose
entries are the complex conjugates of the entries in $u$. Show that $\phi$
restricts to an isomorphism between the group of unit quaternions
(those of length one) and $SU(2)$.

(d) Use this to prove that $SU(2)$ acts transitively on the Riemann
sphere $\mathbb{CP}^1$ defined as the 1-dimensional subspaces in $\C^2$. (Hint:
use the action of $M_2(\C)$ on $\C^2$  by left multiplication.)

(e) Let $U(1)$ denote the image in $SU(2)$ of the multiplicative subgroup
of $\C\setminus \{0\}$ consisting of the complex numbers $z\in \C\subset \mathbb H$  of length
one. Show that the coset space $SU(2)/U(1)$ is isomorphic to the
2-sphere $S^2$. \\

Remark. The solution to this problem gives a realization of the
Hopf fibration $S^3\to S^2$ with fibers $S^1$ because the group of unit
quaternions is isomorphic to the $3$-sphere $S^3$.
.







\end{document}