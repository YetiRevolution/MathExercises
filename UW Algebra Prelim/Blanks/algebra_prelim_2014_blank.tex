\documentclass[11pt]{article}
\usepackage[margin=1in]{geometry}
\usepackage{amssymb,amsmath,amsthm,amscd,url,hyperref}



\newcommand{\Z}{\mathbb{Z}}
\newcommand{\Q}{\mathbb{Q}}
\newcommand{\R}{\mathbb{R}}
\newcommand{\F}{\mathbb{F}}
\newcommand{\C}{\mathbb{C}}

\DeclareMathOperator{\End}{End}

\begin{document}
\begin{center}
\Large 2015 Algebra Prelim\\
\normalsize September 14, 2015
\end{center}
\vspace{1em}

INSTRUCTIONS: Do as many of the eight problems as you can. Four completely
correct solutions will be a pass; a few complete solutions will count more than many
partial solutions. Always carefully justify your answers. If you skip a step or omit
some details in a proof, point out the gap and, if possible, indicate what would be
required to fill it in\\
\vspace{1em}


1. (a) Let $G$ be a group (not necessarily finite) that contains a subgroup of index $n$.
Show that $G$ contains a normal subgroup $N$ such that $n \le  [G : N] \le  n!$.

(b) Use part (a) to show that there is no simple group of order 36.\\


2. Let $p$ be a prime, let $\F_p$ be the $p$-element field, and let $K = F_p(t)$ be the field of rational
functions in $t$ with coefficients in $\F_p$. Consider the polynomial $f(X) = X^p - t \in K[X]$.

(a) Show that $f$ does not have a root in $K$.

(b) Let $E$ be the splitting field of $f$ over $K$. Find the factorization of $f$ over $E$.

(c) Conclude that $f$ is irreducible over $K$.\\

3. Recall that a ring $A$ is called graded if it admits a direct sum decomposition $A = \bigoplus_{n=0}^\infty A_n$ as abelian groups, with the property that $A_iA_j \subseteq A_{i+j}$
for all $i, j \ge 0$.

Prove that a graded commutative ring $A = \bigoplus_{n=0}^\infty A_n$ is Noetherian if and only if $A_0$ is
Noetherian and $A$ is finitely generated as an algebra over $A_0$.\\

4. Let $R$ be a ring with the property that $a^2 = a$ for all $a\in R$.

(a) Compute the Jacobson radical of $R$.

(b) What is the characteristic of $R$?

(c) Prove that $R$ is commutative.

(d) Prove that if $R$ is finite, then $R$ is isomorphic (as a ring) to $\Z/2\Z)^d$ for some $d$.\\

5. Let $R$ be a commutative ring and let $M$ be an $R$-module.

(a) Let $x \in R$ be a nonzero divisor. Compute $\mbox{Tor}_i^R(R/(x), M)$  for $i \ge 0$.

(b) Show that $M$ is a flat $R$-module if and only if $\mbox{Tor}^R_1(M,N) = 0$ for all $R$-modules $N$.

(c) Conclude that if $R$ is a PID and $M$ is a finitely generated $R$-module, then $M$ is
flat if and only if $M$ is free.\\

6. Let $\overline{\F_p}$ denote the algebraic closure of $\F_p$. Show that the Galois group $\mbox{Gal}(\overline{\F_p}/F_p)$ has
no nontrivial finite subgroups.\\

7. Let $C_p$ denote the cyclic group of prime order $p$.

(a) Show that $C_p$ has two irreducible representations over $\Q$ (up to isomorphism),
one of dimension 1 and one of dimension $p- 1$.

(b) Let $G$ be a finite group, and let $\rho : G\to GL_n(\Q)$ be a representation of $G$
over $\Q$. Let $ρ_\C : G \to GL_n(\C)$ denote $\rho$ followed by the inclusion $GL_n(\Q) to 
GL_n(\C)$. Thus $\rho_\C$ is a representation of $G$ over $\C$, called the complexification of
$\rho$. We say that an irreducible representation $\rho$ of $G$ is absolutely irreducible if its
complexification remains irreducible over $\C.$

Now suppose $G$ is abelian and that every representation of $G$ over $\Q$ is absolutely
irreducible. Show that $G \cong  (C_2)^k$
for some $k$ (i.e., is a product of cyclic groups
of order 2).\\

8. Let $G$ be a finite group and $\Z[G]$ the integral group algebra. Let $\mathcal Z$ be the center of
$\Z[G]$. For each conjugacy class $C\subset G$, let $P_C = \sum_{g\in C} g$. 

(a) Show that the elements $P_C$ form a $\Z$-basis for $\mathcal Z$. Hence $\mathcal Z\cong \Z^d$ as an abelian
group, where $d$ is the number of conjugacy classes in $G$.

(b) Show that if a ring $R$ is isomorphic to $\Z^d$ as an abelian group, then every element
in $R$ satisfies a monic integral polynomial. (\textbf{Hint}: Let $\{v_1, . . . , v_d\}$ be a basis of
$R$ and for a fixed non-zero $r\in R$, write $rv_i = \sum_j a_{ij} v_j$
. Use the Hamilton-Cayley
theorem.)

(c) Let $\pi : G \to GL(V )$ be an irreducible representation of $G$ (over $\C$). Show that
$\pi(P_C)$ acts on $V$ as multiplication by the scalar\[
\frac{|C|\chi_\pi(C)}{\dim V},\]
where $\chi_\pi(C)$  is the value of the character $\chi_\pi$ on any element of $C$.

(d) Conclude that $|C|\chi_\pi(C)/ \dim V$ is an algebraic integer.


\end{document}