\documentclass[11pt]{article}
\usepackage[margin=1in]{geometry}
\usepackage{amssymb,amsmath,amsthm,amscd,url,hyperref}



\newcommand{\Z}{\mathbb{Z}}
\newcommand{\Q}{\mathbb{Q}}
\newcommand{\R}{\mathbb{R}}
\newcommand{\F}{\mathbb{F}}
\newcommand{\C}{\mathbb{C}}

\DeclareMathOperator{\End}{End}

\begin{document}
\begin{center}
\Large 2015 Algebra Prelim\\
\normalsize September 14, 2015
\end{center}
\vspace{1em}

INSTRUCTIONS: Do as many of the eight problems as you can. Four completely
correct solutions will be a pass; a few complete solutions will count more than many
partial solutions. Always carefully justify your answers. If you skip a step or omit
some details in a proof, point out the gap and, if possible, indicate what would be
required to fill it in\\
\vspace{1em}

1. Let $\Q^\times$
be the nonzero elements of $\Q$, a group under multiplication.

(a) Prove that the additive group of $\Q$ has no maximal proper subgroups.

(b) Is the same statement true for the multiplicative group $\Q^\times$ ?\\

2. Let $V$ be a finite-dimensional vector space over a field $F$ of characteristic 0.
Let $B : V \times V \to F$ be a non-degenerate, skew-symmetric bilinear form. (In particular, we
have $B(x, y) = −B(y, x)$ for all $x, y \in V$ .) If $U$ is a subset of $V$ , let\[
U^\perp = \{v\in V \mid B(u,v) = 0 \text{ for all } u\in U\}.
\]

(a) Let $U$ be a subspace of $V$ . Prove that $U^\perp$ is a subspace of $V$ and that \[\dim_F(U) + \dim_F(U^\perp) = \dim_F(V).\]

(b) Prove that there exists a subspace $W$ of $V$ such that $W^\perp = W$. \\

3. (a) Suppose that $G$ is a finitely-generated group. Let $n$ be a positive integer. Prove that
$G$ has only finitely many subgroups of index $n$.

(b) Let $p$ be a prime number. If $G$ is any finitely-generated abelian group, let $t_p(G)$ denote
the number of subgroups of $G$ of index $p$. Determine the possible values of $t_p(G)$ as $G$ varies
over all finitely-generated abelian groups.\\
\newpage
4. Suppose that $G$ is a finite group of order 2013. Prove that $G$ has a normal
subgroup $N$ of index 3 and that $N$ is a cyclic group. Furthermore, prove that the center of
$G$ has order divisible by 11. ( You will need the factorization $2013 = 3 \cdot  11 \cdot 61$.)\\\\\textbf{Solution:}\\
\emph{Note: Dummit and Foote section 5.5 is of some relevance to this problem.}\\\\
Since $2013 = 3\cdot 11 \cdot 61$, we know that there exist Sylow subgroups $H$ and $K$ of order $11$ and $61$ respectively. We claim that $K$ is in fact unique and hence normal. By Sylow's theorem we know that the number of Sylow 61-subgroups is congruent to 1 modulo 61, and the number of such subgroups divides $3\cdot 11 = 33$. The only possibility is that $K$ is the unique Sylow 61-subgroup and is normal in $G$. This implies that $HK$ is a subgroup of $G$ with order $11\cdot 61$, and hence index 3. Since 3 is the smallest prime dividing $G$ any subgroup of index 3 is normal, so $HK$ is normal.

To prove that $HK$ is cyclic, we recognize that its order is the product of two primes and hence it is a semidirect product of the cyclic groups $H$ and $K$. Such a semidirect product arises from a homomorphism $\phi:H\to \mbox{Aut}(K)$. Since $K$ has order 61 we have that $\mbox{Aut}(K) \cong \Z/60\Z$. But $|H| = 11$ does not 	divide 60, so the only homomorphism $\phi$ is the zero homomorphism, giving us that $HK \cong H\times K$. Since $HK$ is a direct product of cyclic groups with relatively prime orders it must be cyclic itself.

We next prove the center of $G$ has order divisible by 11. Letting $H'$ be a cyclic subgroup of $G$ with order 3, we recognize that since $HK$ is normal in $G$ we have $G\cong HK \rtimes H'$. To see that the center of $G$ has order divisible by 11, it suffices to show that $H$ (which has order 11) is in the center of $G$. Note that $H$ is in the center of $HK$ since we have already argued that $HK$ is cyclic. Thus it suffices to show that the elements of $H$ and $H'$ commute. First we prove that $H$ is normal in $G$. By Sylow's theorem we know that the number of Sylow 11-subgroups is congruent to 1 modulo 11. Moreover this number divides $3\cdot 61$. But neither 3, 61, nor $3\cdot 61 = 183$ are congruent to 1 modulo 11, so $H$ is the unique Sylow 11-subgroup of $G$ and hence normal.

Since $H$ is normal in $G$, $H'$ acts on $H$ by conjugation. This action arises from a homomorphism $H'\to \mbox{Aut}(H)$. But $\mbox{Aut}(H)$ has order 10, while $H'$ has order 3. Since 3 does not divide 10, the only such homomorphism is trivial, and $H'$ acts trivially by conjugation on $H$. In particular all elements of $H'$ commute with elements of $H$, and so $H\le Z(G)$. By Lagrange's theorem $Z(G)$ has order divisible by 11. 

\newpage
5. Let $V$ be a finite dimensional vector space over $\C$. Let $n = \dim_\C(V )$. Let
$T : V \to V$ be a linear map. Suppose that the following statement is true.

\emph{For every $c \in\C$, the subspace $\{v\in V\mid T(v) = cv\}$ of $V$ has dimension 0 or 1.}

Prove that there exists a vector $w \in V$ such that $\{w, T(w), \ldots, T^{n-1}
(w)\}$ is a linearly
independent set.\\\\\textbf{Solution:}\\
The condition on $T$ implies that $T$ has $n$ distinct eigenvalues $\lambda_1,\ldots, \lambda_n$ with associated eigenvectors $v_1,\ldots, v_n$ which form a basis for $V$. We claim that choosing $w = v_1+\cdots +v_n$ makes $\{w,T(w),\ldots, T^{n-1}(w)\}$ a linearly independent set. Note that \[
T^i(w) = \lambda_1^i v_1 + \cdots + \lambda_n^i v_n
\]
and so to argue that $\{w,T(w),\ldots, T^{n-1}(w)\}$ is linearly independent it suffices to argue that the matrix \[
\begin{bmatrix}
1 & 1 &\cdots & 1\\
\lambda_1 & \lambda_2 & \cdots & \lambda_n \\
\lambda_1^2 & \lambda_2^2 &\cdots & \lambda_n^2\\
\vdots &&\ddots & \vdots \\
\lambda_1^{n-1} & \lambda_2^{n-1} & \cdots & \lambda_n^{n-1}
\end{bmatrix}
\]
has linearly independent rows, i.e. that it is invertible. 
This is a Vandermonde matrix, and its determinant is given by \[
\prod_{1\le i< j \le n} (\lambda_j-\lambda_i).
\]
Since $\lambda_j \neq \lambda_i$ when $j\neq i$, we see that this determinant is nonzero and so the matrix is invertible. Hence $\{w,T(w),\ldots, T^{n-1}(w)\}$ forms a linearly independent set. 

\newpage
6. This question concerns an extension $K$ of $\Q$ such that $[K :\Q] = 8$. Assume
that $K/\Q$ is Galois and let $G = \mbox{Gal}(K/\Q)$. Furthermore, assume that $G$ is nonabelian.

(a) Prove that K has a unique subfield $F$ such that $F/\Q$ is Galois and $[F : \Q] = 4$.

(b) Prove that $F$ has the form $F =\Q(\sqrt{d_1},\sqrt{d_2})$ where $d_1$ and $d_2$ are nonzero integers.

(c) Suppose that $G$ is the quaternionic group. Prove that $d_1$ and $d_2$ are positive integers.\\\\
\textbf{Solution:}\\
(a)\\
Since the extension is Galois we know $|G| = 8$. Since $G$ is nonabelian it is isomorphic to either $D_8$ or $Q_8$. Each of these has a unique normal subgroup of order 2 (generated by $r^2$ and $-1$ respectively). By the Galois correspondence, these unique subgroups of order 2 (i.e. index 4) correspond to Galois extensions $F/\Q$ which have degree 4. \\\\
(b)\\
The Galois group of $F/\Q$ is $G/N$ where $N$ is the unique normal subgroup of index 4 described in (a). We claim that $G/N$ is isomorphic to $(\Z/2\Z)^2$. If $G\cong D_8$, then this is clear since we are taking a quotient which reduces the only elements of $G$ with order greater than 2 to elements of order 2 (these elements are $r$ and $r^3$). If $G\cong Q_8$ then we again are left with elements only of order 2 since $i^2 = j^2 = k^2 = -1$. Hence $G/N \cong (\Z/2\Z)^2$. 

Since $G/N \cong (\Z/2\Z)^2$ we may choose two subgroups of $G/N$ with index 2 which intersect trivially. Each of yields distinct quadratic extensions of $\Q$, which are necessarily of the form $\Q(\sqrt{d_1})$ and $\Q(\sqrt{d_2})$ with $d_1$ and $d_2$ integers. These extensions intersect trivially and hence their composite field is an extension of order 4, namely $F$. Thus we have $F = \Q(\sqrt{d_1},\sqrt{d_2})$. \\\\
(c)\\
Since $K$ is Galois and finite over $\Q$ it is the splitting field of some polynomial $f$. Since roots of $f$ come in conjugate pairs, if $f$ has complex roots then complex conjugation is an element of $G$. Complex conjugation is an automorphism of order 2, and if $G$ is the quaternionic group, then it has a unique element of order 2. Since $N$ (as descbribed in (b)) is the subgroup generated by this element we see that in $G/N$ complex conjugation reduces to the identity. Hence the fixed field of $N$ is totally real, i.e. $F$ is a real extension. This implies that $d_1$ and $d_2$ are positive. 
\newpage 

7. Let $R = \C[x_1, ..., x_n]$ be the polynomial ring over $\C$ in $n$ indeterminates
$x_1, ..., x_n$. Let $S_n$ be the $n$-th symmetric group. If $\sigma \in S_n$, then we can identify $\sigma $ with the
automorphism of $R$ defined as follows: $\sigma (c) = c$ for all $c \in\C$, and $\sigma (x_i) = x_{\sigma(i)}$
for all $i$,
$1 \le i \le n$. Suppose that $G$ is any subgroup of $S_n$. Let
 \[S = R^G = \{r\in R\mid \sigma(r) = r \text{ for all } \sigma\in G\}.
\]
Prove that $S$ is a finitely-generated $\C$-algebra.\\\\
\textbf{Solution:}\\
For every $\gamma\subseteq \{1,2,\ldots, n\}$ define \[
f_\gamma  = \sum_{\sigma\in G} \sigma\left(\prod_{i\in \gamma} x_i\right).
\]
That is, $f_\gamma$ is the sum of all elements in the orbit of the monomial $\prod_{i\in \gamma} x_i$ under the action of $G$ on $R$. We claim that the set of all $f_\gamma$ generate $R^G$ as an algebra. To see that this is the case, we show that every element of $R^G$ is a polynomial combinator of the various $f_\gamma$. We proceed by induction on the degree of $r\in R^G$. 
In the base case that $\deg(r) = 0$ the result follows trivially since $r$ is constant, and hence in any $\C$-algebra.

Otherwise let $\deg(r) = n >0$. Then write $r$ as a sum of monomials with coefficients from $\C$ and consider all degree $n$ monomials which appear in this sum. Since $r = \sigma(r)$ for all $\sigma\in G$ we see that 
\newpage
8. This question concerns the polynomial ring $R = \Z[x, y]$ and the ideal
$I = (5, x^2 + 2)$ in $R$.

(a) Prove that $I$ is a prime ideal of $R$ and that $R/I$ is a $PID$.

(b) Give an explicit example of a maximal ideal of $R$ which contains $I$.
(Give a set of generators for such an ideal.)

(c) Show that there are infinitely many distinct maximal ideals in $R$ which contain $I$.\\\\
\textbf{Solution:}\\
(a)\\
Note that \[
R/I = \Z[x,y]/(5,x^2+2) \cong (\Z/5\Z)[x,y]/(x^2+2).
\]
The polynomial $x^2+2$ is irreducible over $\Z/5\Z$ (having no roots) and so $\Z/5\Z[x]/(x^2+2)$ is a field. Let $F$ denote this field and observe that \[
R/I \cong F[y].
\]
This is a polynomial ring over a field, and so is a PID. Since $R/I$ is an integral domain we conclude that $I$ must be prime. \\\\
(b)\\
The ideal $J = (5,x^2+2, y)$ is maximal since the quotient by this ideal is just $F[y]/(y) \cong F$, a field.\\\\
(c)\\
For any prime $p\in\Z$ not equal to 5 let $J_p = (5,x^2+2, py)$. Clearly the various $J_p$ are distinct since the smallest positive integer $n$ for which $ny\in J_p$ is always $p$ and the various $p$ are distinct. Moreover, since $p\neq 5$ for $J_p$ we have $p$ and $5$ relatively prime so that $p$ is invertible mod 5. Thus $(py) = (y)$ as ideals in $F[y]$. We then have that $R/J_p \cong F[y]/(yp) = F[y]/(y) \cong F$, so $R/J_p$ is a field and $J_p$ is maximal. 


\end{document}