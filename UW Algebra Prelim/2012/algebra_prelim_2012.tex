\documentclass[11pt]{article}
\usepackage[margin=1in]{geometry}
\usepackage{amssymb,amsmath,amsthm,amscd,url,hyperref}



\newcommand{\Z}{\mathbb{Z}}
\newcommand{\Q}{\mathbb{Q}}
\newcommand{\R}{\mathbb{R}}
\newcommand{\F}{\mathbb{F}}
\newcommand{\C}{\mathbb{C}}

\DeclareMathOperator{\End}{End}

\begin{document}
\begin{center}
\Large 2015 Algebra Prelim\\
\normalsize September 14, 2015
\end{center}
\vspace{1em}

INSTRUCTIONS: Do as many of the eight problems as you can. Four completely
correct solutions will be a pass; a few complete solutions will count more than many
partial solutions. Always carefully justify your answers. If you skip a step or omit
some details in a proof, point out the gap and, if possible, indicate what would be
required to fill it in\\
\vspace{1em}

1. Classify all groups of order 2012 up to isomorphism. (Hint: 503 is prime.)\\
\\\textbf{Solution:}\\
Let $G$ be a group of order 2012. We have the prime factorization $2012 = 2^2\cdot 503$, and from Sylow's theorem we obtain a subgroup $H$ of order 503. The number of Sylow 503-subgroups is congruent to 1 modulo 503, and also divides $2^2=4$. This immediately implies that $H$ is the unique Sylow 503-subgroup, and is hence normal. We also have a subgroup $K$ of order 4, and since $H$ is normal we see that $G \cong H\rtimes K$. Thus the structure of $G$ is determined by the possible semidirect products between a group of order 4 and a group of order 503. 

Since 503 is prime we know $H$ is cyclic, and since $K$ has order 4 it is either cyclic or the Klein 4 group. Semidirect products $H\rtimes K$ will arise from a homomorphism $\phi:K\to \mbox{Aut}(H)$, and since $H$ is cyclic with prime order we know $\mbox{Aut}(H)$ is cyclic of order 502. We describe the possible homomorphisms $\phi$ in the cases that $K$ is cyclic or the Klein 4 group below. \begin{itemize}
\item If $K$ is cyclic and $\phi$ is trivial we obtain a direct product and $G\cong \Z_{503}\times \Z_4$. 
\item If $K$ is cyclic and $\phi$ is nontrivial we note that $502$ is divisible by 2 but not 4, and so there is only one possibility for $\phi$, namely that it maps the elements of order 4 in $K$ to the identity  and the element of order 2 to the element of $\mbox{Aut}(H)$ with order 2. This gives us a presentation of $G$ as $\langle a,b \mid a^4 = b^{503} = 1 \text{ and } a^2ba^2 = b^{-1}\rangle$.
\item if $K$ is the Klein 4 group ($\Z_2\times \Z_2$) and $\phi$ is trivial we obtain a direct product and $G\cong \Z_{503}\times \Z_2\times \Z_2$.
\item If $K$ is the Klein 4 group and $\phi$ is nontrivial, we again have a single possibility, up to isomorphism. This arises from mapping one of the generators of $K$ to the element of $\mbox{Aut}(H)$ with order 2, and the other to the identity. This gives us a presentation $G\cong \langle a,b,c \mid a^2 = b^2 = c^{503} = 1\text{ and } ab = ba \text{ and } aca = c^{-1} \text{ and } bcb = c\rangle$. 
\end{itemize}



\newpage


2. For any positive integer $n$, let $G_n$ be the group generated by $a$ and $b$ subject to the following three
relations:\[
a^2=1,\quad b^2=1,\quad\text{and}\quad (ab)^n=1.
\]

(a) Find the order of the group $G_n$.

(b) Classify all irreducible complex representations of $G_4$ up to isomorphism.\\\\
\textbf{Solution:}\\
(a)\\
We claim that the order is $2n$. One way to see this is to recognize that $G_n$ is in fact the dihedral group with $a = s$ and $b = sr$, but we take a slightly more direct approach. First we argue that every element of $G_n$ can be presented as either $(ab)^k$ or $a(ab)^k$  by reducing modulo the relation $a^2=b^2=1$. Moreover we have $k\le n$ by the relation $(ab)^n = 1$. This clearly gives us $2n$ possible words of the letters $a$ and $b$ which are in $G$.

Our next task is to prove that these $2n$ words are all distinct. Note that if $(ab)^k = (ab)^{k'}$ then we have $(ab)^{k-k'} = 1$ and so $k-k' = 0$ if we assume $k$ and $k'$ are both between $0$ and $n-1$. This tells us that words of the form $(ab)^k$ are all distinct from one another. Likewise, words of the form $a(ab)^k$ are all distinct, cancelling $a$ from both sides of the same equality. The only other possibility is that $a(ab)^k = (ab)^{k'}$ for some $k$ and $k'$ between 0 and $n-1$. By cancelling appropriately we obtain an expression of the form $b(ab)^m = 1$ for a nonnegative integer $m$. But we can multiply both sides of this expression on the left and right by $b$ to obtain $a(ba)^{m-1} = b^2 = 1$. We can then multiply by $a$ on the left and right, yielding $b(ab)^{m-2}  = 1$ and so on. Repeating this process eventually yields $a=1$ or $b=1$, in either case a contradiction. We conclude that the $2m$ words of the form $a(ab)^k$ with $0\le k \le n-1$ are distinct, and $G_n$ contains $2n$ elements.\\\\
(b)\\
First, we know that $G_4$ has 8 elements by part (a). Second, we know it has at least one 1-dimensional representation, the trivial representation. We also see that $G_4$ is nonabelian by considering the elements $a$ and $ab$. We have \[
(ab)a = (ab)a(ab)^4 = (ab)b(ab)^3 = a(ab)^3 \neq a(ab)
\]
since by part (a) the representation of elements of $G_4$ of the form $a(ab)^k$ are unique. Hence not all irreducible representations of $G_4$ will be 1-dimensional. The sum of squares of dimensions of irreducible representations of $G_4$ will be equal to 8, and so there will be exactly one 2-dimensional irrep and none of higher dimension. We conclude that there are five irreps of $G_4$, four with dimension 1 and one with dimension 2. We classify these below.

The 1-dimensional representations of $G_4$ are homomorphisms $G_4\to \C^{\times}$. We see that $a$ and $b$ must map to either $\pm 1$ since they have order 2, and there are clearly four possibilities for mapping $a$ and $b$ to $\pm 1$. Each of these possibilities yields a distinct 1-dimensional representation of $G_4$ and since we have already concluded there are four total 1-dimensional representations this accounts for everything.

Next we consider the 2-dimensional irrep $V$ of $G_4$. Let $\{v_1,v_2\}$ be a basis for $V$. If we can find an action of $a$ and $b$ on $V$ which is not commutative we will be done, since such an action cannot be decomposed as a direct sum of 1-dimensional representations. Let $a$ act by switching $v_1$ and $v_2$, and let $b$ act by fixing $v_1$ while mapping $v_2\mapsto -v_2$. Note that $a^2$ and $b^2$ both act as identity, satisfying the relations $a^2=b^2=1$ in $G_4$. The element $ab$ acts by mapping $v_1\mapsto v_2$ and $v_2\mapsto -v_1$. One can verify that this action has order exactly 4 and so satisfies the relation $(ab)^4$. This proves that the described action of $a$ and $b$ on $V$ is a valid representation of $G_4$. Note that $a(ab) = b$ does not fix $v_2$, but $(ab) a$ does, and so this action of $G_4$ is not commutative. We conclude that this representation is irreducible. 

The following list summarizes all irreps of $G_4$:\begin{itemize}
\item The trivial representation, in which $a,b$ act as identity on $\C$.
\item The representation in which $a$ acts by negation on $\C$ and $b$ acts as identity.
\item The representation in which $a$ acts as identity on $\C$ while $b$ acts by negation.
\item The representation in which $a$ and $b$ both act by negation on $\C$.
\item The representation $V = \C^2$ with basis $\{v_1,v_2\}$ in which $a$ switches $v_1$ and $v_2$, while $b$ fixes $v_1$ and negates $v_2$. Intuitively, this corresponds to the geometric action of the dihedral group of order 8 (i.e. $G_4$) on a square embedded in the plane. 
\end{itemize}
\newpage
3. Let $R$ be a (commutative) principal ideal domain, let $M$ and $N$ be finitely generated free $R$-modules,
and let $\phi : M \to N$ be an $R$-module homomorphism.

(a) Let $K$ be the kernel of $\phi$. Prove that $K$ is a direct summand of $M$.

(b) Let $C$ be the image of $\phi$. Show by example (specifying $R$, $M$, $N$ and $\phi$) that $C$ need not be a direct summand of $N$.\\\\
\textbf{Solution:}\\
(a)\\
Note that $\phi(M)$ is a submodule of a free module over a PID, and hence free. In particular, $\phi(M)$ is projective and so the short exact sequence $0\to \ker\phi\to M\to \phi(M) \to 0$ splits and we have $M \cong \ker \phi \oplus \phi(M)$ and $\ker \phi$ is a direct summand of $M$. \\\\
(a) \emph{(A more direct proof)}\\
Note that $\phi(M)$ is a free module by virtue of being a finitely generated torsion free module over a PID. Let $\{e_1,\ldots, e_n\}$ be a basis for $\phi(M)$ over $R$ and for $1\le i \le n$ let $e_i'$ be an element of $M$ so that $\phi(e_i') = e_i$. Letting $M'$ be the submodule of $M$ generated by $\{e_1',\ldots, e_n'\}$ we claim that $M = \ker \phi \oplus M'$. Note immediately that $\ker \phi$ and $M'$ intersect trivially since no $e_i'$ is in the kernel of $\phi$. 

To see that $M =\ker\phi \oplus M'$ it then suffices to show that every element of $M$ can be written as $k+m'$ for $k\in \ker\phi$ and $m'\in M'$. By the universal property of free modules the map $e_i\mapsto e_i'$ can be extended to an $R$-module homomorphism $\psi:\phi(M)\to M$. For any $m\in M$ write $m = (m-\psi(\phi(m)) + \psi(\phi(m))$. Note that $\phi\circ \psi$ is the identity on $\phi(M)$, and so we have \[
\phi(m-\psi(\phi(m))) = \phi(m) - \phi(m) = 0
\]
i.e. $(m-\psi(\phi(m))$ is in the kernel of $\phi$. We also have $\psi(\phi(m)) \in M'$, and so we have written $m$ in the form $k+m'$ as desired. This proves that $M = \ker \phi \oplus M'$ and $\ker\phi$ is a direct summand of $M$ as desired.\\\\
(b)\\
Let $R = M = N = \Z$, and consider the map $x\mapsto 2x$. Then we have that $C = 2\Z$. But $2\Z$ is a proper submodule of $\Z$ and it also intersects every nonzero submodule of $\Z$ nontrivially. Hence it is not a direct summand.


\newpage

4. Let $G$ be an abelian group. Prove that the group ring $\Z[G]$ is noetherian if and only if $G$ is finitely
generated.\\

5. Let $M_3(\R)$ be the $3\times 3$-matrix algebra over the real numbers $\R$. For any $b\in \R$ let $B\in M_3(\R)$ be the matrix $\begin{pmatrix}1&b&0\\b&1&b\\0&b&1\end{pmatrix}$. Find the set of numbers $b$ so that the matrix equation $X^2 = B$ has at least one, and only finitely many, solutions in $M_3(\R)$. \\
\newpage
6. Determine the Galois groups of the following polynomials over $\Q$.

(a) $f(x) = x^4+4x^2+1$

(b) $f(x) = x^4 +4x^2 -5$\\\\
\textbf{Solution:}\\
(a)\\
 The roots of $f(x)$ are $\pm \sqrt{-2\pm \sqrt{3}}$. Let $\alpha = \sqrt{-2+\sqrt{3}}$ and $\beta = \sqrt{-2-\sqrt{3}}$ so that the roots of $f$ are $\pm \alpha$ and $\pm \beta$. We claim that the splitting field of $f$ is $K = \Q(\alpha)$. Note that this is certainly contained in the splitting field of $f$ since it is a simple extension by a root of $f$. To see that this extension contains all roots of $f$, note that  \[
\alpha^{-1} = \left(\sqrt{-2+\sqrt{3}}\right)^{-1} = \sqrt{-2-\sqrt{3}} = \beta
\]
and so $K$ contains $\pm\alpha$ and $\pm \beta$. Now, $f(x)$ is irreducible over $\Q$ by the rational roots theorem, and so this simple extension has degree four. We conclude that the Galois group of $f$ has order 4, and is isomorphic to either $\Z/4\Z$ or $\Z/2\Z \times \Z/2\Z$. 

We claim that the Galois group is in fact $\Z/2\Z\times \Z/2\Z$. To prove this it suffices to exhibit two automorphisms of $K$ over $\Q$ with order two. One is given by $\alpha \mapsto -\alpha$ and $\beta\mapsto -\beta$. Since $f$ is irreducible there must also be an automorphism $\phi$ sending $\alpha \mapsto \beta$. Note that for this automorphism we must have \[
\phi(\beta) = \phi(\alpha^{-1}) = \phi(\alpha)^{-1} = \beta^{-1} = \alpha
\]
and so $\phi$ has order two. Hence the Galois group of $f$ is $\Z/2\Z\times\Z/2\Z$.\\\\
(b)\\
The four roots of $f$ in this case are $\pm \sqrt{-2\pm 3}$, i.e. $\pm 1$ and $\pm\sqrt{-5}$. The splitting field of $f$ is then simply $\Q(\sqrt{-5})$, a quadratic extension. This implies that the Galois group of $f$ is simply $\Z/2\Z$, with its nonidentity permutation mapping $\sqrt{-5}\mapsto -\sqrt{-5}$.  


\newpage

7. Prove that if $A$ is a finite abelian group, then $\mbox{Hom}_\Z(A,\Q/\Z)\cong \mbox{Ext}_\Z^1(A,\Z) \cong A$. (Here $\mbox{Ext}_\Z^1(-,-)$ is also sometimes written as $\mbox{Ext}(-,-)$. \\

8. Let $A$ be the $\C$-algebra $\C[x,y]$, and define algebra automorphisms $\sigma$ and $\tau$ of $A$ by \[
\sigma(x)= y, \quad \sigma(x) = y
\]
and \[
\tau(x) = x,\quad \tau(y) = \zeta y,
\]
where $\zeta\in \C$ is a primitive third root of unity (namely, $\zeta \neq 1$ and $\zeta^3 = 1$). Let $G$ be the group of algebra automorphisms of $A$ generated by $\sigma$ and $\tau$. Define \[
A^G = \{f\in A \mid \phi(f) = f\text{ for all } \phi\in G\}.
\]
Then $A^G$ is a subalgebra of $A$ -- you do not need to prove this. Describe the algebra $A^G$ by finding a set of generators and a set of relations. 


\end{document}