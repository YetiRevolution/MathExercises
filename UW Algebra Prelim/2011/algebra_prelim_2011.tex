\documentclass[11pt]{article}
\usepackage[margin=1in]{geometry}
\usepackage{amssymb,amsmath,amsthm,amscd,url,hyperref}



\newcommand{\Z}{\mathbb{Z}}
\newcommand{\Q}{\mathbb{Q}}
\newcommand{\R}{\mathbb{R}}
\newcommand{\F}{\mathbb{F}}
\newcommand{\C}{\mathbb{C}}

\DeclareMathOperator{\End}{End}

\begin{document}
\begin{center}
\Large 2015 Algebra Prelim\\
\normalsize September 14, 2015
\end{center}
\vspace{1em}

INSTRUCTIONS: Do as many of the eight problems as you can. Four completely
correct solutions will be a pass; a few complete solutions will count more than many
partial solutions. Always carefully justify your answers. If you skip a step or omit
some details in a proof, point out the gap and, if possible, indicate what would be
required to fill it in\\
\vspace{1em}

1. Let $GL_2(\C)$ be the general linear group of $2\times 2$ complex matrices, let $H$ be the subgroup
of $GL_2(\C)$ consisting of non-zero multiples of the identity matrix, and let $PGL_2(\C)$ be
the quotient group $GL_2(\C)/ H$.

Let $A, B\in PGL_2(\C)$, and assume that both elements have order $n$. Prove that there
exist $C\in PGL_2(\C)$ and a positive integer $m$ such that\[
CBC^{-1} = A^m.\]

2. In this problem, as you apply Sylow’s Theorem, state precisely which portions you are
using.

(a) Prove that there is no simple group of order 30.

(b) Suppose that $G$ is a simple group of order $60$. Determine the number of $p$-Sylow
subgroups of G for each prime $p$ dividing 60, then prove that $G$ is isomorphic to
the alternating group $A_5$.

Note: In the second part, you needn’t show that $A_5$ is simple. You need only show
that if there is a simple group of order 60, then it must be isomorphic to $A_5$.\\

3. Describe the Galois group and the intermediate fields of the cyclotomic extension
$\Q(\zeta_{12})/\Q$.\\

4. Let \[
R = \Z[x]/(x^2+x+1).
\]

(a) Answer the following questions with suitable justification.\\
i. Is $R$ a Noetherian ring?\\
ii. Is $R$ an Artinian ring?

(b) Prove that $R$ is an integerally closed domain.\\



5. Let $R$ be a commutative ring. Recall that an element $r$ of $R$ is nilpotent if $r^n = 0$ 
for some positive integer $n$ and that the nilradical of $R$ is the set $N(R)$ of nilpotent
elements.

(a) Prove that
\[
N(R) = \bigcap_{P\text{ prime}} P.
\]
(Hint: Given a non-nilpotent element $r$ of $R$, you may wish to construct a prime
ideal that does not contain $r$ or its powers.)

(b) Given a positive integer $m$, determine the nilradical of $\Z/(m)$.

(c) Determine the nilradical of $\C[x, y]/(y^2-x^3)$.

(d) Let $p(x, y)$ be a polynomial in $\C[x, y]$ such that for any complex number $a$,
 $p(a,a^{3/2}) = 0$.  Prove that $p(x, y)$ is divisible by $y^2-x^3$. \\

6. Given a finite group $G$, recall that its regular representation is the representation on
the complex group algebra $\C[G]$ induced by left multiplication of $G$ on itself and its
adjoint representation is the representation on the complex group algebra $\C[G]$ induced
by conjugation of $G$ on itself.

(a) Let $G = GL_2(\F_2)$. Describe the number and dimensions of the irreducible representations
of $G$. Then describe the decomposition of its regular representation as
a direct sum of irreducible representations.


(b) Let $H$ be a group of order 12. Show that its adjoint representation is reducible;
that is, there is an $H$-invariant subspace of $\C[H]$ besides 0 and $\C[H]$.\\

7. Let $M, N$ be finitely generated modules over $\Z$. Recall that $\mbox{Ann}(M)$ is the ideal in $\Z$
defined as follows:
\[
\mbox{ann}(M) = \{a\in \Z \mid am = 0 \text{ for any } m\in M\}
\]
Prove that $M \otimes_\Z  N = 0$ if and only if $\mbox{Ann}(M) + \mbox{Ann}(N) = (1)$.\\

8. Let $R$ be a commutative integral domain. Show that the following are equivalent:
(a) $R$ is a field;
(b) $R$ is a semi-simple ring;
(c) Any $R$-module is projective.


\end{document}