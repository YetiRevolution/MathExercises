\setcounter{chapter}{9}
\begin{chapter}{Introduction to Module Theory}

\begin{section}{Basic Definitions and Examples}
In these exercises $R$ is a ring with 1 and $M$ is a left $R$-module.
\begin{problem}\label{ex:10.1.1}
Prove that $0m = 0$ and $(-1)m = -m$ for all $m\in M$. 
\end{problem}
\begin{solution}We have via straightforward application of the module axioms that \[
0m = (0-0)m = 0m-0m = 0.
\]
Likewise, we can compute that \[
(-1)m = -m+m+(-1)m =  -m + (1)m + (-1)m = -m +(1-1)m = -m-0m = -m.
\]

\end{solution}\oneperpage



\begin{problem}\label{ex:10.1.2}
Prove that $R^\times$ and $M$ satisfy the two axioms in Section 1.7 for a \emph{group action} of the multiplicative group $R^\times$ on the set $M$.
\end{problem}
\begin{solution}We know that $R^\times$ is a group, and by the module axioms we know $1\cdot m = m$ for all $m\in M$ and hence the identity acts on $M$ in accordance with a group action. We also have via the module axioms that $uv\cdot m = u\cdot(v\cdot m)$ for all $u,v\in R^\times$, and so the action of $R^\times$ satisfies both axioms of a group action.

\end{solution}\oneperpage



\begin{problem}\label{ex:10.1.3}
Assume that $rm=0$ for some $r\in R$ and some $m\in M$ with $m\neq 0$. Prove that $r$ does not have a left inverse (i.e., there is no $s\in R$ such that $sr = 1$).
\end{problem}
\begin{solution}Suppose otherwise, so that there exists $s\in R$ so that $sr=1$. Then we have that \[
m = (sr)m = s(rm) = s0 = 0
\]
a contradiction. 
\end{solution}\oneperpage



\begin{problem}\label{ex:10.1.4}
Let $M$ be the module $R^n$ described in Example 3 and let $I_1, I_2,\ldots, I_n$ be left ideals of $R$. Prove that the following are submodules of $M$:\begin{enumerate}\item[(a)] $\{(x_1,x_2,\ldots,x_n)\mid x_i\in I_i\}$\item[(b)] $\{(x_1,x_2,\ldots,x_n)\mid x_i\in R \text{ and } x_1+x_2+\ldots +x_n = 0\}$.\end{enumerate}
\end{problem}
\begin{solution}
(a)\\
The set  is clearly nonempty since $(0,0,\ldots,0)$ is in it. The second condition of the submodule criterion is also satisfied since \begin{align*}
(x_1,x_2,\ldots,x_n) + r(x_1',x_2',\ldots,x_n') = (x_1+rx_1',x_2+rx_2',\ldots, x_n+rx_n')
\end{align*}
for any $r\in R$ and $x_i+rx_i'\in I$ by virtue of $I$ being an ideal. Thus the set is a submodule.\\\\
(b)\\
As in (a) we notice that $(0,0,\ldots,0)$ is in the set, and so it is nonempty. Letting $x = (x_1,\ldots,x_n)$ and $y= (x_1',\ldots,x_n')$ be two elements of the set we have that $x+ry$ is in the set since \begin{align*}
(x_1+rx_1')+(x_2+rx_2') + \cdots + (x_n+rx_n') &= (x_1+x_2+\cdots+x_n)+r(x_1'+x_2'+\cdots+x_n')\\
& = 0+r0\\
&=0.
\end{align*}
Thus the set satisfies the submodule criterion and is a submodule.

\end{solution}\oneperpage



\begin{problem}\label{ex:10.1.5}
For any left ideal $I$ of $R$ define \[
IM = \{\sum_{\text{finite}}a_im_i\mid a_i\in I, m_i\in M\}
\]
to be the collection of all finite sums of elements of the form $am$ where $a\in I$ and $m\in M$. Prove that $IM$ is a submodule of $M$. 
\end{problem}
\begin{solution}
Note that $0_M \in IM$ since $0_R\in I$ and $0_M\in M$ so $0_M = 0_R\cdot 0_M \in IM$. Now let $x = \sum a_im_i$ and $y= \sum b_jm_j$ be two elements of $IM$. Then notice for any $r\in R$ that \begin{align*}
x+ry &= \sum a_im_i + \sum rb_jm_j
\end{align*}
which is again in $IM$ since both sums are finite and $r b_j\in I$ by virtue of $I$ being a left ideal. Thus $IM$ satisfies the submodule criterion and is a submodule.
\end{solution}\oneperpage



\begin{problem}\label{ex:10.1.6}
Show that the intersection of any nonempty collection of submodules of an $R$-module is a submodule.
\end{problem}
\begin{solution}
Let $M$ be an $R$-module and let $\{N_\alpha\}$ be an arbitrary collection of submodules of $M$. Let $N = \cap_{\alpha} N_\alpha$. Notice that $N$ is nonempty since each $N_\alpha$ must contain zero by virtue of being a subgroup over the overall module. Then let $x,y\in N$. Since each $N_\alpha$ is a submodule we have $x+ry\in N_\alpha$ for all $r\in R$ and all $\alpha$. We conclude that $x+ry\in N$ and so $N$ satisfies the submodule criterion. This proves the result.
\end{solution}\oneperpage



\begin{problem}\label{ex:10.1.7}
Let $N_1\subseteq N_2\subseteq \cdots$ be an ascending chain of submodules of $M$. Prove that $\cup_{i=1}^\infty N_i$ is a submodule of $M$.
\end{problem}
\begin{solution}
Let $N=\cup_{i=1}^\infty N_i$. Note that $0\in N$ so $N$ is nonempty. Then let $x,y\in N$. There must exist $N_i$ so that $x,y\in N_i$ and by virtue of $N_i$ being a submodule we will have $x+ry\in N_i$ for all $r\in R$ and hence $x+ry\in N$. This proves that $N$ is a submodule.

\end{solution}\oneperpage



\begin{problem}\label{ex:10.1.8}
An element $m$ of the $R$-module $M$ is called a \emph{torsion element} if $rm=0$ for some nonzero element $r\in R$. The set of torsion elements is denoted \[
\Tor(M) \od \{m\in M \mid rm=0 \text{ for some nonzero } r\in R\}.
\]
\begin{enumerate}
\item[(a)] Prove that if $R$ is an integral domain then $\Tor(M)$ is a submodule of $M$ (called the \emph{torsion} submodule of $M$).
\item[(b)] Give an example of a ring $R$ and an $R$-module $M$ such that $\Tor(M)$ is not a submodule. [Consider the torsion elements in the $R$-module $R$.]
\item[(c)] If $R$ has zero divisors show that every nonzero $R$-module has nonzero torsion elements.
\end{enumerate}
\end{problem}
\begin{solution}
(a)\\
Let $R$ be an integral domain and observe that $\Tor(M)$ is nonempty  since it contains zero. Then  let $x,y\in \Tor(M)$ and let $r_1,r_2\in R$ be nonzero so that $r_1x = 0$ and $r_2y = 0$. For an arbitrary $r\in R$ we can notice that \[
r_1r_2(x+ry) = r_1r_2x + r_1r_2ry = r_2r_1x + r_1rr_2y = r_2\cdot 0 + r_1r\cdot 0 = 0+0 = 0
\] where above we have used the commutativity of $R$. Furthermore observe that $r_1r_2$ is nonzero since $R$ is an integral domain, and so $x+ry\in \Tor(M)$. This proves that $\Tor(M)$ is a submodule by the submodule criterion.\\\\
(b)\\
Consider $\Z/6\Z$. The torsion elements of this ring as a module over itself are $\{0,2,3,4\}$ which do not even form an additive subgroup, much less a submodule.\\\\
(c)\\
Suppose $R$ has zero divisors and let $x,y\in R$ be nonzero so that $xy=0$. Then for some nonzero $m\in M$ consider $ym$. If $ym=0$ then $m$ is a nonzero torsion element. Otherwise $ym$ is a nonzero torsion element since $x(ym) = (xy)m = 0m = 0$. 

\end{solution}\oneperpage



\begin{problem}\label{ex:10.1.9}
If $N$ is a submodule of $M$, the \emph{annihilator of $N$ in $R$} is defined to be $\{r\in R\mid rn=0 \text{ for all } n\in N\}$. Prove that the annihilator of $N$ in $R$ is a 2-sided ideal of $R$. 
\end{problem}
\begin{solution}
Let $N$ be a submodule and let $I$ be its annihilator. Clearly $I$ contains 0 and so is nonempty. Furthermore if $a,b\in I$ then $a-b\in I$ since for any $n\in N$ we have \[
(a-b)n = an + (-b)n = 0 -(bn) = 0-0 = 0
\]
where above we have used the fact that $(-b)n = -(bn)$ which can be proved analogously to property 2 in Problem 1. Thus $I$ is an additive subgroup of $R$. 

Finally let $r\in R$ be arbitrary and let $a\in I$. Clearly $ra\in I$ since \[
ran = r(an) = r0 = 0
\]
for any $n\in N$. We also have $ar\in I$ since \[
arn = a(rn) = 0
\]
for any $n\in N$, where above we have used that $an\in N$. This proves that $I$ is a 2-sided ideal in $R$.
\end{solution}\oneperpage



\begin{problem}\label{ex:10.1.10}
If $I$ is a right ideal of $R$, the \emph{annihilator of $I$ in $M$} is defined to be $\{m\in M \mid am=0\text{ for all } a\in I\}$. Prove that the annihilator of $I$ in $M$ is a submodule of $M$.
\end{problem}
\begin{solution}
Let $I$ be a right ideal of $R$ and let $N$ be its annihilator. Notice immediately that $0\in N$ since $an = 0$ for all $a\in I$. Then let $n,n'\in N$ and $r\in R$. We have that \begin{align*}
a(n+rn') &= an + arn'\\
& = 0 + (ar)n'\\
& = 0+0\\
& = 0
\end{align*}
where above we have used that $ar\in I$ by virtue of $I$ being a right ideal. This proves that $N$ satisfies the submodule criterion, and so it is a submodule. 
\end{solution}\oneperpage



\begin{problem}\label{ex:10.1.11}
Let $M$ be the abelian group (i.e., $\Z$-module) $\Z/24\Z \times \Z/15\Z \times \Z/50\Z$. \begin{enumerate}
\item[(a)] Find the annihilator of $M$ in $\Z$ (i.e. a generator for this principal ideal).
\item[(b)] Let $I = 2\Z$. Describe the annihilator of $I$ in $M$ as a direct product of cyclic groups.
\end{enumerate}
\end{problem}
\begin{solution}
(a)\\
Notice that if $r\in \Z$ annihilates $M$ it must annihilate each coordinate. In particular, it must be a multiple of $24$, of $15$, and of $50$. This condition is both necessary and sufficient and so the annihilator of $M$ is $600\Z$, the ideal generated by the least common multiple of 24, 15, and 50.\\\\
(b)\\
The ideal $2\Z$ annihilates 0 and 12 in the first coordinate, 0 in the second coordinate, and 0 and 25 in the third coordinate. Hence the annihilator of $2\Z$ is the set \[
\{(0,0,0), (12,0,0), (0,0,25), (12,0,25)\}
\]
which as a direct product of cyclic groups is isomorphic to $\Z/2\Z \times \Z/2\Z$. 

\end{solution}\oneperpage



\begin{problem}\label{ex:10.1.12}
In the notation of the preceding exercises prove the following facts about annihilators.
\begin{enumerate}
\item[(a)] Let $N$ be a submodule of $M$ and let $I$ be its annihilator in $R$. Prove that the annihilator of $I$ in $M$ contains $N$. Give an example where the annihilator of $I$ in $M$ does not equal $N$. 
\item[(b)] Let $I$ be a right ideal of $R$ and let $N$ be its annihilator in $M$. Prove that the annihilator of $N$ in $R$ contains $I$. Give an example where the annihilator of $N$ in $R$ does not equal $I$. 
\end{enumerate}
\end{problem}
\begin{solution}
(a)\\
Let $A$ be the annihilator of $I$ in $M$ and let $n\in N$. Then $an = 0$ for all $a\in I$ by definition. But this means that $n\in A$. This proves that $N\subseteq A$ as desired. As an example where containment is strict let $M = \Z/2\Z\times \Z/2\Z$ be a $\Z$-module and let $N$ be the subgroup $\{(0,0),(1,0)\}$. Notice that $2\Z$ is the annihilator of $N$, but the annihilator of $2\Z$ is all of $M$.  \\\\
(b)\\
Let $J$ be the annihilator of $N$ in $R$ and let $a\in I$. Then $an = 0$ for all $n\in N$. But then by definition $a\in J$, and so $I\subseteq J$ as desired. An example where containment is strict occurs when considering the annihilator of $6\Z$ in the $\Z$-module $M= N = \Z/2\Z$. This ideal annihilates all of $M$, but the annihilator of $M$ is $2\Z$ which strictly contains $6\Z$. 

\end{solution}\oneperpage



\begin{problem}\label{ex:10.1.13}
Let $I$ be an ideal of $R$. Let $M'$ be the subset of elements $a$ of $M$ that are annihilatored by some power, $I^k$ of the ideal $I$, where the power may depend on $a$. Prove that $M'$ is a submodule of $M$. [Use Excercise 7.]
\end{problem}
\begin{solution}
Let $N_k$ be the annihilator of $I^k$. Elements of $I^k$ are of the form $\sum a_i^k$ where the sum is finite and each $a_i$ is an element of $I$. We thus notice that $N_k\subseteq N_{k+1}$ since if $n$ is annihilated by all finite sums $\sum a_i^k$ with $a_i\in I$ then \[
\left(\sum a_i^{k+1}\right)n =\sum (a_i^{k+1}n) = \sum (a_i a_i^{k}n) = \sum (a_i 0) = 0
\]
and so it is also annihilated by elements of $I^{k+1}$. Thus the union of all $N_k$ is a submodule by Exercise 7. This union is exactly $M'$, proving the desired result.
\end{solution}\oneperpage



\begin{problem}\label{ex:10.1.14}
Let $z$ be an element of the center of $R$, i.e. $zr=rz$ for all $r\in R$. Prove that $zM$ is a submodule of $M$, where $zM = \{zm\mid m\in M\}$. Show that if $R$ is the ring of $2\times 2$ matrices over a field and $e$ is the matrix with a $1$ in position 1, 1 and zeros elsewhere then $eR$ is \emph{not} a left $R$-submodule (where $M=R$ is considered as a left $R$-module as in Example 1)---in this case the matrix $e$ is not in the center of $R$.
\end{problem}
\begin{solution}Note that $0 = z0 \in zM$ and so $zM$ is nonempty. Letting $zx,zy\in zM$ where $x,y\in M$ are abitrary and letting $r\in R$ we have that \[
zx + rzy = zx+zry = z(x+ry) \in zM
\]
and so $zM$ satisfies the submodule criterion. 

Notice that \[
\begin{bmatrix}
0&1\\
0&0
\end{bmatrix}\begin{bmatrix}
a&b\\
c&d
\end{bmatrix} = \begin{bmatrix}
c&d\\
0&0
\end{bmatrix}
\]
and so in the example $eM$ is the set of matrices with zero entries in the bottom row and arbitrary entries in the top row. This collection is not a submodule since as a set it is not invariant under the left action of $R$ on it. In particular, \[
\begin{bmatrix}
0&0\\
1&0
\end{bmatrix}\begin{bmatrix}
1&1\\
0&0
\end{bmatrix} = \begin{bmatrix}
0&0\\
1&1
\end{bmatrix}
\]
which is not a matrix with zero entries in the bottom row. We conclude that $e$ is indeed not in the center of $R$.

\end{solution}\oneperpage



\begin{problem}\label{ex:10.1.15}
If $M$ is a finite abelian group then $M$ is naturally a $\Z$-module. Can this action be extended to make $M$ into a $\Q$-module?
\end{problem}
\begin{solution}
No, not always. Consider the $\Z$-module $\Z/2\Z$. If this were naturally a $\Q$-module then it would have some element $\frac{1}{2}\cdot 1$. This element would satisfy \[
\frac{1}{2}\cdot 1 + \frac{1}{2}\cdot 1 =\left(\frac{1}{2}+\frac{1}{2}\right)\cdot 1 = 1\cdot 1 = 1
\]
and in particular it would have order at least three as an element of the group $\Z/2\Z$. This is not possible. More generally, for any finite abelian group $G$ one can consider the action of $\frac{1}{|G|}$ to derive a contradiction. Thus finite abelian group never has a $\Q$ action compatible with the natural $\Z$ action. 

  However, if an abelian group is divisible then we can extend its natural $\Z$ action to a $\Q$ action. Of course nonzero divisible abelian groups are necessarily infinite, so this falls outside the scope of the problem.
\end{solution}\oneperpage



\begin{problem}\label{ex:10.1.16}
Prove that the submodules $U_k$ describe in the example of $F[x]$-modules are all of the $F[x]$-submodules for the shift operator.
\end{problem}
\begin{solution}
Let $V = F^n$ be a $F[x]$ module where $x$ acts as the shift operator and $F$ acts as normal. Let $U\subseteq V$ be a submodule of $V$. Let $k$ be the largest index such that there exists a vector in $U$ whose $k$-th coordinate is nonzero. Then we claim $U = U_k$. The inclusion $U\subseteq U_k$ is trivial since $U_k$ is all vectors in $V$ where coordinates following the $k$-th are zero. Hence we only have to show $U_k\subseteq U$. 

To show that $U_k\subseteq U$  we will show straightforwardly that $e_i$ is in $U$ for $1\le i\le k$. The set of these $e_i$ forms a basis for $U_k$ and so it will follow that $U_k\subseteq U$. Notice that we really only need to construct $e_k$, since all $e_i$ for $i<k$ can be obtained by the action of $x$, which will still be in $U$ since $U$ is a submodule. To construct $e_k$, let $v = (v_1,v_2,\ldots,v_k,0,0,\ldots, 0)$ be a vector in $U$ where $v_k\neq 0$. Then we can construct the basis vector $e_k$ by repeatedly zeroing out smaller coordinates in $v_k$: first consider \[
v-\left(\frac{v_{k-1}}{v_k} x\right) v \in U.
 \] 
 The $(k-1)$-th coordinate of this vector will be $v_k-v_k = 0$. We can repeat this process, acting on our new vector by $x^2$ multiplied by an appropriate scalar, subtracting the result, and so on. This eventually leads to a vector $(0,0,\ldots,0,v_k,0,0,\ldots,0)$ which can be transformed to $e_k$ via multiplication by the scalar $\frac{1}{v_k}$. This proves that $e_k\in U$, and as previously discussed this implies that $e_i\in U$ for all $1\le i\le k$. Hence $U_k\subseteq U$ and we are done.
\end{solution}\oneperpage



\begin{problem}\label{ex:10.1.17}
Let $T$ be the shift operator on the vector space $V$ and let $e_1,\ldots,e_n$ be the usual basis vector described in the example of $F[x]$-modules. If $m\ge n$ find $(a_mx^m + a_{m-1}x^{m-1}+\cdots +a_0)e_n$. 
\end{problem}
\begin{solution}
For convenience let $p(x) = a_mx^m + a_{m-1}x^{m-1} + \cdots + a_0$. We compute directly that \begin{align*}
p(x)\cdot e_n &= \left(\sum_{i=0}^m a_ix^i\right)\cdot e_n\\
& = \sum_{i=0}^m a_i(x^i\cdot e_n)&&\text{Via module axioms}\\
& = \sum_{i=0}^n a_i(x^i\cdot e_n) &&\text{Since $x^i\cdot e_n = 0$ for $i>n$}\\
& = \sum_{i=0}^n a_i(e_{n-i}) &&\text{Since $x$ acts as shift operator}\\
& = (a_n,a_{n-1},\ldots, a_1,a_0).
\end{align*}
Thus $p(x)\cdot e_n$ gives us the first $n+1$ coefficients in $p(x)$ in a vector in reverse order.
\end{solution}\oneperpage



\begin{problem}\label{ex:10.1.18}
Let $F=\R$. Let $V=\R^2$ and let $T$ be the linear transformation from $V$ to $V$ which is rotation clockwise about the origin by $\pi/2$ radians. Show that $V$ and $0$ are the only $F[x]$-submodules for this $T$. 
\end{problem}
\begin{solution}
It suffices to show that every nontrivial submodule is equal to $V$. Given a nontrivial submodule $U$, let $v$ be a nonzero vector in $U$. Then notice that $x\cdot v \in U$ is linearly independent from $v$. Since $U$ must also be a subspace of the vector space $V$, we see that $U$ contains $\Span\{v,x\cdot v\} = V$. Hence $U$ is all of $V$.
\end{solution}\oneperpage



\begin{problem}\label{ex:10.1.19}
Let $F=\R$, let $V=\R^2$ and let $T$ be the linear transformation from $V$ to $V$ which is projection onto the $y$-axis. Show that $V,0,$ the $x$-axis and the $y$-axis are the only $F[x]$-submodules for this $T$. 
\end{problem}
\begin{solution}
We know that $0$ and $V$ are always submodules. It remains to characterize the nontrivial proper submodules. Notice that such submodules are necessarily 1-dimensional subspaces of $V = \R^2$ since submodules under the action of $F[x]$ are always subspaces and 0- and 2-dimensional subspaces are trivial and non-proper submodules respectively. 

Let $U = \Span\{v\}$ be some nontrivial proper submodule. Since $U$ is 1-dimensional we must have that $x\cdot v = ax$ for some scalar $a$. In particular $v$ is an eigenvector of $T$ and so $U$ is an eigenspace of $T$. The only eigenspaces are clearly the $x$ and $y$ axes. One can verify quickly that these are submodules: they both are subspaces (in particular subgroups) of $V$ and are invariant under the action of $F[x]$ since the $y$-axis is only scaled and the $x$-axis is annihilated by any nonunits in $F[x]$. 
\end{solution}\oneperpage



\begin{problem}\label{ex:10.1.20}
Let $F=\R$, let $V=\R^2$ and let $T$ be the linear transformation from $V$ to $V$ which is rotation clockwise about the origin by $\pi$ radians. Show that \emph{every} subspace of $V$ is an $F[x]$ submodule for this $T$. 
\end{problem}
\begin{solution}
Rotating by $\pi$ radians is the same as additive negation. Hence we have $x\cdot v = -v$ for all vectors $v$. Being invariant under the action of $F$ and $x$ is enough to be a submodule, and subspaces are invariant under both by the definition of being a subspace (and hence an additive subgroup). Thus all subspaces are submodules.
\end{solution}\oneperpage



\begin{problem}\label{ex:10.1.21}
Let $n\in \Z^+$, $n>1$ and let $R$ be the ring of $n\times n$ matrices with entries from a field $F$. Let $M$ be the set of $n\times n$ matrices with arbitrary elments of $F$ in the first column and zeros elsewhere. Show that $M$ is a submodule of $R$ when $R$ is considered as a left module over itself, but $M$ is not a submodule of $R$ when $R$ is considered as a right $R$-module.
\end{problem}
\begin{solution}
It is clear that $M$ is an additive subgroup of the module $R$. When $R$ acts on $M$ from the left $M$ is invariant since the $i$-th column of $rm$ for $r\in R$ and $m\in M$ is just the product of $r$ with the $i$-th column in $m$. For $i>1$ this column is zero and so must be $r$'s product with it. Hence $rm\in M$. 

On the other hand when $R$ acts from the right the columns in $mr$ beyond the first may nonzero, as illustrated by the small example below.\[
\begin{bmatrix}
1&0\\
1&0
\end{bmatrix}\begin{bmatrix}
1&1\\
1&1
\end{bmatrix} = \begin{bmatrix}
1&1\\
1&1
\end{bmatrix} \notin M.
\]
\end{solution}\oneperpage



\begin{problem}\label{ex:10.1.22}
Suppose that $A$ is a ring with identity $1_A$ that is a (unital) left $R$-module satisfying $r\cdot(ab) = (r\cdot a)b = a(r\cdot b)$ for all $r\in R$ and $a,b\in A$. Prove that the map $f:R\to A$ defined by $f(r) = r\cdot 1_A$ is a ring homomorphism mapping $1_R$ to $1_A$ and $f(R)$ is contained in the center of $A$. Conclude that $A$ is an $R$-algebra and that the $R$-module structure on $A$ induced by its algebra structure is precisely the original $R$-module structure.
\end{problem}
\begin{solution}
That $f$ maps $1_R$ to $1_A$ follows from the fact that $f(1_R) = 1_R\cdot 1_A = 1_A$. Given $r,s\in R$ we have that \[
f(r+s) = (r+s)\cdot 1_S = r\cdot 1_S + s\cdot 1_S = f(r)+f(s)
\]
and \[
f(rs) = rs\cdot 1_A  = r\cdot (s\cdot 1_A) = r\cdot (s\cdot 1_A1_A) = r\cdot (1_A(s\cdot 1_A)) = (r\cdot 1_A)(s\cdot 1_A) = f(r)f(s)
\]
so $f$ is a ring homomorphism. Let $r\cdot 1_A \in f(R)$ and $a\in A$. Then we have that \[
(r\cdot 1_A)a = r\cdot (1_A a) = r\cdot a = r\cdot (a1_A) = a(r\cdot 1_A) 
\]
and so $f(R)$ is in the center of $A$. This proves that $A$ is an $R$-algebra. The $R$-module structure on $A$ as an algebra is the same as its original structure since $r\cdot a = r\cdot (1_A a) = (r\cdot 1_A)a.$


\end{solution}\oneperpage



\begin{problem}\label{ex:10.1.23}
Let $A$ be the direct product ring $\C\times \C$ (cf Section 7.6). Let $\tau_1$ denote the identity map on $\C$ and let $\tau_2$ denote complex conjugation. For any pair $p,q\in \{1,2\}$ (not necessarily distinct) define \[
f_{p,q}:\C\to\C\times \C\quad\text{ by }\quad f_{p,q}(z) = (\tau_p(z),\tau_q(z)).
\]
So, for example $f_{2,1}: z\mapsto (\overline{z},z)$ where $\overline{z}$ is the complex conjugate of $z$, i.e. $\tau_2(z)$.\begin{enumerate}
\item[(a)] Prove that each $f_{p,q}$ is an injective ring homomorphism, and that they all agree on the subfield $\R$ of $\C$. Deduce that $A$ has four distinct $\C$-algebra structures. Explicitly give the action $z\cdot(u,v)$ of a complex number $z$ on an ordered pair in $A$ in each case.
\item[(b)] Prove that if $f_{p,q}\neq f_{p',q'}$ then the identity map on $A$ is \emph{not} a $\C$-algebra homomorphism from $A$ considered as a $\C$-algebra via $f_{p,q}$ to $A$ considered a $\C$ algebra via $f_{p',q'}$ (although the identity is an $\R$ algebra isomorphism).
\item[(c)] Prove that for any pair $p,q$ there is some ring isomorphism from $A$ to itself such that $A$ is isomorphic as a $\C$ algebra via $f_{p,q}$ to $A$ considered as a $\C$ algebra via $f_{1,1}$ (the ``natural'' $\C$-algebra structure on $A$).
\end{enumerate}
\emph{Remark:} In the preceding exercise $A=\C\times \C$ is not a $\C$-algebra over either of the direct factor component copies of $\C$ (for example the subring $\C\times 0 \cong \C$) since it is not a unital module over these copies of $\C$ (the 1 of these subrings is not the same as the 1 of $A$).
\end{problem}
\begin{solution}
(a)\\
That each $f_{p,q}$ agrees on $\R$ is trivial since complex conjugation fixes $\R$. Also recall that complex conjugation is an automorphism of $\C$ and so each $\tau_p$ is an automorphism. Hence $f_{p,q}$ behaves as a ring homomorphism in each coordinate and overall will be a homomorphism. It is a proper ring homomorphism since it maps $1_\C = 1$ to $1_{\C\times \C} = (1,1)$.
  That each $f_{p,q}$ is injective follows from the injectivity of $\tau_p$ for $p=1,2$. In particular if $z$ is nonzero then $f_{p,q}(z)$ is nonzero for all $p,q$ and hence the kernel of $f_{p,q}$ is trivial.


The explicit action induced by $f_{p,q}$ is just\[
z\cdot(u,v) = (\tau_p(z)u, \tau_q(z)v).
\]
In particular, $f_{1,1}$ acts via natural scalar multiplication. \\\\
(b)\\
If $f_{p,q} \neq f_{p',q'}$ then we notice that \[
f_{p,q}(i) \neq f_{p',q'}(i)
\]
since there must be a coordinate in which one map conjugates and the other does not. Hence the action of $i\in \C$ induced by $f_{p,q}$ differs from that induced by $f_{p',q'}$ and in particular there exists $(z_1,z_2)\in \C\times\C$ so that the action of $i$ on $(z_1,z_2)$ induced by each is a different element of $\C\times \C$. Denote by $\cdot$ the action induced by $f_{p,q}$ and by $\circ$ the action induced by $f_{p',q'}$. If the identity map $\Id$ on $\C \times \C$ were a $\C$-algebra homomorphism we would have that \[i\cdot(z_1,z_2) = \Id(i\cdot (z_1,z_2)) = i\circ \Id((z_1,z_2)) = i\circ (z_1,z_2)\] which is a contradiction. Hence the identity is not a $\C$-algebra homomorphism.\\\\
(c)\\
For $f_{p,q}$ the isomorphism of $\C\times\C$ which makes it isomorphic to the natural action is the isomorphism which acts as $\tau_p$ in the first coordinate and $\tau_q$ in the second. Let $\phi$ denote this map. The map $\phi$ is clearly a ring isomorphism since $\tau_p$ and $\tau_q$ are ring isomorphisms of each coordinate. To see that this gives $\C\times\C$ the natural $\C$-algebra structure, let $\cdot$ denote the natural action and $\circ$ denote the action induced by $f_{p,q}$. Then we have that $\phi$ is a $\C$-algebra isomorphism since \begin{align*}
\phi(z\circ (z_1,z_2))& = \phi((\tau_p(z)z_1,\tau_q(z)z_2))\\
& = \big(\tau_p(\tau_p(z)z_1),\tau_q(\tau_q(z)z_2)\big)\\
& = (z\tau_p(z_1),z\tau_q(z_2)) &&\text{Since $\tau_p(\tau_p(z)) = z$ for all $\tau_p$}\\
& = z\cdot(\tau_p(z_1),\tau_q(z_2))\\
&=z\cdot \phi((z_1,z_2)).
\end{align*}
Hence $\C\times\C$ with the $f_{p,q}$ action is $\C$-algebra isomorphic to $\C\times\C$ with the natural action, as desired.

\end{solution}\oneperpage


\end{section}









\begin{section}{Quotient Modules and Module Homomorphisms}
In these exercises $R$ is a ring with 1 and $M$ is a left $R$-module.
\begin{problem}\label{ex:10.2.1}
Use the submodule criterion to show that kernels and images of $R$-module homomorphisms are submodules.
\end{problem}
\begin{solution}Kernels and images of $R$-module homomorphisms always contain zero by virtue of being kernels and images of the underlying group homomorphisms. Thus they are nonempty. Let $\phi:N\to M$ be an $R$-module homomorphism. We will check the second condition of the submodule criterion for $\ker \phi$ and $\phi(N)$. Letting $x_1,x_2\in \ker \phi$ and $r\in R$ we notice that \[
\phi(x_1+rx_2) = \phi(x_1)+r\phi(x_2) = 0+r0 = 0
\]
and so $x_1+rx_2\in \ker \phi$. This proves that $\ker\phi$ is a submodule of $N$. Letting $\phi(n_1)$ and $\phi(n_2)$ be arbitrary elements of $\phi(N)$ and letting $r\in R$ we have \[
\phi(n_1) + r\phi(n_2) = \phi(n_1+rn_2) \in \phi(N).
\]
Hence $\phi(N)$ also satisfies the second condition of the submodule criterion and is a submodule.

\end{solution}\oneperpage



\begin{problem}\label{ex:10.2.2}
Show that the relation ``is $R$-module isomorphic to'' is an equivalence relation on any set of $R$-modules.
\end{problem}
\begin{solution}We verify each property of an equivalence relation directly. \begin{itemize}
\item \emph{Reflexivity:} Any $R$-module is isomorphic to itself via the identity map.
\item \emph{Symmetry:} Let $\phi:N\to M$ be an isomorphism of $R$-modules. We claim that the map $\phi^{-1}$ is also an $R$-module isomorphism. We know it is a group isomorphism since $\phi$ is a group isomorphism, and so all we have to verify is that it preserves the action of $R$. Let $m\in M$ and $r\in R$. We know $m=\phi(n)$ for some $n\in N$ and since $\phi$ is an $R$-module isomorphism we also have $\phi(rn) = r\phi(n) = rm$. Putting this together, we have\[
\phi^{-1}(rm) = \phi^{-1}(\phi(rn)) = rn = r\phi^{-1}(m)
\]
and so $\phi^{-1}$ is a homomorphism of $R$-modules. This proves that $M$ is $R$-module isomorphic to $N$.
\item \emph{Transitivity:} Let \[
\begin{CD} N@>\phi>> M @>\psi>> L\end{CD}\]
be a sequence of $R$-module isomorphisms. We claim that $\psi\circ\phi$ is an $R$-module isomorphism from $N$ to $L$. It is a group isomorphism by virtue of $\phi$ and $\psi$ being group isomorphisms, so we need only verify that the action of $R$ is preserved. Given $r\in R$ and $n\in N$ we have directly that \[
\psi(\phi(rn)) = \psi(r\phi(n)) = r\psi(\phi(n))
\]
by virtue of $\phi$ and $\psi$ being $R$-module isomorphisms. This proves that $N$ is $R$-module isomorphic to $L$, as desired. We conclude that ``is $R$-module isomorphic to'' is an equivalence relation. 
\end{itemize}

\end{solution}\oneperpage



\begin{problem}\label{ex:10.2.3}
Give an explicit example of a map from one $R$-module to another which is a group homomorphism but not an $R$-module homomorphism.
\end{problem}
\begin{solution}
Natural examples occur whenever a module $M$ has two distinct $R$-module structures on it. In this case the identity map from $M$ to $M$ is a group homomorphism, but not an $R$-module homomorphism. Some examples of modules $M$ which can have distinct structures are described below. \begin{itemize}
\item The algebra $A = \C\times\C$ described in 10.1.23 as a module over $\C$.
\item A vector space as an $F[x]$ module, where the action of $x$ can be various linear transformations.
\item  Example 2 on page 346 also works: the map $x\mapsto x^2$ in $M = F[x]$ is never an $F[x]$-module homomorphism. Indeed, one can generalize this by sending $\phi:x\mapsto f(x)$ for any $f(x)\neq x$. This is a group homomorphism but not an $F[x]$ module homomorphism since we would have $f(x) = \phi(x) = \phi(x\cdot 1) = x\phi(1) = x$. Perhaps most generally one can consider a ring with unity and a nontrivial endomorphism. This endomorphism serves as a group homomorphism that is not an $R$-module homomorphism.
\end{itemize}
\end{solution}\oneperpage



\begin{problem}\label{ex:10.2.4}
Let $A$ be any $\Z$-module, let $a$ be any element of $A$ and let $n$ be a positive integer. Prove that the map $\phi_a:\Z/n\Z\to A$ given by $\phi_a(\overline{k}) = ka$ is a well defined $\Z$-module homomorphism if and only if $na=0$. Prove that $\Hom_\Z(\Z/n\Z, A)\cong A_n$ where $A_n=\{a\in A \mid na=0\}$ (so $A_n$ is the annihilator in $A$ of the ideal $(n)$ of $\Z$ --- cf. Exercise 10, Section 1).
\end{problem}
\begin{solution}
We begin by proving that $\phi_a$ is a well defined $\Z$-module homomorphism if and only if $na=0$.

$(\Rightarrow)$ Suppose $\phi_a$ is a well defined $\Z$-mdoule homomorphism. Then we have that $na = \phi_a(\overline{n}) = \phi_a(0)$ which must be zero since $\phi_a$ is a homomorphism of groups.

$(\Leftarrow)$ Suppose $na=0$. To show $\phi_a$ is well defined we need to show that $\phi_a(\overline{k})$ does not depend on our choice of representative for $\overline{k}$. Letting $k+bn$ be an arbitrary representative of $\overline{k}$ we have that \[
\phi_a(\overline{k+bn}) = (k+bn)a = ka + bna = ka  + b(na) = ka+b0 = ka 
\]
and so the map is well defined. To prove it is a group homomorphism let $\overline{k_1},\overline{k_2}\in \Z/n\Z$. Then we have \[
\phi_a(\overline{k_1}+\overline{k_2}) = (\overline{k_1}+\overline{k_2})a = \overline{k_1}a+\overline{k_2}a = \phi(\overline{k_1})+\phi(\overline{k_2}).
\]
To see it is a $\Z$-module homomorphism, let $z\in \Z$ and observe that \[
\phi_a(z\overline{k}) = \phi_a(\overline{zk}) = \overline{zk}a = z\overline{k}a = z\phi_a(\overline{k})
\]
where the second to last equality follows from the fact that $z$ acts the same on multiples of $a$ as any $z'$ congruent to $z$ mod $n$. This shows that $\phi_a$ is a homomorphism of $\Z$-modules. 

To prove that $\Hom_\Z(\Z/n\Z, A) \cong A_n$ we show that each homomorphism $\phi$ from $\Z/n\Z$ to $A$ is uniquely determined by $\phi(1)$ and $\phi(1)\in A_n$. In fact, we show that all $\phi\in \Hom_\Z(\Z/n\Z, A)$ are of the form $\phi_a$ for some $a\in A_n$. Given an homomorphism $\phi:\Z/n\Z\to A$ consider $\phi(1) = a$. We know that $\phi(1)\in A_n$ since \[
na = n\phi(1) = \phi(n) = \phi(0) = 0.
\]
Extending $\phi$ to the rest of $\Z/n\Z$ we see that necessarily $\phi = \phi_a$. By the result proven earlier in the problem, we conclude that every homomorphism in $\Hom_\Z(\Z/n\Z,A)$ is of the form $\phi_a$ for $a\in A_n$. To prove that $\Hom_\Z(\Z/n\Z,A)$ is isomorphic to $A_n$ as a module, notice that by the properties of homomorphisms we have $\phi_a+\phi_b = \phi_{a+b}$ and $z\phi_a = \phi_za$ and also $\phi_a = \phi_b$ if and only if $a=b$. Hence the map $\phi_a\mapsto a$ is an isomorphism of $\Z$-modules and we conclude the desired result.
\end{solution}\oneperpage



\begin{problem}\label{ex:10.2.5}
Exhibit all $\Z$-module homomorphisms from $\Z/30\Z$ to $\Z/21\Z$. 
\end{problem}
\begin{solution}
By the previous exercise we know that $\Hom_\Z(\Z/30\Z,\Z/21\Z)$ consists of maps $\phi_a$ where $a\in \Z/21\Z$ is annihilated by $30\Z$. The elements in $\Z/21\Z$ annihilated by 30 are exactly those which are multiples of 7. Hence the only maps are the zero map, $a\mapsto 7a$ and $a\mapsto 14a$. 
\end{solution}\oneperpage



\begin{problem}\label{ex:10.2.6}
Prove that $\Hom_\Z(\Z/n\Z, \Z/m\Z) \cong \Z/(n,m)\Z$.
\end{problem}
\begin{solution}
By 10.2.4 we have that $\Hom_\Z(\Z/n\Z, \Z/m\Z)$ is isomorphic to the annihilator of $n\Z$ in $\Z/m\Z$. This annihilator will consist of exactly the $a\in\Z/m\Z$ for which $na$ is a multiple of $m$. Let $d$ be the greatest common divisor of $n$ and $m$. Then this annihilator can be easily described as the cyclic module generated by $m/d$ in $\Z/m\Z$. Indeed, $na$ is a multiple of $m$ if and only if $a$ is a multiple of $m/d$. The cyclic module generated by $m/d$ has $d$ elements, and hence is isomorphic to $\Z/d\Z$. This proves the result.

\end{solution}\oneperpage



\begin{problem}\label{ex:10.2.7}
Let $z$ be a fixed element of the center of $R$. Prove that the map $m\mapsto zm$ is an $R$-module homomorphism from $M$ to itself. Show that for a commutative ring $R$ the map from $R$ to $\End_R(M)$ given by $r\mapsto rI$ is a ring homomorphism (where $I$ is the identity endomorphism).
\end{problem}
\begin{solution}
This is a group homomorphism since $z(m_1+m_2) = zm_1 + zm_2$ by the module axioms. Since $z$ is in the center of $r$ we also have $r(zm) = z(rm)$ for all $r\in R$ and so this map also respects the $R$-module structure. 

Let $\phi$ denote the map $r\mapsto rI$. Then the ring homomorphism conditions are easily verified: $\phi(r_1+r_2) = (r_1+r_2)I = r_1I+r_2I = \phi(r_1) + \phi(r_2)$, and $\phi(r_1r_2) = r_1r_2I = r_1I r_2I = \phi(r_1)\phi(r_2)$. This proves the result.
\end{solution}\oneperpage



\begin{problem}\label{ex:10.2.8}
Let $\phi:M\to N$ be an $R$-module homomorphism. Prove that $\phi(\Tor(M)) \subseteq \Tor(N)$ (cf. Exercise 8 in Section 1). 
\end{problem}
\begin{solution}
Let $m\in \Tor(M)$ and $r\in R$ be nonzero so that $rm=0$. Then $r\phi(m) = \phi(rm) = \phi(0) = 0$ and so $\phi(m)\in \Tor(N)$. This proves the result.
\end{solution}\oneperpage



\begin{problem}\label{ex:10.2.9}
Let $R$ be a commutative ring. Prove that $\Hom_R(R,M)$ and $M$ are isomorphic as left $R$-modules. [Show that each element of $\Hom_R(R,M)$ is determined by its value on the identity of $R$.]
\end{problem}
\begin{solution}
Let $\phi\in \Hom_R(R,M)$ and let $r\in R$. We will show that $\phi(r)$ can be expressed in terms of $\phi(1)$. Notice that \[
\phi(r) = \phi(r\cdot 1) = r\phi(1)
\]
by definition of being an $R$-module homomorphism. Hence each $\phi$ can be expressed as $\phi_m$ for $m\in M$ where $\phi_m(r) = rm$. We claim that the map $m\mapsto \phi_m$ is a homomorphism of the $R$-modules $M$ and $\Hom_R(R,M)$. 

First, note that this map is injective since $\phi_{m_1} = \phi_{m_2}$ means that $m_1 = \phi_{m_1}(1) = \phi_{m_2}(1) = m_2$. Furthermore it is surjective since every homomorphism is uniquely determined by its value on 1 and can be written as $\phi_m$. This map is also a group homomorphism since \[
\phi_{m_1+m_2}(s)  = s(m_1+m_2) = sm_1+sm_2 = \phi_{m_1}(s) + \phi_{m_2}(s)
\] for all $s\in R$ and hence $\phi_{m_1+m_2} = \phi_{m_1} + \phi_{m_2}$. To show this map respects the $R$-module structure, let $r\in R$ and observe that \[
r\phi_m(s) =rsm = s(rm) = \phi_{rm}(s)\]
for all $s\in R$, and so $r\phi_m = \phi_{rm}$. We conclude  that $m\mapsto \phi_m$ is an $R$-module isomorphism as desired.
\end{solution}\oneperpage



\begin{problem}\label{ex:10.2.10}
Let $R$ be a commutative ring. Prove that $\Hom_R(R,R)$ and $R$ are isomorphic as rings.
\end{problem}
\begin{solution}
Consider the map $r\mapsto rI$ where $I$ is the identity map on $R$. By 10.2.7 this is a homomorphism from $R$ to $\End_R(R) = \Hom_R(R,R)$. But this is also the exact map described in the proof of 10.2.9. In particular, this is an isomorphism of the $R$-module $\Hom_R(R,R)$ with the $R$-module $R$. We conclude that this map is bijective, and by virtue of being a ring homomorphism it must be a ring isomorphism. This proves the result. 
\end{solution}\oneperpage



\begin{problem}\label{ex:10.2.11}
Let $A_1,A_2,\ldots, A_n$ be $R$-modules and let $B_i$ be a submodule of $A_i$ for each $i=1,2,\ldots,n$. Prove that \[
(A_1\times\cdots \times A_n)/(B_1\times \cdots \times B_n) \cong (A_1/B_1)\times \cdots \times (A_n/B_n).
\] [Recall Exercise 14 in Section 5.1.]
\end{problem}
\begin{solution}
Consider the map $\phi: A_1\times\cdots \times A_n \to (A_1/B_1)\times \cdots \times (A_n/B_n)$ defined by \[
\phi(a_1,a_2,\ldots, a_n) = (a_1+B_1, a_2+B_2,\ldots, a_n+B_n).
\]
Note that this is a homomorphism of $R$-modules since it is $R$-linear in each coordinate. Indeed, \[
a_i + ra_i' + B_i = (a_i+B_i) + r(a_i'+B_i)
\]
by definition of the quotient module $A_i/B_i$. Then consider the kernel of this map. If $(a_1,\ldots,a_n)\in \ker\phi$ we must have $a_i + B_i = 0+B_i$ for all $i$. That is, we must have $a_i\in B_i$ and in particular $(a_1,\ldots,a_n)\in B_1\times\cdots\times B_n$. This condition is obviously necessary and sufficient to be in the kernel, and so the kernel is  $B_1\times\cdots\times B_n$. Also note that the map is surjective, with a preimage of $(a_1+B_1,\ldots, a_n+B_n)$ being simply $(a_1,\ldots, a_n)$. By the first isomorphism theorem we conclude that \begin{align*}
(A_1\times\cdots \times A_n)/(B_1\times \cdots \times B_n) &= (A_1\times\cdots \times A_n)/\ker\phi \\
& \cong \phi(A_1\times\cdots \times A_n)\\
& = (A_1/B_1)\times \cdots \times (A_n/B_n)
\end{align*}
which proves the result.
\end{solution}\oneperpage



\begin{problem}\label{ex:10.2.12}
Let $I$ be a left ideal of $R$ and let $n$ be a positive integer. Prove \[
R^n/IR^n \cong R/IR\times\cdots\times R/IR \quad \text{($n$ times)}
\]
where $IR^n$ is defined as in Exercise 5 of Section 1. [Use the preceding exercise.]
\end{problem}
\begin{solution}
By definition $R^n = R\times \cdots\times R$ where the product is taken $n$ times. Thus we only need to show that $IR^n = (IR)^n$, and the result will follow immediately from the previous problem. To prove this we show containment in both directions. Elements of $IR^n$ are of the form $a(r_1,\ldots, r_n) = (ar_1,\ldots, ar_n)$ where $a\in I$. Such elements are clearly in $(IR)^n$ since elements in $(IR)^n$ have the form $(a_1r_1,\ldots, a_nr_n)$ for $a_i\in I$. Thus we have $IR^n\subseteq (IR)^n$ immediately.

To show that $(IR)^n\subseteq IR^n$ consider an arbitrary element $(a_1r_1,\ldots, a_nr_n)\in (IR)^n$. Notice that the tuple $v_i = (0,\ldots, a_ir_i,\ldots, 0)$ which is zero in all coordinates but the $i$-th is in $IR^n$ since it is just $a_i(0,\ldots, a_i,\ldots, 0)$. But $IR^n$ is closed under finite sums, and so we can write \[
(a_1r_1,\ldots,a_nr_n) = \sum_{i=1}^n v_i  \in IR^n.
\]
This proves that $(IR)^n\subseteq IR^n$, and so we conclude the desired result. As an interesting aside, I believe this also holds when the product is infinite since we only allow finitely many nonzero coordinates.
\end{solution}\oneperpage



\begin{problem}\label{ex:10.2.13}
Let $I$ be a nilpotent ideal in a commutative ring $R$ (cf. Exercise 37, 7.3), let $M$ and $N$ be $R$-modules and let $\phi:M\to N$ be an $R$-module homomorphism. Show that if the induced map $\overline{\phi}:M/IM \to N/IN$ is surjective, then $\phi$ is surjective.
\end{problem}
\begin{solution}
\emph{Note: I refered to \url{https://crazyproject.wordpress.com/aadf/\#df-10} for the solution to this problem. Wrote my own version of the solution however.}\\\\
We will first prove that $N = \phi(M) + I^kN$ for all $k$, independent of the fact that $I$ is nilpotent. Consider the following diagram: \[
\begin{CD}
M @>\phi>> N\\
@V\pi_MVV @VV\pi_NV\\
M/IM @>\overline{\phi}>> N/IN
\end{CD}
\]
Above we have $\pi_M$ and $\pi_N$ as projection mod $IM$ and $IN$ respectively. This diagram commutes by virtue of $\overline{\phi}$ being the induced map. We begin by showing that $N = \phi(M) + IN$. Notice that $N$ is clearly the preimage of $N/IN$ under $\pi_N$. Also $N/IN = \overline{\phi}(M/IM)$ and so any $n+IN\in N/IN$ can be written as $\phi(m) + IN$ for some $m\in M$. This implies that the preimage of $N/IN$ under $\pi_N$ will be $\phi(M) + IN$. Indeed, $\pi_N(n)  = \phi(m)+IN$ implies that $n$ is the sum of something in $\phi(M)$ and the kernel of $\pi_N$ which is $IN$. So far we have shown that $N = \phi(M) + IN$. 

To prove that $N = \phi(M) + I^kN$ we use induction on $k$, where we have just proven the base case. For the inductive step, we have\[
N = \phi(M) + I^kN = \phi(M) + I^k(\phi(M) + IN) = \phi(M) + I^k\phi(M) + I^{k+1}N = \phi(M) + I^{k+1}N
\]
where the last equality follows from the fact that $I^k\phi(M) \subseteq \phi(M)$. By induction we conclude that $N = \phi(M) + I^kN$ for all $k$. Taking $k$ large enough we have $I^k = 0$ and so $\phi(M) = N$ as desired. 

It is illustrative to see the equality $N = \phi(M) + I^kN$ for some non-nilpotent ideal. For an example, we take $R = M = N = \Z$. Let $\phi:\Z\to\Z$ be the doubling map (i.e. $\phi(z) = 2z$), which is indeed a homomorphism of $\Z$ modules since it is a homomorphism of abelian groups. Notice that it is not surjective. For our ideal $I$ we choose  $3\Z$. Then our diagram of modules becomes \[
\begin{CD}
\Z @>\text{$\phi$ (doubling)}>> \Z\\
@V\pi VV @VV\pi V\\
\Z/3\Z @>\text{$\overline{\phi}$ (doubling)}>> \Z/3\Z
\end{CD}
\]
Now, the induced map is surjective since we have $0\mapsto 0$, $1\mapsto 2$ and $2\mapsto 1$. Our result states that $\Z = \phi(\Z) + 3^k\Z$ for all $k$. Since $\phi(\Z) = 2\Z$ and $2\Z$ and $3^k\Z$ are always comaximal ideals, we see that the result holds. 


\end{solution}\oneperpage



\begin{problem}\label{ex:10.2.14}
Let $R=\Z[x]$ be the ring of polynomials in $x$ and let $A = \Z[t_1,t_2,\ldots]$ be the ring of polynomials in the independent indeterminates $r_1,r_2,\ldots$. Define an action of $R$ on $A$ as follows: 1) let $1\in R$ act on $A$ as the identity, 2) for $n\ge 1$ let $x^n\circ 1 = t_n$, let $x^n\circ t_i = t_{n+i}$ for $i=1,2\ldots$, and let $x^n$ act as $0$ on monomials in $A$ of (total) degree at least two, and 3) extend $\Z$-linearly, i.e., so that the module axioms 2(a) and 2(c) are satisfied. \begin{enumerate}
\item[(a)] Show that $x^{p+q}\circ t_i = x^p\circ(x^q\circ t_i) = t_{p+q+i}$ and use this show that under this action the ring $A$ is a (unital) $R$-module.
\item[(b)] Show that the map $\phi:R\to A$ defined by $\phi(r) = r\circ 1_A$ is an $R$-module homomorphism of the ring $R$ into the ring $A$ mapping $1_R$ to $1_A$, but not a ring homomorphism from $R$ to $A$.
\end{enumerate}
\end{problem}
\begin{solution}
(a)\\
We can compute directly that \begin{align*}
x^{p+q}\circ t_i = t_{p+q+i} = x^p\circ t_{q+i} = x^p\circ(x^q\circ t_i)
\end{align*}
as desired. We can use this to show that $A$ is an $R$-module by considering arbitrary polynomials $f = \sum_{i=0}^n a_ix^i$ and $g =\sum_{j=0}^m b_jx^j$ in $ \Z[x]$. To prove that $fg\circ T = f\circ g\circ T$ for all $T\in A$ it suffices to consider $T=t_k$ since the action is by definition extended linearly and acts as zero on monomials of higher degree. We have that\begin{align*}
fg\circ t_k& = \left(\sum_{i=0}^n a_ix^i\right)\left(\sum_{j=0}^m b_jx^j\right)\circ t_k\\
& = \left(\sum_{i=0}^{n+m} \left(\sum_{j=0}^ia_jb_{i-j}\right)x^i\right)\circ t_k\\
& = \sum_{i=0}^{n+m} \left(\sum_{j=0}^ia_jb_{i-j}\right)(x^i\circ t_k)&&\text{By $R$-linearity}\\
& = \sum_{i=0}^{n+m} \left(\sum_{j=0}^ia_jb_{i-j}\right)t_{k+i}&&\text{By definition of the action}\\
\end{align*}
Now, we can change the indices in this sum as follows. The various coefficients $a_jb_{j-i}$ are all of the form $a_{i'}b_{j'}$ where $0\le i' \le n$ and $0\le j'\le m$ (there are some additional pairs but for these we have $a_j = 0$ or $b_{j-i} = 0$). The coefficient $a_{i'}b_{j'}$ appears as the coefficient of $t_{k+i'+j'}$. Hence this all simplifies  as \begin{align*}
fg\circ t_k & = \sum_{i=0}^n\sum_{j=0}^m a_ib_j t_{k+i+j}\\
& = \sum_{i=0}^n\sum_{j=0}^m a_ib_j x^i \circ (x^j \circ t_{k})\\
& = \sum_{i=0}^n\sum_{j=0}^m a_ix_i\circ b_j \left( x^j \circ t_{k}\right)\\
& = \sum_{i=0}^na_ix^i\circ \left( \sum_{j=0}^m b_j  (x^j \circ t_{k})\right)\\
& = \sum_{i=0}^na_ix^i\circ (g\circ t_k)\\
& = f\circ(g\circ t_k). 
\end{align*}
  This shows that the action obeys axiom 2(b) for modules. We already know it satisfies the other axioms so $A$ is indeed an $R$-module. That the action is unital follows directly from the definition since $1\in R$ acts as identity. Thus $A$ is a unital $R$-module as desired.\\\\
(b)\\
This map is naturally a homomorphism of the abelian groups since \[
\phi(r_1+r_2) = (r_1+r_2)\circ 1_A = r_1\circ 1_A + r_2\circ 1_A = \phi(r_1)+\phi(r_2).
\]
Indeed this is an example of the maps $\phi_a$ described in the solution to Problem 10.2.9. It maps $1_R$ to $1_A$ since the module action is unital. 

To see that this is not a ring homomorphism, consider the image of $x^2$. We have that $\phi(x^2) = t_2$. But $\phi(x)\phi(x) = t_1^2 \neq t_2$ so the map is not a ring homomorphism.

\end{solution}\oneperpage

\end{section}









\begin{section}{Generation of Modules, Direct Sums, and Free Modules}
In these exercises $R$ is a ring with 1 and $M$ is a left $R$-module.
\begin{problem}\label{ex:10.3.1}
Prove that if $A$ and $B$ are sets of the same cardinality, then the free modules $F(A)$ and $F(B)$ are isomorphic. 
\end{problem}
\begin{solution}
Let $\phi:A\to B$ be a bijection and $\phi^{-1}$ be its inverse. Then Theorem 6 tells us that there exist unique $R$-module homomorphisms $\Phi:F(A)\to F(B)$ and $\Phi^{-1}:F(B)\to F(A)$ so that $\Phi$ agrees with $\phi$ on $A$ and $\Phi^{-1}$ agrees with $\phi^{-1}$ on $B$. We claim that $\Phi$ is an $R$-module isomorphism. It is clear that $\Phi^{-1}\circ \Phi$ is identity on $F(A)$ and so $\Phi$ must be injective. On the other hand $\Phi\circ \Phi^{-1}$ is identity on $F(B)$ which means that $\Phi$ must be surjective. Hence $\Phi$ is an isomorphism, proving the result.
\end{solution}\oneperpage



\begin{problem}\label{ex:10.3.2}
Assume $R$ is commutative. Prove that $R^n\cong R^m$ if and only if $n=m$, i.e., two free $R$-modules of finite rank are isomorphic if and only if they have the same rank. [Apply Exercise 12 of Section 2 with $I$ a maximal ideal of $R$. You may assume that if $F$ is a field, then $F^n\cong  F^m$ if and only if $n=m$, i.e. two finite dimensional vector spaces over $F$ are isomorphic if and only if they have the same dimension --- this will be proved later in Section 11.1]
\end{problem}
\begin{solution}
$(\Leftarrow)$ If the modules have the same rank then they are isomorphic by the previous problem together with the result that any free module of rank $n$ is free over its basis. 

$(\Rightarrow) $ We begin by proving the following general fact. If $M\cong N$ as $R$-modules and $I$ is an ideal of $R$, then $M/IM \cong N/IN$. To prove this let $\phi:M\to N$ be an $R$-module isomorphism and consider its induced map $\overline{\phi}:M/IM\to N/IN$ which maps $m+IM\mapsto \phi(m)+IN$. This map is well defined since we are taking a quotient of each module by the action of the same ideal. This map is surjective since if $n+IN\in N/IN$ it has as a preimage $\phi^{-1}(n) + IM\in M/IM$. On the other hand it is well defined to talk about the induced inverse $\overline{\phi^{-1}}:N/IN\to M/IM$. One can observe that $\overline{\phi^{-1}}\circ \overline{\phi}$ acts as identity on $M/IM$ since \[
\overline{\phi^{-1}}(\overline{\phi}(m+IM)) = \overline{\phi^{-1}}(\phi(m)+IN) = \phi^{-1}(\phi(m))+IM = m+IM.
\]
Hence $\overline{\phi}$ must be injective. We conclude that $\overline{\phi}$ is an isomorphism.
 
   Now suppose that $R^n\cong R^m$. Letting $I$ be a maximal ideal, we have from 10.2.12 that \[
(R/IR)^n \cong R^n/IR^n \cong  R^m/IR^m\cong(R/IR)^m\]
where the middle isomorphism is the one induced from $R^n\cong R^m$ when modding out by the action of $I$. But this says that two vectors spaces of dimension $m$ and $n$ respectively are isomorphic, and hence $m=n$. This proves the result.
\end{solution}\oneperpage



\begin{problem}\label{ex:10.3.3}
Show that the $F[x]$-modules in Exercises 18 and 19 of Section 1 are both cyclic. 
\end{problem}
\begin{solution}
\textbf{Exercise 18:} This module is $V = \R^2$ with the action of $x$ being given by the linear transformation that rotates by $\pi/2$. We notice that $V$ is generated by $(1,0)$ since we have $x\cdot(1,0) = (0,1)$ and $\{(1,0),(0,1\}$ spans $\R^2$ over $\R$, which is a subring of $\R[x]$. In fact we could choose any nonzero vector and $V$ would be cyclicly generated by it. \\\\
\textbf{Exercise 19:} Again the module if $V = \R^2$, but now the action of $x$ is given by projection onto the $y$-axis. In this case we see that $V$ is not cyclicly generated by $(1,0)$ since the projection of this is just the zero vector. However, $V$ is generated by $(1,1)$ since $x\cdot (1,1) = (0,1)$ which is linearly independent from $(1,1)$. Hence together $(1,1)$ and $x\cdot (1,1)$ span $V$ over $\R$ and since $\R\subseteq \R[x]$ we see that $(1,1)$ generated $V$. 

\end{solution}\oneperpage



\begin{problem}\label{ex:10.3.4}
An $R$-module $M$ is called a \emph{torsion} module if for each $m\in M$ there is a nonzero element of $r\in R$ such that $rm=0$, where $r$ may depend on $m$ (i.e., $M=\Tor(M)$ in the notation of Exercise 8 of Section 1).Prove that every finite abelian group is a torsion $\Z$-module. Give an example of an infinite abelian group that is a torsion $\Z$-module.
\end{problem}
\begin{solution}
Let $G$ be an abelian group. Then the nonzero element $|G|\in \Z$ annihilates $G$ and we conclude that $G$ is a torsion module. 

For an example of an infinite abelian group one can consider $\Q/\Z$. Every element has finite order and hence is annihilated by some integer. A less interesting example is any infinite product of finite abelian groups.
\end{solution}\oneperpage



\begin{problem}\label{ex:10.3.5}
Let $R$ be an integral domain. Prove that every finitely generated torsion $R$-module has a nonzero annihilator i.e., there is a nonzero element $r\in R$ such that $rm=0$ for all $m\in M$ --- here $r$ does not depend on $m$ (the annihilator of a module was defined in Exercise 9 of Section 1). Give an example of a torsion $R$-module whose annihilator is the zero ideal. 
\end{problem}
\begin{solution}
Let $M = RA$ for a finite set $\{a_1,\ldots, a_n\}$. For each $a_i$ let $r_i\neq 0$ be such that $r_ia_i = 0$. We claim that $r = r_1r_2\cdots r_n$ is a nonzero element of the annihilator of $M$ in $R$. That $r\neq 0$ follows from the fact that $R$ is an integral domain. To see that $r$ is in the annihilator of $M$ notice that $r$ annihilates each $a_i$ by the commutativity of $R$. Since $r$ annihilates a generating set for $M$ it must annihilate $M$, proving the result.

For an example where the annihilator is zero and the module is still torsion, consider the group $G$ which is the product of $\Z/n\Z$ for all $n\ge 2$, considered as a $\Z$-module. Every element is annihilated by the least common multiple of its nonzero components, but no nonzero integer can annihilate every element of $G$ simultaneously.
\end{solution}\oneperpage



\begin{problem}\label{ex:10.3.6}
Prove that if $M$ is a finitely generated $R$-module that is generated by $n$ elements then every quotient of $M$ may be generated by $n$ (or fewer) elements. Deduce that quotients of cyclic modules are cyclic. 
\end{problem}
\begin{solution}Let $\{a_1,\ldots, a_n\}$ be a generating set for an $R$-module $M$. Then we claim that $\{a_1+N, \ldots, a_n+N\}$ generates $M/N$ as an $R$-module. Indeed, notice that any $m+N\in M/N$ can be written as \[
m+N = \left(\sum r_i a_i\right) +N  = \sum r_i(a_i+N)
\]
proving the result. Hence the quotient of a cyclic module can be generated by 1 or 0 elements and is again cyclic. 

\end{solution}\oneperpage



\begin{problem}\label{ex:10.3.7}
Let $N$ be a submodule of $M$. Prove that if both $M/N$ and $N$ are finitely generated then so is $M$.
\end{problem}
\begin{solution}
Let $\{a_1,\ldots, a_n\}$ be a finite generating set for $N$ and let $\{b_1+N,\ldots, b_m+N\}$ be a finite generating set for $M/N$. We claim that the set \[
A = \{a_1,\ldots, a_n,b_1\ldots, b_m\}
\]
generates $M$ as an $R$-module. Let $\pi:M\to M/N$ be the natural projection map. For an arbitrary $m\in M$, let $r_1,\ldots, r_m$ be such that \[
m+N = \sum r_i (b_i+N) = \left(\sum r_i b_i \right) + N
\]
Now notice that $m-\sum r_ib_i$ must be in the kernel of $\pi$, i.e. in $N$. Then there must exist $s_1,\ldots, s_n$ so that \[
m-\sum r_ib_i = \sum s_j a_j
\]
which implies \[
m = \sum r_ib_i + \sum s_j a_j \in RA.
\]
Hence $RA = M$ and $A$ is a finite generating set for $M$. This proves the result.
\end{solution}\oneperpage



\begin{problem}\label{ex:10.3.8}
Let $S$ be the collection of sequences $(a_1,a_2,a_3,\ldots)$ of integers $a_1,a_2,a_3,\ldots$ where all but finitely many of the $a_i$ are 0 (called the \emph{direct sum} of infinitely many copies of $\Z$). Recall taht $S$ is a ring under componentwise addition and multiplication and $S$ does not have a multiplicative identity --- cf. Exercise 20, Section 7.1. Prove that $S$ is not finitely generated as a module over itself. 
\end{problem}
\begin{solution}
Given any finite set $A=\{a_1,\ldots, a_n\}$ let $n_i$ be an integer such that $N>n_i$ implies that the $N$-th component of $a_i$ is zero. Taking the maximum $M$ of all $n_i$ we see that every $a_i$ is zero past index $M$. Hence the set $A$ does not generate any list which is nonzero after $M$ and $A$ does not generate $S$ as amodule. 
\end{solution}\oneperpage



\begin{problem}\label{ex:10.3.9}
An $R$-module $M$ is called \emph{irreducible} if $M\neq 0$ and if $0$ and $M$ are the only submodules of $M$. Show that $M$ is irreducible if and only if $M\neq 0$ and $M$ is a cyclic module with any nonzero element as a generator. Determine all the irreducible $\Z$-modules. 
\end{problem}
\begin{solution}$(\Rightarrow)$ Suppose that $M$ is irreducible. We know by definition $M\neq 0$. Taking some nonzero $m\in M$, we see that $Rm$ is a nonzero submodule of $M$, and so $Rm = M$. This proves that $M$ is generated by any nonzero element.

$(\Leftarrow)$ Suppose $M\neq 0$ and $M$ is cyclic with any nonzero element as a generator. Then let $N\subseteq M$ be any nonzero submodule of $M$. Let $n\in N$ be nonzero and notice then that $M = Rn \subseteq N$ and so $N=M$. This proves that the only nonzero submodule of $M$ is $M$ itself and so $M$ is irreducible.

 To classify all irreducible $\Z$-modules,  we need only consider cyclic modules. If a cyclic module is not a torsion module it is isomorphic to $\Z$. But this is not irreducible since it contains a submodule isomorphic to $2\Z$. This leaves cyclic torsion modules of $\Z$. These are simply finite cyclic groups. Among these we see that the only irreducible ones are $\Z/p\Z$ for some prime $p$. 

\end{solution}\oneperpage



\begin{problem}\label{ex:10.3.10}
Assume $R$ is commutative. Show that an $R$-module $M$ is irreducible if and only if $M$ is isomorphic (as an $R$-module) to $R/I$ where $I$ is a maximal ideal of $R$. [By the previous exercise, if $M$ is irreducible then there is a natural map $R\to M$ defined by $r\mapsto rm$ where $m$ is any fixed nonzero element of $M$.]
\end{problem}
\begin{solution}
$(\Rightarrow)$ Suppose that $M$ is an irreducible $R$-module and fix some nonzero $m\in M$. We know that $Rm = M$. Let $\phi:R\to M$ be the map $r\mapsto rm$. This is certainly a homomorphism of $R$-modules and furthermore it is surjective. Thus we have $M \cong R/\ker \phi$. If we can show $ \ker\phi$ (as a submodule of $R$) is a maximal ideal of $R$ then we are done. First by virtue of being a submodule of the commutative ring $R$ we know that $\ker\phi$ is an ideal. 

 Next notice that $\ker\phi$ is exactly the annihilator of $m$ (and hence $M$) in $R$. Thus any ideal $J$ strictly containing $\ker\phi$ must contain some $r$ so that $rm\neq 0$. But then we have that $JM \neq 0$ and so $JM = M$. This means that $J$ contains some element $s$ so that $sm = m$, or equivalently $(s-1)m  = 0$. We conclude that $s-1\in \ker\phi$. But $\ker\phi\subseteq J$ and so we have $s,s-1\in J$ which means $1\in J$. We conclude that $J=R$ and so $I$ is maximal. This proves the result. 
 
 $(\Leftarrow)$ Suppose $M \cong R/I$ for a maximal ideal $I$. We aim to show that $Rm = M$ for all nonzero $m\in M$. We can write any nonzero $m\in M$ as $a+I$ via the isomorphism between $M$ and $R/I$ where $a\notin I$. But then $R(a+I) = Ra+RI = Ra+I$. Notice that $Ra+I$ is an ideal strictly containing $I$ since it contains $a$, and so $Ra+I = R$ since $I$ is maximal. We conclude that $R(a+I) = R/I$ in the module $R/I$, proving the result.
\end{solution}\oneperpage



\begin{problem}\label{ex:10.3.11}
Show that if $M_1$ and $M_2$ are irreducible $R$-modules, then any nonzero $R$-module homomorphism from $M_1$ to $M_2$ is an isomorphism. Deduce that if $M$ is irreducible then $\End_R(M)$ is a division ring (this result is called \emph{Schur's Lemma}). [Consider the kernel and the image.]
\end{problem}
\begin{solution}
Let $\phi:M_1\to M_2$ be a nonzero $R$-module homomorphism. We know that $\ker\phi$ is not all of $M_1$, and hence $\ker\phi = \{0\}$. On the other hand we know $\phi(M_1) \neq \{0\}$ and so it is all of $M_2$. This tells us that $\phi$ is injective and surjective, and so we conclude that $\phi$ is an isomorphism.
\end{solution}\oneperpage



\begin{problem}\label{ex:10.3.12}
Let $R$ be a commutative ring and let $A,B$ and $M$ be $R$-modules. Prove the following isomorphisms of $R$-modules:\begin{enumerate}
\item[(a)] $\Hom_R(A\times B,M)\cong \Hom_R(A,M)\times \Hom_R(B,M)$
\item[(b)] $\Hom_R(M,A\times B)\cong \Hom_R(M,A)\times \Hom_R(M,B)$.
\end{enumerate}
\end{problem}
\begin{solution}
(a)\\
Let $(\phi,\psi)\in \Hom_R(A,M)\times \Hom_R(B,M)$. We claim that the map sending $(\phi,\psi)\mapsto \Phi$ where $\Phi$ acts as $\phi$ in the first coordinate and $\psi$ in the second is an isomorphism of $R$-modules. More specifically, we define $\Phi(a,b) = \phi(a) + \psi(b)$. First we show that it is even a well defined map between the hom-sets of concern, namely that $\Phi\in \Hom_R(A\times B,M)$. This is straightforward since if $(a,b),(a',b')\in A\times B$ and $r\in R$ then we have \begin{align*}
\Phi((a,b)+r(a',b')) &= \Phi(a+ra',b+rb') \\&= \phi(a+ra')+\psi(b+rb')\\& = \phi(a)+\psi(b) + r\phi(a') +r\psi(b') \\&= \Phi(a+b) + r\Phi(a'+b').
\end{align*}
Now we need to show that $(\phi,\psi)\mapsto \Phi$ is a homomorphism of $R$-modules. Suppose we have $(\phi,\psi)\mapsto \Phi$ and $(\phi',\psi')\mapsto \Phi'$. It is clear that $(\phi+r\phi', \psi+r\psi')$ maps to $\Phi + r\Phi'$ and so  we see that the map is an $R$-module homomorphism. 

Injectivity is straightforward since the only map $\Phi$ that can act as zero in both coordinates comes from $\phi=\psi = 0$. For surjectivity, notice that any $\Phi$ acts as an $R$-module homomorphism in each coordinate. In particular, if we define $\Phi_A:A\to M$ by $\Phi_A(a) = \Phi(a,0)$ and $\Phi_B$ symmetrically then we see that $(\Phi_A,\Phi_B)\mapsto \Phi$. Hence the map is surjective and we conclude the desired isomorphism.\\\\
(b)\\
The proof here is essentially the same as (a): the isomorphism is given by decomposing any homomorphism in $\Hom_R(M,A\times B)$ into its coordinate pieces on $A$ and $B$. In particular, we associate $\Phi\in \Hom_R(M,A\times B)$ with the pair $(\phi,\psi)$ where $\phi(a) $ is the first coordinate of $\Phi(a)$ and $\psi(b)$ is the second coordinate of $\Phi(b)$. 
\end{solution}\oneperpage



\begin{problem}\label{ex:10.3.13}
Let $R$ be a commutative ring and let $F$ be a free $R$-module of finite rank. Prove the following isomorphism of $R$-modules: $\Hom_R(F,R)\cong F$. 
\end{problem}
\begin{solution}Write $F\cong R^n$. Applying the result of the previous exercise we have that \begin{align*}
\Hom_R(F,R) &\cong \Hom_R(R^n,R)\\
&\cong \Hom_R(R,R)^n\\
& \cong R^n\\
&\cong F.
\end{align*}
Note that above we have used the fact that $\Hom_R(R,R) \cong R$, which was proven in another exercise.

\end{solution}\oneperpage



\begin{problem}\label{ex:10.3.14}
Let $R$ be a commutative ring and let $F$ be the free $R$-module of rank $n$. Prove that $\Hom_R(F,M)\cong M\times \cdots\times M$ ($n$ times). [Use Exercise 9 in Section 2 and Exercise 12.]
\end{problem}
\begin{solution}Recall from 10.2.9 that $\Hom_R(R,M) \cong M$ since every homomorphism is determined by its value on $1\in R$. Hence we have (similar to the previous exercise) that \begin{align*}
\Hom_R(F,M) &\cong \Hom_R(R^n,M)\\
&\cong \Hom_R(R,M)^n\\
& \cong M^n.
\end{align*}
This is exactly what we hoped to show. Note that the previous exercise is in fact a special case of this. 

\end{solution}\oneperpage



\begin{problem}\label{ex:10.3.15}
An element $e\in R$ is called a \emph{central idempotent} if $e^2=e$ and $er = re$ for all $r\in R$. If $e$ is a central idempotent in $R$, prove that $M= eM\oplus (1-e)M$. [Recall Exercise 14 in Section 1.]
\end{problem}
\begin{solution}We must show two things. First, that $M$ is generated by $eM$ together with $(1-e)M$. Second, that $eM\cap (1-e)M = \{0\}$. 

For the first statement, notice that since $e$ is in the center of $R$ the subsets $eM = \{em\mid m\in M\}$ and $(1-e)M = \{(1-e)m\mid m\in M\}$ are indeed submodules of $M$. Then let $m\in M$ be arbitrary and notice that \[
m = 1\cdot m = (e+(1-e))\cdot m = e\cdot m + (1-e)\cdot m
\]
and so $eM$ together with $(1-e)M$ generates $M$. Note this is independent of $e$ being central idempotent. 

To prove that the sum of $eM$ and $(1-e)M$ is direct, suppose that $m\in eM\cap (1-e)M$. Then there exist $m_1,m_2\in M$ so that \[
em_1 = m = (1-e)m_2.
\]
But acting on the left and right quantities above by $e$ yields \[
e^2m_1 = (e-e^2)m_2
\]
which simplifies to $em_1 = 0$ since $e-e^2=0$. But this tells us immediately that $m = 0$, so the sum of $eM$ and $(1-e)M$ is direct. 
\end{solution}\oneperpage


\quad\\The next two exercises establish the Chinese Remainder Theorem for modules (cf. Section 7.6).

\begin{problem}\label{ex:10.3.16}
For any ideal $I$ of $R$ let $IM$ be the submodule defined in Exercise 5 of Section 1. Let $A_1,\ldots,A_k$ be any ideals in the ring $R$. Prove that the map \[
M\to M/A_1M\times\cdots  \times M/A_k M\quad\text{ defined by }\quad m\mapsto (m+A_1M,\ldots,m+A_kM)
\]
is an $R$-module homomorphism with kernel $A_1M\cap A_2M\cap \cdots \cap A_kM$.
\end{problem}
\begin{solution}Suppose that $m$ is in the kernel of this map. Then $m+A_iM$ is zero for all $A_i$ in the quotient. In particular $m\in A_iM$ for all $i$. Since this condition is necessary and sufficient we conclude that the kernel is indeed $A_1M\cap A_2M\cap\cdots \cap A_kM$. 

\end{solution}\oneperpage



\begin{problem}\label{ex:10.3.17}
In the notation of the preceding exercise, assume further that the ideals $A_1,\ldots A_k$ are pairwise comaximal (i.e. $A_i+A_j = R$ for all $i\neq j$). Prove that \[
M/(A_1\cdots A_k)M\cong M/A_1M\times \cdots\times MA_kM.
\]
[See the proof of the Chinese Remainder Theorem for rings in Section 7.6.]
\end{problem}
\begin{solution} \noindent We will show that $A_1M\cap \cdots \cap A_kM = (A_1\cdots A_k)M$.   First note that $A_1\cdots A_k\subseteq A_i$ for all $i$ since each $A_i$ is an ideal and hence absorbs multiplication by other ideals on both the left and right. This tells us that inclusion in the $\supseteq$ direction holds.

For inclusion in the other direction, we will proceed by induction on $k$. For $k=1$ we have only one ideal and the inclusion is obvious. For the inductive step we assume that $A_2M\cap \cdots \cap A_{k}M = (A_2\cdots A_k)M$. From this we have that $A_1M\cap A_2M\cap\cdots \cap A_kM = A_1M\cap (A_2\cdots A_k)M$. Now notice that $A_1$ and $A_2\cdots A_k$ are comaximal. Indeed, for each $2\le i \le k$ we are guaranteed that there is some $a_i\in A_1$ and $a_i'\in A_i$ so that $1=a_i+a_i'$. We then have that $1=(a_1+a_1')(a_2+a_2')\cdots(a_k+a_k')$ is an element of $A_1+(A_2\cdots A_k)$ since every term in the expanded product is a product of something in $A_1$ with something in $A_2\cdots A_k$. This tells us that $A_1$ and $A_2\cdots A_k$ are comaximal and so we can write $1 = a+a'$ with $a\in A_1$ and $a'\in A_2\cdots A_k$. But this means that $ A_1\cap A_2\cap\cdots \cap A_k \subseteq A_1A_2\cdots A_k $ since for any $b$ in the intersection we have $b = 1b = (a+a')b = ab+a'b = ab+ba'$ which is in $A_1(A_2\cdots A_k)$ since $a\in A_1$, $a'\in A_2\cdots A_k$ and $b$ is in both. (Here we have assumed $R$ is commutative to write $a'b = ba'$, which seems necessary.)

Putting all of this together, we may write \begin{align*}
A_1M\cap A_2M\cap\cdots \cap A_kM & =  A_1M\cap (A_2\cdots A_k)M\\
& \subseteq A_1M \cap (A_2\cap A_3\cap\cdots \cap A_k) M &&\text{We are intersecting with the}\\
&&&\text{ \emph{larger} submodule $(A_2\cdots A_k)M$.}\\
& \subseteq A_1M\cap A_2M\cap \cdots \cap A_kM&&\text{$(A\cap B)M\subseteq AM\cap BM$ in general.}
\end{align*}
This concludes the proof that $A_1M\cap \cdots \cap A_kM = (A_1\cdots A_k)M$.

Now we show that the map defined in the previous problem is surjective. Call this map $\phi$ for convenience. We will proceed by induction on $k$. When $k=2$ we have by the comaximality of $A_1$ and $A_2$ that there are some $a_1\in A_1$ and $a_2\in A_2$ so that $1=a_1+a_2$. To show the map is surjective it suffices to demonstrate a preimage for $(m+A_1,0)$ and $(0,m+A_2)$ for all $m\in M$. For this, notice that \[\phi(a_1m) = (0, a_1m+A_2) = (0,(1-a_2)m+A_2) = (0,m-a_2m + A_2) = (0,m+A_2)\]
and similarly \[
\phi(a_2m) = (a_2m+A_1, 0) = ((1-a_1)m+A_1,0) = (m-a_1m+A_1,0) = (m+A_1,0). 
\]
Hence the map is surjective in this base case. For the inductive step, the inductive hypothesis tells us that the map is surjective onto all elements of the form $(m_1+A_1,m_2+A_2M,\cdots, a_k+A_kM)$ where all values of $m_2,\ldots, m_k$ are achieved but possibly not all values of $m_1$. To show that the map is surjective it thus suffices to find preimages for $(m_1+A_1,0,\ldots, 0)$ for all possible $m_1$. Recall we showed in earlier in our solution that $A_1$ and $A_2\cdots A_k$ are comaximal. Hence we can write $1 = a+a'$ for $a\in A_1$ and $a'\in A_2\cdots A_k$. Then notice that \begin{align*}
\phi(a'm_1) &= (a'm_1+A_1M, a'm_1+A_2M,\ldots, a'm_1+A_kM)\\
& = (m_1-am_1 +A_1M, 0,\ldots,0)&&\text{Since $a'\in A_i$ for all $2\le i \le k$}\\
& = (m_1+A_1M, 0,\ldots, 0).
\end{align*}
Hence the image of $M$ under $\phi$ is all of $M/A_1M\times \cdots \times M/A_kM$. Now, by the first isomorphism theorem for modules we have that \[
M/\ker\phi \cong M/A_1M\times \cdots \times M/A_kM 
\]
But $\ker\phi = A_1M\cap \cdots \cap A_kM$, and earlier in the solution we proved that $A_1M\cap\cdots \cap A_kM = (A_1\cdots A_k)M$. We conclude that \[
M/(A_1\cdots A_k) \cong M/A_1M\times \cdots \times M/A_kM 
\]
as desired.
\end{solution}\oneperpage



\begin{problem}\label{ex:10.3.18}
Let $R$ be a Principal Ideal Domain and let $M$ be an $R$-module that is annihilated by the nonzero, proper ideal $(a)$. Let $a = p_1^{\alpha_1}p_2^{\alpha_2}\cdots p_k^{\alpha_k}$ be the unique factorization of $a$ into distinct prime powers in $R$. Let $M_i$ be the annihilator of $p_i^{\alpha_i}$ in $M$, i.e. $M_i$ is the set $\{m\in M\mid p_i^{\alpha_i}m = 0\}$---called the \emph{$p_i$-primary component of $M$}. Prove that \[
M = M_1\oplus M_2\oplus \cdots \oplus M_k.
\]
\end{problem}
\begin{solution}
 We first claim that the sum of the various $M_i$ is direct. To show this we prove that $M_i\cap (\sum_{j\neq i} M_j) = 0$ for all $i$. Let $m_i\in M_i$ and assume $m_i\in \sum_{j\neq i} M_j$. Then we see that $m_i$ is annihilated by $p_i^{\alpha_i}$ as well as $\prod_{j\neq i} p_j^{\alpha_j}$. Hence these are both elements of the ideal annihilating $m_i$. This ideal then contains their greatest common divisor, which is clearly 1 since $p_i^{\alpha_i}$ does not share any prime factors with $\prod_{j\neq i} p_j^{\alpha_j}$. Thus we have $m_i = 1m_i = 0$, proving that the sum of the $M_i$ is direct. 
 
 

It remains to show that the sum of the various $M_i$ is in fact all of $M$. We have from the previous problem that \[
M\cong M/(a)M \cong M/(p_1^{\alpha_1})M \times M/(p_2^{\alpha_2})M\times\cdots\times M/(p_k^{\alpha_k})M
\]
since $(a)M = 0$ and the various ideals $(p_i^{\alpha_i})$ are comaximal by virtue of the various $p_i$ being distinct prime factors. In the previous problem we saw that the isomorphism is given by the natural map \[\phi(m) = (m+p_1^{\alpha_1}M, m+p_2^{\alpha_2}M,\ldots, m+p_k^{\alpha_k}M).\]
We want to show that this map, when restricted to the direct sum $M_1\oplus\cdots \oplus M_k\subseteq M$ is still surjective. In fact it suffices to show that the inverse of $\phi$ maps into this direct sum. To prove this we need only show that the tuple with $m+(p_i^{\alpha_i})M$ in the $i$-th component and zeroes elsewhere maps into the direct sum of the $M_i$, since the set of all vectors of this form generate the direct product. The image of such a tuple will be some $m'\in M$ which is zero modulo $p_j^{\alpha_j}M$ for all $j\neq i$. In particular, we have that $m' =q_i m''$ for some $m''\in M$, where $q_i = a/p_i^{\alpha_i}$. But then clearly $m'\in M_i$ since \[
p_i^{\alpha_i} m = p_i^{\alpha_i}q_i m'' = am'' = 0.
\]
This proves that $\phi^{-1}$ maps to values in the direct sum of the various $M_i$ and so the direct sum is itself $M$. This concludes the proof.

 
  


 %We first prove that the various $M_i$ intersect trivially. Let $q_i$ be $a/p_i^{\alpha_i}$ where division means removing $p_i^{\alpha_i}$ from the product, and observe that $M_i \subseteq (q_i)M$. This is clear since $p_i^{\alpha_i}q_i = a$ annihilates all of $M$. 
 
 %We will use the result of the previous problem. We know that \[
%M = M/(a)M \cong M_1\times M_2\times\cdots \times M_k
%\]
%via the map $m\mapsto (m,m,\ldots, m)$ (note that the first equality follows from the fact that $M$ is annihilated by $(a)$).
\end{solution}\oneperpage



\begin{problem}\label{ex:10.3.19}
Show that if $M$ is a finite abelian group of order $a=p_1^{\alpha_1}p_2^{\alpha_2}\cdots p_k^{\alpha_k}$ then, considered as a $\Z$-module, $M$ is annihilated by $(a)$, the $p_i$-primary component of $M$ is the unique Sylow $p_i$-subgroup of $M$ and $M$ is isomorphic to the direct product of its Sylow subgroups.
\end{problem}
\begin{solution}That $M$ is annihilated by $a$ follows immediately from Lagrange's theorem. We know that Sylow $p_i$-subgroups are unique since $M$ is an abelian group and all Sylow subgroups are conjugates of one another. This leaves that the $p_i$-primary component is indeed a Sylow $p_i$-subgroup. Let $H_i$ be the Sylow $p_i$-subgroup guaranteed by the Sylow theorems. Each element in this group has order dividing $p_i^{\alpha_i}$, since this is the order of $H_i$ and so we see right away that $H_i\subseteq M_i$ where $M_i$ is the annihilator of $p_i^{\alpha_i}$ as described in the previous problem statement. To prove containment in the other direction, suppose $m\in M$ is annihilated by $p_i^{\alpha_i}$. Then its order is a power of $p_i$ and so it must be in $H_i$. Hence the Sylow $p_i$-subgroups are exactly the $p_i$-primary components of $M$ . By the previous problem we conclude that $M$ is the direct sum of its Sylow $p_i$-subgroups.

\end{solution}\oneperpage



\begin{problem}\label{ex:10.3.20}
Let $I$ be a nonempty index set and for each $i\in I$ let $M_i$ be an $R$-module. The \emph{direct product} of the modules $M_i$ is defined to be their direct product as abelian groups (cf. Exercise 15 in Section 5.1) with the action of $R$ componentwise multiplication. The \emph{direct sum} of the modules $M_i$ is defined to be the restricted direct product of the abelian groups $M_i$ (cf. Exercise 17 in Section 5.1) with the action of $R$ componentwise multiplication. In other words, the direct sum of the $M_i$'s is the subset of the direct product $\prod_{i\in I}M_i$, which consists of all elements $\prod_{i\in I}m_i$ such that only finitely many of the components $m_i$ are nonzero; the action of $R$ on the direct product or direct sum is given by $r\prod_{i\in I}m_i = \prod_{i\in I}rm_i$ (cf. Appendix I for the definition of the Cartesian products of infinitely many sets). The direct sum will be denoted by $\oplus_{i\in I} M_i$. \begin{enumerate}
\item[(a)] Prove that the direct product of the $M_i$'s is an $R$-module and the direct sum of the $M_i$'s is a submodule of their direct product.
\item[(b)] Show that if $R=\Z$, $I=\Z^+$ and $M_i$ is the cyclic group of order $i$ for each $i$, then the direct sum of the $M_i$'s is not isomorphic to their direct product. [Look at torsion.]
\end{enumerate}
\end{problem}
\begin{solution}
(a)\\
We know already that the direct product of the $M_i$ will be an abelian group. Thus we only have to verify that the action of $r$ satisfies the module axioms. If $\prod_{i\in I}m_i$ and $\prod_{i\in I}m_i'$ are two elements in $\oplus_{i\in I} M_i$ then we observe that \begin{align*}
r\left(\prod_{i\in I}m_i+\prod_{i\in I}m_i'\right) &=r\prod_{i\in I}(m_i+m_i')\\
&=\prod_{i\in I}r(m_i+m_i')\\
&=\prod_{i\in I}(rm_i+rm_i')\\
& = \prod_{i\in I}rm_i+\prod_{i\in I}rm_i'\\
& = r\prod_{i\in I}m_i+r\prod_{i\in I}m_i'.
\end{align*}
If $s\in R$ then we can also verify that \[
(sr)\prod_{i\in I} m_i = \prod_{i\in I}(sr)m_i = \prod_{i\in I}s(rm_i) = s\prod_{i\in I}rm_i = s\left(r\prod_{i\in I}m_i\right)
\]
and furthermore\begin{align*}
(s+r)\prod_{i\in I}m_i &= \prod_{i\in I}(s+r)m_i\\
& = \prod_{i\in I}(sm_i+rm_i)\\
& = \prod_{i\in I}sm_i + \prod_{i\in I}rm_i\\
& = s\prod_{i\in I}m_i + r\prod_{i\in I}m_i.
\end{align*}
Finally, if $R$ has unity then we see the direct product is unital since $1\prod_{i\in I}m_i = \prod_{i\in I} 1m_i = \prod_{i\in I} m_i$.

Next we show that the direct sum is a submodule. We know it is a subgroup, and so we only need to check it is invariant under the action of $R$. This is immediate because $r\prod_{i\in I}m_i = \prod_{i\in I}rm_i$ will have no more nonzero terms than $\prod_{i\in I}m_i$, since $r0 = 0$. \\\\
(b)\\
We claim that the direct sum is a torsion module but the direct product is not. For any element $\prod_{i\in I}m_i$ we see that it is annihilated by the product of the $i$ for which $m_i\neq 0$. Hence the direct sum is torsion.  Of course we cannot do this when there are infinitely many nonzero $m_i$. In particular the element $\prod_{i\in I} 1$ in the direct product is not annihilated by any nonzero integer since no nonzero integer is congruent to zero mod every other integer. 
\end{solution}\oneperpage



\begin{problem}\label{ex:10.3.21}
let $I$ be a nonempty index set and for each $i\in I$ let $N_i$ be a submodule of $M$. Prove that the following are equivalent:\begin{enumerate}
\item[(i)] the submodule of $M$ generated by all the $N_i$'s is isomorphic to the direct sum of the $N_i$'s
\item[(ii)] if $\{i_1,i_2,\ldots,i_k\}$ is any finite subset of $I$ then $N_{i_1}\cap (N_{i_2}+\cdots +N_{i_k}) = 0$
\item[(iii)] if $\{i_1,i_2,\ldots,i_k\}$ is any finite subset of $I$ then $N_{i_1}+\cdots +N_{i_k} = N_{i_1}\oplus \cdots \oplus N_{i_k}$
\item[(iv)] for every element $x$ of the submodule of $M$ generated by the $N_i$'s there are unique elements $a_i\in N_i$ for all $i\in I$ such that all but a finite number of the $a_i$ are zero and $x$ is the (finite) sum of the $a_i$. 
\end{enumerate}
\end{problem}
\begin{solution}
\emph{(Note: Referenced \url{https://crazyproject.wordpress.com/aadf/} to clarify details since it was not clear to me whether the direct sum of the $N_i$'s was internal or not.)}\\\\
$( i\Rightarrow ii)$ Suppose $\sum_{i\in I} N_i \cong \oplus_{i\in I} N_i$ via the natural isomorphism (which we will denote $\phi$) and let $\{i_1,i_2,\ldots, i_k\}\subseteq I$. Then consider the intersection of the submodules $N_{i_1}$ and $(N_{i_2}+\cdots +N_{i_k})$ of $M$. Let $n$ be in this intersection. Then we know $n = a_1$ for some $a_1\in N_{i_1}$ and also that $n= \sum_{j=2}^k a_j$ for $a_j\in N_{i_j}$. Then $a_1 - \sum_{j=2}^k a_j$ is in the kernel of $\phi$ since it is zero. But its image under $\phi$ maps $a_1$ and each $a_j$ to a separate coordinate, and hence each coordinate must be zero. In particular $a_1$ and all $a_j$ are zero. Hence $n=0$. 

$(ii\Rightarrow iii)$ This is immediate since the property that $N_{i_1}\cap (N_{i_2}+\cdots + N_{i_k}) = 0$ is the definition of being an (internal) direct sum.

$(iii\Rightarrow iv)$ Suppose $x\in \sum_{i\in I} N_i$. Then by definition of being generated by the $N_i$ we know that there exists a finite collection of nonzero $a_{i_1},\ldots, a_{i_k}$ with $a_{i_j}\in N_{i_j}$ and \[
x = \sum_{j=1}^k a_{i_j}.
\]
Since the sum of the $N_{i_j}$ is direct we know this representation is unique within the module generated by the $N_{i_j}$. We must show it is unique overall. Given another representation of $x$ as \[
x = \sum_{j=1}^{k'} a'_{i'_j}.
\]we consider it in the submodule generated by all $N_{i_j}$ and $N_{i'_j}$. This is still a finite sum of submodules and so by (iii) we see that these representations must be the same. Hence the representation of $x$ as a sum of elements in various $N_{i_j}$ is unique.

$(iv \Rightarrow i)$ Let $\phi: \oplus_{i\in I} N_i \to \sum_{i\in I} N_i$ be the natural projection map. This is clearly a surjective $R$-module homomorphism, and we want to show it is injective. For any $n\in \sum_{i\in I} N_i$ we have from (iv) a unique representation of it as a sum of finitely many nonzero $a_i$ with $a_i\in N_i$. This is equivalent to $n$ having a unique preimage under $\phi$ since $\phi$ is by definition the sum of components map. Hence $\phi$ is injective, and an isomorphism. 
\end{solution}\oneperpage



\begin{problem}\label{ex:10.3.22}
Let $R$ be a Principal Ideal Domain, let $M$ be a torsion $R$-module (cf. Exercise 4) and let $p$ be a prime in $R$ (do not assume $M$ is finitely generated, hence it need not have a nonzero annihilator --- cf. Exercise 5). The \emph{$p$-primary component} of $M$ is the set of all elements of $M$ that are annihilated by some positive power of $p$.
\begin{enumerate}
\item[(a)] Prove that the $p$-primary component is a submodule. [See Exercise 13 in Section 1.]
\item[(b)] Prove that this definition of $p$-primary component agrees with the one given in Exercise 18 when $M$ has a nonzero annihilator.
\item[(c)] Prove that $M$ is the (possible infinite) direct sum of its $p$-primary components, as $p$ runs over all primes of $R$.
\end{enumerate}
\end{problem}
\begin{solution}
(a)\\
Note that $m$ is annihilated by some power of $p$ if and only if it is annihilated by the principal ideal $(p^k) = (p)^k$ for some $k$. Exercise 10.1.13 tells us immediately that that the $p$-primary component is a submodule. \\\\
(b)\\
Suppose $M$ has a nonzero annihilator generated by $a = p_1^{\alpha_1}\cdots p_k^{\alpha_k}$. Then we want to show that $m$ is annihilated by some power of $p_i$ if and only if it is annihilated by $p_i^{\alpha_i}$. The reverse implication is trivial, and so we show that if $p_i^km = 0$ then $p_i^{\alpha_i}m = 0$. Notice that $(p_i^k)$ contains $a$ and hence $a$ is a multiple of $p_i^k$. In particular, $k\le \alpha_i$ and so $p_i^{\alpha_i}m = p_{i}^{\alpha_i-k}p_i^{k}m = p_i^{\alpha_i-k} 0 = 0$. This proves the desired statement.\\\\
(c)\\
We first show that every element of $M$ can be written as a finite sum of $p$-primary elements. Given $m\in M$ we know there is some nonzero $r\in R$ so that $rm = 0$ since $M$ is a torsion module. Let $r=p_1^{\alpha_1}\cdots p_k^{\alpha_k}$ be the unique factorization of $r$ in the PID $R$. Then define $q_i =\prod_{j\neq i} p_j^{\alpha_j}$ (that is $q_i$ is $r$ with $p_i^{\alpha_i}$ removed as a factor). Notice that taken together all $q_i$ have a gcd of one, and hence the ideal generated by them is all of $R$. This tells us that there exist $r_1,\ldots, r_k$ so that \[
1 = r_1q_1+\cdots + r_kq_k.
\]
We then have for any $m\in M$ that \[
m = 1m = r_1q_1m + r_2q_2m + \cdots r_kq_km.
\]
Now notice that $r_iq_im$ is annihilated by $p_i^{\alpha_i}$, and in particular is in the $p_i$-primary component of $M$. This shows that the internal sum of all $p$-primary components is in fact $M$. 

Next we show that the sum is direct. We will show in particular that condition (ii) given in Exercise 10.3.21 is satisfied by the various $p$-primary components. Suppose that $M_1,\ldots, M_k$ is a finite collection of distinct $p$-primary components, with associated primes $p_1,\ldots, p_k$. We want to show that if $m\in M_1$ and $m\in \sum_{i=2}^k M_i$ then $m$ is zero. Let $p_1^{\alpha_1}$ be a power of $p_1$ annihilating $m$. Then also write \[
m = m_2+\cdots +m_k
\]
with $m_i\in M_i$ and let $p_i^{\alpha_i}$ be a power of $p_i$ annihilating $m_i$. Then notice that $p = p_2^{\alpha_2}\cdots p_k^{\alpha_k}$ annihilates $m$ since it annihilates each $m_i$ in the sum. Now consider the ideal containing all elements of $R$ that annihilate $m$. We see that $p_1^{\alpha_1}$ and $p$ are both in this ideal, and so must be their greatest common divisor, namely 1. Thus we have $m = 1m = 0$, as desired. This proves that the sum of all $p$-primary components is direct, and we conclude the desired result.

\end{solution}\oneperpage



\begin{problem}\label{ex:10.3.23}
Show that any direct sum of free $R$-modules is free.
\end{problem}
\begin{solution}Let $\{M_i\}_{i\in I}$ be a collection of free $R$-modules, each with basis $A_i$. We claim that $\bigoplus_{i\in I} M_i$ is free over $\bigcup_{i\in I} A_i$. Letting $m\in \bigoplus_{i\in I}$ we know we can write $m$ as a finite sum $m = m_{i_1} + \cdots + m_{i_k}$ with $m_{i_j}\in M_{i_j}$. Furthermore this expression of $m$ is unique since the coordinates in a direct sum are independent. But each $m_{i_j}$ has a unique representation over the basis $A_{i_j}$. Hence we can express $m$ over the basis $\bigcup_{i\in I} A_i$, and furthermore this representation of $m$ is unique. This proves the result. Note that we relied on the direct \emph{sum} structure (as opposed to direct product) so that we could write $m$ as a finite sum of elements in the various $M_i$. 

\end{solution}\oneperpage



\begin{problem}\label{ex:10.3.24}
\emph{(An arbitrary direct product of free modules need not be free)} For each positive integer $i$ let $M_i$ be the free $\Z$-module $\Z$, and let $M$ be the direct product $\prod_{i\in\Z^+}M_i$ (cf. Exercise 20). Each element of $M$ can be written uniquely in the form $(a_1,a_2,a_3,\ldots)$ with $a_i\in \Z$ for all $i$. Let $N$ be the submodule of $M$ consisting of all such tuples with only finitely many nonzero $a_i$. Assume $M$ is a free $\Z$ module with basis $\cB$. \begin{enumerate}
\item[(a)] Show that $N$ is countable.
\item[(b)] Show that there is some countable subset $\cB_1$ of $\cB$ such that $N$ is contained in the submodule, $N_1$, generated by $\cB_1$. Show also that $N_1$ is countable.
\item[(c)] Let $\overline{M} = M/N_1$. Show that $\overline{M}$ is a free $\Z$-module. Deduce that if $\overline{x}$ is any nonzero element of $\overline{M}$ then there are only finitely many distinct positive integers $k$ such that $\overline{x} = k\overline{m}$ for some $m\in M$ (depending on $k$).
\item[(d)] Let $\cS = \{(b_1,b_2,b_3,\ldots)\mid b_i=\pm i!\text{ for all } i\}$. Prove that $\cS$ is uncountable. Deduce that there is some $s\in \mathcal{S}$ with $s\notin N_1$.
\item[(e)] Show that the assumption $M$ is free leads to a contradiction: By (d) we may choose $s\in\cS$ with $s\notin N_1$. Show that for each positive integer $k$ there is some $m\in M$ with $\overline{s} = k\overline{m}$, contrary to (c). [Use the fact that $N\subseteq N_1$.]
\end{enumerate}
\end{problem}
\begin{solution}
(a)\\
We will show that $N$ is a countable union of countable sets. For each $i\in \Z_{\ge 0}$ define $N_i$ to be the set of elements in $N$ with nonzero entries only in indices less than or equal to $i$. Clearly $N$ is the union of all $N_i$, of which there are countably many. To see that $N_i$ is countable notice that it is (effectively) a finite product of countable sets, i.e. $N_i = \Z^i$, and hence is countable.\\\\
(b)\\
For each $n\in N$, we know that $n$ can be expressed as an $\Z$-linear combination of a finite number of basis elements. For each $n$ let $\mathcal B_n$ be the set of these basis elements. Then the set $\mathcal B_1 = \bigcup \mathcal B_n$ is a countable union of finite sets, and hence countable, and furthermore the submodule generated by it clearly contains $N$.

To see that $N_1$ is countable note that $N_1$ is exactly all finite $\Z$-linear combinations of elements in $\mathcal B_1$, and so it will suffice show that we can count all such combinations. But each combination is uniquely determined by a countable tuple containing elements of $\Z$ with only finitely many nonzero entries (each coordinate corresponds to some $b\in\mathcal B_1$ and the integers give weights to each). The set of such tuples is countable since it is in fact exactly $N$. \\\\
(c)\\
We know that $M$ is free over $\mathcal B$ and since $N_1$ is generated by $\mathcal B_1\subseteq \mathcal B$ taking the quotient $M/N_1$ gives us a module isomorphic to the free module over $\mathcal B\setminus \mathcal B_1$. Indeed, the quotient corresponds to simply zeroing out the basis elements in $\mathcal B_1$.

For some $\overline x \in\overline M$ we know then that $\overline x$ is a finite $\Z$-linear combination of basis elements for $\overline M$. Among these finitely many integers we may choose one with  a maximum absolute value $k$, and notice that for any $|k'|>|k|$ we cannot write $\overline x = k'\overline m$ for any $m\in M$, since this would be writing $\overline x$ as a strictly different $\Z$-linear combination of basis elements (after writing $\overline m$ in terms of the basis for $\overline M$) and this would contradict that $\overline M$ is free.\\\\
(d)\\
To show that $\mathcal S$ is uncountable, we notice that every element of $\mathcal S$ can be uniquely associated with a countable binary vector, where $0$ corresponds to $b_i = i!$ and 1 corresponds to $b_i = -i!$. It suffices to prove that this set of binary vectors is uncountable. This can be done straightforwardly via diagonalization. In particular, if we are given an enumeration $v_1,v_2,\ldots,$ of these vectors then we can construct a vector with disagrees with each $v_i$, in particular in the $i$-th coordinate. Since $N_1$ is countable we know there is some $s\in \mathcal S$ with $s\notin N_1$. \\\\
(e)\\
Choose $s\in\mathcal S$ so that $s\notin N_1$. Then $\overline s$ is nonzero in $\overline M$. Furthermore, we see in $\overline M$ that elements whose coordinates differ in a finite number of entries are the same, since their difference is in $N\subseteq N_1$. With this in mind, let $s_k$ be the tuple whose first $k$ entries are exactly $k!$, and whose following entries agree with $s$. We notice that $\overline s = \overline s_k$ for all $k$. 
Furthermore, $s_k = km_k$ for some $m_k\in M$ since each entry is divisible by $k$. We conclude that $\overline s = k\overline m_k$ for all $k$. But this implies that $\overline s = 0$ by (c), a contradiction. This concludes the proof.

\end{solution}\oneperpage



\begin{problem}\label{ex:10.3.25}
In the construction of direct limits, Exercise 8 of Section 7.6, show that if all $A_i$ are $R$-modules and the maps $\rho_{ij}$ are $R$-module homomorphisms, then the direct limit $A=\varinjlim A_i$ may be given the structure of an $R$-module in a natural way such that the maps $\rho_i:A_i\to A$ are all $R$-module homomorphisms. Verify the corresponding universal property (part (e)) for $R$-module homomorphism $\phi_i:A_i\to C$ commuting with the $\rho_{ij}$. 
\end{problem}
\begin{solution}
We know from the construction that $A$ is an abelian group, and so we need only show that there is an action of $R$ on $A$ that makes it an $R$-module, and that the various $\rho_i$ are $R$-module homomorphisms. If $[a]\in A$ is some equivalence class of elements in the various $A_i$ and $r\in R$ then we will define the action by $r[a] = [ra]$. First we show this is well defined. If $[a_i] = [a_j]$ then we know there is some $k$ so that $\rho_{ik}(a_i) = \rho_{jk}(a_j)$. Since $\rho_{ik}$ and $\rho_{jk}$ are $R$-module homomorphisms we have that  \[\rho_{ik}(ra_i) = r\rho_{ik}(a_i) = r\rho_{jk}(a_j) = \rho_{jk}\]
which proves that $[ra_i] = [ra_j]$. Hence the action is well defined. Since we may choose arbitrary representatives for the equivalence classes it is clear that the action gives $A$ an $R$-module structure, which arises from the action of $R$ on the representatives for the equivalence classes.

Next we show that $\rho_i:A_i\to A$ is an $R$-module homomorphism for all $i$. Recall that $\rho_i(a_i) = [a_i]$. Then we have directly by definition of the action of $R$ on $A$ that \[
\rho_i(ra_i) = [ra_i] = r[a_i] = r\rho_i(a_i)
\]
and so clearly each $\rho_i$ is an $R$-module homomorphism. 

Finally, we must show that the universal property holds. The universal property for the direct limit of modules is as follows: If $C$ is an $R$-module  such that for each $i\in I$ there is an $R$-module homomorphism $\phi_i: A_i\to C$ with $\phi_i = \phi_j\circ \rho_{ij}$ for all $j\le i$, then there is a unique $R$-module homomorphism $\phi:A\to C$ such that $\phi\circ \rho_i = \phi_i$ for all $i$ (where $\rho_i$ is the natural map from $A_i$ to $A$). That is, there exists a unique $\phi$ so that the diagram below commutes for all $i$:\[
\begin{CD}
A_i @>\rho_i >> A\\
@VVV @VV\phi V\\
@>\phi_i>> C
\end{CD}
\]
(The CD package doesn't allow for diagonal arrows, sorry).

Under the given hypotheses constructing a candidate for $\phi$ is not difficult: For each equivalence class $[a]\in A$, fix some $a_i \in A_i$ so that $\rho_i(a_i) = [a]$. Then define $\phi([a]) = \phi_i(a_i)$. We first argue that this is well defined. Suppose that we choose different representatives for some $[a]$, so that $a_i\in A_i$ and $a_j\in A_j$ had the property $\rho_i(a_i) = [a]$ and $\rho_j(a_j) = [a]$. Since $[a] = [a_i] = [a_j]$ we have that there is some $k$ so that $i\le k$, $j\le k$  and \[
\rho_{ik}(a_i) = \rho_{jk}(a_j).
\]
Now observe from the hypothesis of the problem that \[
\phi_i(a_i) =  \phi_k(\rho_{ik}(a_i)) = \phi_k(\rho_{jk}(a_j))  = \phi_j(a_j)
\]
and so our definition of $\phi$ is independent of the representatives that we choose for the various equivalence classes $[a]\in A$.

With this it is straightforward to show that $\phi$ is a homomorphism of $R$-modules. The action of $R$ on $A$ is given by $r[a] = [ra]$ and we have argued previously that this action is independent of the choice of representative for $[a]$. Then we have that \[
\phi(r[a]) = \phi([ra]) = \phi_i(ra_i) = r\phi_i(a_i) = r\phi([a_i]) = r\phi([a])
\]
where the representative $a_i$ is in $A_i$. One can also verify easily that this map respects the group structure and is therefore a homomorphism. 

Our final task is to argue uniqueness of $\phi$. To this end suppose we have some $R$-module homomorphism $\psi: A\to C$ so that each $\phi_i$ factors through $\psi$. Furthermore suppose that $\psi$ disagrees with $\phi$ on some $[a_i]$. This means that $\psi([a_i]) \neq \phi_i(a_i)$, but this is a direct contradiction to the fact that we require the diagram described previously to commute for all $i$. Hence the choice of $\phi$ is unique. 


\end{solution}\oneperpage



\begin{problem}\label{ex:10.3.26}
Carry out the analysis of the preceding exercise corresponding to the inverse limits to show that the inverse limit of $R$-modules is an $R$-module satisfying the appropriate universal property (cf. Exercise 10 of Section 7.6).
\end{problem}
\begin{solution}
Let $P$ be the inverse limit of the direct system of $A_i$. We know that $P$ is an abelian group, which is a subgroup of the direct product of all $A_i$.  As a result we need only show that $P$ is invariant under the natural action of $R$ (i.e. the action of $R$ on the direct product). Given some $\prod a_i$ in $P$, the action is given by $r\prod a_i = \prod ra_i$. Notice that this is still in $P$ since for any $i\le j$ we have \[
\mu_{ji}(ra_i) = r\mu_{ji}(a_i) = ra_j.
\]
Thus $P$ is invariant under the action and must be an $R$-module. 

We next prove the desired universal property. The propert is as follows: If $D$ is any $R$-module such that for each $i\in I$ there is an $R$-module homomorphism $\pi_i:D\to A_i$ with $\pi_i = \mu_{ji}\circ \pi_j$ for all $i\le j$, then there is a unique $R$-module homomorphism $\pi:D\to P$ such that $\mu_i\circ \pi = \pi_i$. I.e. there is a unique $\pi:D\to P$ so that the following diagram commutes for all $i$ (recall that $\mu_i$ is the natural projection from $P$ to $A_i$):\[
\begin{CD}
D@>\pi >> P\\
@VVV @VV\mu_i V\\
@>\pi_i>> A_i
\end{CD}
\]
(The CD package doesn't allow for diagonal arrows, sorry).
\end{solution}\oneperpage

We define a candidate for the map $\pi$ by $\pi(d) = \prod \pi_i(d)$. Notice that this certainly makes the diagram above commute, since \[
\mu_i(\pi(d)) = \mu_i(\prod \pi_i(d)) = \pi_i(d)
\]
for all $i$. It is also a homomorphism of $R$-modules since for any $r\in R$ we have \[
\pi(rd) = \prod \pi_i(rd) = \prod r\pi_i(d) = r\prod pi_i(d) = r\pi(d)
\]
and additivity can be verified quickly as well. For uniqueness consider a homomorphism $\nu :D\to P$ that does not agree with $\pi$ on some $d$. If $\nu(d) = \prod a_i$ then for some $a_i$ we have $a_i\neq \pi_i(d)$. But then the diagram from before clearly does not commute for this $i$. This proves uniqueness.

\begin{problem}\label{ex:10.3.27}
\emph{(Free modules over noncommutative rings need not have a unique rank)} Let $M$ be the $\Z$-module $\Z\times \Z\times \cdots$ of Exercise 24 and let $R$ be its endomorphism ring, $R = \End_\Z(M)$ (cf. Exercises 29 and 30 in Section 7.1). Define $\phi_1,\phi_2\in R$ by \begin{align*}
\phi_1(a_1,a_2,a_3,\ldots)&=(a_1,a_3,a_5,\ldots)\\
\phi_2(a_1,a_2,a_3,\ldots)&=(a_2,a_4,a_6,\ldots)
\end{align*}
\begin{enumerate}
\item[(a)] Prove that $\{\phi_1,\phi_2\}$ is a free basis of the left $R$-module $R$. [Define the maps $\psi_1$ and $\psi_2$ by $\psi_1(a_1,a_2,\ldots) =(a_1,0,a_2,0,\ldots)$ and $\psi_2(a_1,a_2,\ldots) =(0,a_1,0,a_2,\ldots)$. Verify that $\phi_i\psi_i = 1$, $\phi_1\psi_2 = 0 = \phi_2\psi_1$ and $\psi_1\phi_1+\psi_2\phi_2 = 1$. use these relations to prove that $\phi_1,\phi_2$ are independent and generate $R$ as a left $R$-module.]
\item[(b)] Use $(a)$ to prove that $R\cong R^2$ and deduce that $R\cong R^n$ for all $n\in \Z^+$.
\end{enumerate}
\end{problem}
\begin{solution}
(a)\\
Let $\phi_1, \phi_2, \psi_1$ and $\psi_2$ be as described in the problem statement. Notice that \begin{align*}
\phi_1(\psi_1(a_1,a_2,a_3,\ldots)) &= \phi_1(a_1,0,a_2,0,\ldots) = (a_1,a_2,a_3,\ldots)
\end{align*}
and \begin{align*}
\phi_2(\psi_2(a_1,a_2,a_3,\ldots)) &= \phi_1(0,a_1,0,a_2,0,\ldots) = (a_1,a_2,a_3,\ldots)
\end{align*}
and so $\phi_1\psi_1$ and $\phi_2\psi_2$ are both identity. Via similarly straightforward computations one can show that $\phi_1\psi_2$ and $\phi_2\psi_1$ are both the zero map. We will use these facts to show that $\phi_1$ and $\phi_2$ are $R$-linearly independent. Suppose for contradiction that there were some $\pi_1, \pi_2\in R$ at least one of which is nonzero so that \[
0 = \pi_1\phi_1 + \pi_2\phi_2.
\]
Multiplying on the right by $\psi_1+\psi_2$ we obtain\begin{align*}
0 &= (\pi_1\phi_1 + \pi_2\phi_2)(\psi_1+\psi_2)\\
&= \pi_1\phi_1\psi_1 + +\pi_1\phi_1\psi_2 + \pi_2\phi_2\psi_1 + \pi_2\phi_2\psi_2\\
& = \pi_1 + \pi_2.
\end{align*}
On the other hand we may multiply by $\psi_1-\psi_2$ and find that \begin{align*}
0 &= (\pi_1\phi_1 + \pi_2\phi_2)(\psi_1-\psi_2)\\
&= \pi_1\phi_1\psi_1 + +\pi_1\phi_1\psi_2 - \pi_2\phi_2\psi_1 - \pi_2\phi_2\psi_2\\
& = \pi_1 - \pi_2.
\end{align*}
Now taking the difference and sum of these two results we have $2\pi_1 = 0$ and $2\pi_2 = 0$. It is clear that composing a map with the doubling map yields the zero map if and only if the original map was zero, and hence $\pi_1 = \pi_2 = 0$. This proves that $\phi_1$ and $\phi_2$ form a free basis for a submodule of $R$. 

To see that the free module generated by $\phi_1$ and $\phi_2$ is all of $R$ we show the final relation: $\psi_1\phi_1+\psi_2\phi_2 = 1$. We can compute directly that \begin{align*}
(\psi_1\phi_1+\psi_2\phi_2)(a_1,a_2,a_3,\ldots) &= \psi_1(a_1,a_3,a_5,\ldots) + \psi_2(a_2,a_4,a_6,\ldots)\\
& = (a_1,0,a_3,\ldots) + (0,a_2,0,a_4,\ldots)\\
& = (a_1,a_2,a_3,a_4,\ldots).
\end{align*}
This shows that $\psi_1\phi_1+\psi_2\phi_2 = 1$. Since this is a left $R$-linear combination of $\phi_1$ and $\phi_2$ we conclude that the free module generated by $\phi_1$ and $\phi_2$ contains 1, and is all of $R$.\\\\
\noindent (b)\\
In part (a) we showed that $R$ is a free $R$-module over a basis of size 2. In particular, $R\cong R^2$. By induction we naturally have $R\cong R^n$. In fact, I believe that we could choose the maps which pick out the various $n$-th components of vectors in $M$ as a free basis for $R$ of size $n$, if we wanted to be explicit. 
\end{solution}\oneperpage



\end{section}









\begin{section}{Tensor Products of Modules}
Let $R$ be a ring with 1. 
\begin{problem}\label{ex:10.4.1}
Let $f:R\to S$ be a ring homomorphism from the ring $R$ to the ring $S$ with $f(1_R) = 1_S$. Verify the details that $sr = sf(r)$ defines a right $R$-action on $S$ under which $S$ is an $(S,R)$-bimodule. 
\end{problem}
\begin{solution}
We first show that this right action gives $S$ an $R$-module structure. We verify directly that\[
(s+s')r = (s+s')f(r) = sf(r) + s'f(r) = sr + s'r
\]
and \[
s(r+r') = sf(r+r') = s(f(r)+f(r')) = sf(r) + sf(r') = sr + sr'
\]
and \[
(sr)r' = sf(r)r' = sf(r)f(r') = sf(rr')  = s(rr')
\]
which proves that $S$ has the structure of a right $R$-module. We also need to verify that the left action of $S$ is compatible with the right action of $R$. However this is immediate from the associativity of multiplication in $S$:\[
(s's)r = (s's)f(r) = s'(sf(r)) = s'(sr) .
\]
This proves that $S$ has an $(S,R)$-bimodule structure.
\end{solution}\oneperpage



\begin{problem}\label{ex:10.4.2}
Show that the element ``$2\otimes 1$'' is $0$ in $\Z\otimes_\Z \Z/2\Z$ but is nonzero in $2\Z\otimes_\Z \Z/2\Z$. 
\end{problem}
\begin{solution}
This arises from the fact that $2$ is ``divisible by two'' in $\Z$ but not $2\Z$. In particular, in $\Z\otimes_\Z \Z/2\Z$ we have that \[
2\otimes 1  = (1\cdot 2)\otimes 1 = 1\otimes (2\cdot 1) = 1\otimes 2 = 1\otimes 0 = 0.
\]
To prove that $2\otimes 1 $ is nonzero in the second tensor product, we consider a group homomorphism $\phi:2\Z\otimes_\Z \Z/2\Z \to \Z/2\Z$ defined by \[
\phi(2k\otimes x) = kx
\]
where $k\in \Z$ and $x\in \Z/2\Z$. To see that this is a group homomorphism it suffices to show that it is bilinear in $k$ and $x$. Let $a,b,k,k'\in \Z$ and $x,x'\in \Z/2\Z$. Then we compute directly that\begin{align*}
\phi(a(2k) + b(2k'), x) &= \phi(2(ak+bk'),x)\\
& =(ak+bk')x\\
& = akx + bk'x\\
& = a\phi(2k,x) + b\phi(2k',x)
\end{align*}
and \begin{align*}
\phi(2k, ax+bx') & = k(ax+bx')\\
& = akx + bkx'\\
& = a\phi(2k,x) + b\phi(2k,x')
\end{align*}
and so $\phi$ is linear in each coordinate. We conclude that it is a valid homomorphism. We then notice that $\phi(2\otimes 1) = 1 \neq 0$ and hence $2\otimes 1$ is not in the kernel of $\phi$. This implies that $2\otimes 1$ is nonzero as desired.

\end{solution}\oneperpage



\begin{problem}\label{ex:10.4.3}
Show that $\C\otimes_\R\C$ and $\C\otimes _\C\C$ are both left $\R$-modules but are not isomorphic as $\R$-modules.
\end{problem}
\begin{solution}
For this problem we use the natural $\R$ module structure obtained by considering $\R$ as a subring of $\C$. First we establish a general result. Let $R$ be an integral domain contained in an integral domain $S$. Then $s\otimes s' \in S\otimes_R S$ is nonzero if and only if $s$ and $s'$ are both nonzero. This can be seen by defining a bilinear map $s\otimes s'\mapsto ss'$ and noticing that $s\otimes s'$ is not in the kernel exactly when $s$ and $s'$ are both nonzero. 

In the chapter we have seen that $R\otimes_R R \cong R$ for any ring $R$, and so clearly $\C\otimes_\C \C$ is a 1-dimensional vector space over $\C$. In particular, it is a 2-dimensional vector space over $\R$. We will show that $\C\otimes_\R\C$ is a 4-dimensional vector space over $\R$, and hence not isomorphic to $\C\otimes_\C\C$. Consider the simple tensors $1\otimes 1, 1\otimes i, i\otimes 1, $ and $i\otimes i$. These span $\C\otimes_\R\C$ since every simple tensor can be written in terms of them \[
(a+bi)\otimes (c+di) = ac(1\otimes 1) + ad(1\otimes i) + bc(i\otimes 1) + bd(i\otimes i).
\]
Then consider the map from $\C\times \C$ to $\R^4$ given by \[
 (a+bi, c+di) \mapsto (ac,ad,bc,bd).
\]
One can verify that this map is $\R$-bilinear, and hence induces an $R$-module homomorphism from $\C\otimes_\R\C$ to $\R^4$. Furthermore this map is clearly surjective (the image of the simple tensors described before yields the natural basis for $\R^4$). The inverse mapping from $\R^4$ can be constructed by sending the standard basis vectors to the simple tensors described before. Since these tensors generate $\C\otimes_\R\C$ we see that the inverse is also surjective. The composition of these maps is clearly the identity in both directions and hence they are isomorphisms. This proves that $\C\otimes_\R\C$ is 4-dimensional over $\R$.
\end{solution}\oneperpage



\begin{problem}\label{ex:10.4.4}
Show that $\Q\otimes_\Z\Q$ and $\Q\otimes_\Q\Q$ are isomorphic left $\Q$-modules. [Show they are both 1-dimensional vector spaces over $\Q$.]
\end{problem}
\begin{solution}Similar to the previous problem we know that $\Q\otimes_\Q\Q \cong \Q$. In particular, all simple tensors $a\otimes b$ can be written as $1\otimes ab$ and so all tensors are simple and of the form $1\otimes q$ for some $q\in \Q$. In particular $1\otimes 1$ is a basis for $\Q\otimes_\Q\Q$ as a $\Q$-vector space. 

We claim that $1\otimes 1$ is also a basis for $\Q\otimes _\Z \Q$. First we argue that everything in $\Q\otimes_\Z\Q$ can be written as $1\otimes q$ for $q\in \Q$. Given an arbitrary simple tensor $\frac{a}{b}\otimes \frac{c}{d}$ we have that \begin{align*}
\frac{a}{b}\otimes \frac{c}{d} & = \frac{a}{b}\otimes b\frac{c}{db}\\
& = \frac{ab}{b}\otimes \frac{c}{db}\\
& = a\otimes\frac{c}{db}\\
& = 1\otimes \frac{ac}{db}.
\end{align*}
Hence every simple tensor has the form $1\otimes q$ and it follows that every element of $\Q\otimes_\Z\Q$ can be written this way. Next we argue that $1\otimes q$ is nonzero exactly when $q$ is nonzero. To do this we simply define a map $1\otimes q\mapsto q$ from the tensor product to $\Q$. One can verify that this map is $\Z$-linear, and we observe easily that $1\otimes q$ is in the kernel only if $q=0$. Hence the only element of $\Q\otimes_\Z\Q$ that is zero is $1\otimes 0$ and the result follows.

\end{solution}\oneperpage



\begin{problem}\label{ex:10.4.5}
Let $A$ be a finite abelian group of order $n$ and let $p^k$ be the largest power of the prime $p$ dividing $n$. Prove that $\Z/p^k\Z\otimes_\Z A$ is isomorphic to the Sylow $p$-subgroup of $A$. 
\end{problem}
\begin{solution}Since $A$ is a finite abelian group we may write it as a direct sum of cyclic groups $A \cong C_1\oplus C_2\oplus\cdots \oplus C_n$ where each $C_i$ is a cyclic group of order $a_i$. But then we have by properties of the tensor product that \begin{align*}
\Z/p^k\Z\otimes_\Z A &\cong \Z/p^k\Z\otimes_\Z \left(\bigoplus_{i=1}^n C_i\right)\\
& \cong \bigoplus_{i=1}^n (\Z/p^k\Z\otimes_\Z C_i)\\
& \cong \bigoplus_{i=1}^n (\Z/p^k\Z\otimes_\Z \Z/a_i\Z)
\end{align*}
But $\Z/n\Z \otimes_\Z \Z/m\Z$ is isomorphic to $\Z/(m,n)\Z$ and so all terms in the direct sum above for which $a_i$ is not divisible by $p$ vanish. We conclude that \[
\Z/p^k\Z\otimes_\Z A\cong \bigoplus_{i=1}^n \Z/p^{\alpha_i}\Z
\]
where $\alpha_i$ is the largest power of $p$ dividing $a_i$ for all $i$. But this is just the direct sum of the Sylow $p$-subgroup for each $C_i$. It is fairly clear that this is the same as the Sylow $p$-subgroup for $A$ overall. 


\end{solution}\oneperpage



\begin{problem}\label{ex:10.4.6}
If $R$ is any integral domain with a quotient field $Q$, prove that $(Q/R)\otimes_R(Q/R) = 0$.
\end{problem}
\begin{solution}
We will show that all simple tensors are zero. Let $\frac{a}{b}\otimes \frac{c}{d}$ be an arbitrary element of the tensor product, where $a,b,c,d\in R$. Then we have that \begin{align*}
\frac{a}{b}\otimes \frac{c}{d} & =\frac{a}{b}\otimes b\frac{c}{db}\\
& =\frac{ab}{b}\otimes \frac{c}{db}\\
& =a\otimes \frac{c}{db}\\
& = 0\otimes \frac{c}{db}\\
& = 0.
\end{align*}
Hence all simple tensors vanish and we are done. 
\end{solution}\oneperpage



\begin{problem}\label{ex:10.4.7}
If $R$ is any integral domain with quotient field $Q$ and $N$ is a left $R$-module, prove that every element of the tensor product $Q\otimes_R N$ can be written as a simple tensor of the form $(1/d)\otimes n$ for some nonzero $d\in R$ and some $n\in N$. 
\end{problem}
\begin{solution}
Note that simple tensors clearly have this form since we can write $\frac{a}{d}\otimes n = \frac{1}{d}\otimes an$. Every element of $Q\otimes_R N$ is a finite sum of simple tensors, which may be written as \[
\sum_{i=1}^k \frac{1}{d_i}\otimes n_i.
\]
Letting $d$ be the product of all $d_i$ and defining $a_i = \prod_{j\neq i} d_i$ we have that \begin{align*}
\sum_{i=0}^k \frac{1}{d_i}\otimes n_i &= \sum_{i=1}^n \frac{a_i}{d} \otimes n_i\\
& = \sum_{i=1}^n \frac{1}{d}\otimes a_in_i\\
& = \frac{1}{d}\otimes\sum_{i=1}^n a_in_i.
\end{align*}
This proves the desired result. 
\end{solution}\oneperpage



\begin{problem}\label{ex:10.4.8}
Suppose $R$ is an integral domain with quotient field $Q$ and let $N$ be any $R$-module. Let $U = R^\times$ be the set of nonzero elements in $R$ and define $U^{-1}N$ to be the set of equivalence classes of ordered pairs of elements $(u,n)$ with $u\in U$ and $n\in N$ under the equivalence relation $(u,n)\sim (u',n)$ if and only if $u'n = un'$ in $N$. 
\begin{enumerate}
\item[(a)] Prove that $U^{-1}N$ is an abelian group under the addition defined by $\overline{(u_1,n_1)} + \overline{(u_2,n_2)} = \overline{(u_1u_2,u_2n_1+u_1n_2)}$. Prove that $r\overline{(u,n)} = \overline{(u,rn)}$ defines an action of $R$ on $U^{-1}N$ making it into an $R$-module. [This is an example of \emph{localization} considered in general in Section 4 of Chapter 15, cf. also Section 6 in Chapter 7.]
\item[(b)]  Show that the map from $Q\times N$ to $U^{-1}N$ defined by sending $(a/b,n)$ to $\overline{(b,an)}$ for $a\in R, b\in U,n\in N$, is an $R$-balanced map, so induces a homomorphism $f$ from $Q\otimes_RN$ to $U^{-1}N$. Show that the map $g$ from $U^{-1}N$ to $Q\otimes_RN$ defined by $g(\overline{(u,n)}) = (1/u)\otimes n$ is well defined and is an inverse homomorphism to $f$. Conclude that $Q\otimes _R N \cong U^{-1}N$ as $R$-modules. 
\item[(c)] Conclude from (b) that $(1/d)\otimes n$ is 0 in $Q\otimes_R N$ if and only if $rn=0$ for some nonzero $r\in R$. 
\item[(d)] If $A$ is an abelian group show that $\Q\otimes_Z A = 0$ if and only if $A$ is a torsion abelian group (i.e., every element of $A$ has finite order). 
\end{enumerate}
\end{problem}
\begin{solution}TODO

\end{solution}\oneperpage



\begin{problem}\label{ex:10.4.9}
Suppose $R$ is an integral domain with the quotient field $Q$ and let $N$ be any $R$-module. Let $Q\otimes_RN$ be the module obtained from $N$ by extension of scalars from $R$ to $Q$. Prove that the kernel of the $R$-module homomorphism $\iota:N\to Q\otimes_RN$ is the torsion submodule of $N$ (cf. Exercise 8 in Section 1).  [Use the previous exercise.] 
\end{problem}
\begin{solution}TODO

\end{solution}\oneperpage



\begin{problem}\label{ex:10.4.10}
Suppose $R$ is commutative and $N\cong R^n$ is a free $R$-module of rank $n$ with $R$-module basis $e_1,\ldots,e_n$. \begin{enumerate}
\item[(a)] For any nonzero $R$-module $M$ show that every element of $M\otimes N$ can be written uniquely in the form $\sum_{i=1}^nm_i\otimes e_1$ where $m_i\in M$ . Deduce that if $\sum_{i=1}^n m_i\otimes e_i = 0$ in $M\otimes N$ then $m_i=0$ for $i=1,\ldots,n$. 
\item[(b)] Show that if $\sum m_i\otimes n_i = 0$ in $M\otimes N$ where $n_i$ are merely assumed to be $R$-linearly independent then it is not necessarily true that all $m_i$ are 0. [Consider $R=\Z,n=1,M=\Z/2\Z$ and the element $1\otimes 2$.]
\end{enumerate}
\end{problem}
\begin{solution}TODO

\end{solution}\oneperpage



\begin{problem}\label{ex:10.4.11}
Let $\{e_1,e_2\}$ be a basis of $V=\R^2$. Show that the element $e_1\otimes e_2 + e_2\otimes e_1$ in $V\otimes_\R V$ cannot be written as a simple tensor $v\otimes w$ for any $v,w\in \R^2$. 
\end{problem}
\begin{solution}TODO

\end{solution}\oneperpage



\begin{problem}\label{ex:10.4.12}
Let $V$ be a vector space over the field $F$ and let $v,v'$ be nonzero elements of $V$. Prove that $v\otimes v' = v'\otimes v$ in $V\otimes_FV$ if and only if $v=av'$ for some $a\in F$.
\end{problem}
\begin{solution}TODO

\end{solution}\oneperpage



\begin{problem}\label{ex:10.4.13}
Prove that the usual dot product of vectors defined by letting $(a_1,\ldots,a_n)\cdots(b_1,\ldots,b_n)$ be $a_1b_1+\cdots+a_nb_n$ is a bilinear map from $R^n\times R^n$ to $\R$. 
\end{problem}
\begin{solution}TODO

\end{solution}\oneperpage



\begin{problem}\label{ex:10.4.14}
Let $I$ be an arbitrary nonempty index set and for each $i\in I$ let $N_i$ be a left $R$-modules. Let $M$ be a right $R$-module. Prove the group isomorphism: $M\otimes (\oplus_{i\in I} N_i)\cong \oplus_{i\in I}(M\otimes N_i)$, where the direct sum of an arbitrary collection of modules is defined in Exercise 20, Section 3. [Use the same argument as for the direct sum of two modules, taking care to note where the direct \emph{sum} hypothesis is needed --- cf. the next exercise.]
\end{problem}
\begin{solution}TODO

\end{solution}\oneperpage



\begin{problem}\label{ex:10.4.15}
Show that tensor products do not commute with direct products in general. [Consider the extension of scalars from $\Z$ to $\Q$ of the direct product of the modules $M_i = \Z/2^i\Z$, $i=1,2,\ldots$]. 
\end{problem}
\begin{solution}TODO

\end{solution}\oneperpage



\begin{problem}\label{ex:10.4.16}
Suppose $R$ is commutative and let $I$ and $J$ be ideals of $R$, so $R/I$ and $R/J$ are naturally $R$-modules. \begin{enumerate}
\item[(a)] Prove that every element of $R/I\otimes _R R/J$ can be written as a simple tensor of the form $(1\text{ mod } I)\otimes (r\text{ mod } J)$. 
\item[(b)] Prove that there is an $R$-module isomorphism $R/I\otimes_R R/J\cong R/(I+J)$ mapping $(r\text{ mod } I)\otimes (r'\text{ mod } J)$ to $rr'$ mod $(I+J)$. 
\end{enumerate}
\end{problem}
\begin{solution}TODO

\end{solution}\oneperpage



\begin{problem}\label{ex:10.4.17}
Let $I=(2,x)$ be the ideal generated by $2$ and $x$ in the ring $R=\Z[x]$. The ring $\Z/2\Z = R/I$ is naturally an $R$-module annihilated by both $2$ and $x$. \begin{enumerate}
\item[(a)] Show that the map $\phi:I\times I\to \Z/2\Z$ defined by \[
\phi(a_0+a_1x+\cdots a_nx^n,b_0+b_1x+\cdots b_mx^m) = \frac{a_0}{2}b_1\text{ mod } 2 
\]
is $R$-bilinear.
\item[(b)] Show that there is an $R$-module homomorphism from $I\otimes_R I\to\Z/2\Z$ mapping $p(x)\otimes q(x)$ to $\frac{p(0)}{2}q'(0)$ where $q'$ denotes the usual polynomial derivative of $q$. 
\item[(c)] Show that $2\otimes x \neq x\otimes 2$ in $I \otimes_RI$. 
\end{enumerate}
\end{problem}
\begin{solution}TODO

\end{solution}\oneperpage



\begin{problem}\label{ex:10.4.18}
Suppose $I$ is a principal ideal in the integral domain $R$. Prove that the $R$-modules $I\otimes_R I$ has no nonzero torsion elements (i.e. $rm=0$ with $0\neq r \in R$ and $m\in I\otimes _RI$ implies $m=0$). 
\end{problem}
\begin{solution}TODO

\end{solution}\oneperpage



\begin{problem}\label{ex:10.4.19}
Let $I=(2,x)$ be the ideal generated by $2$ and $x$ in the ring $R-\Z[x]$ as in Exercise 17. Show that the nonzero element $2\otimes x-x\otimes 2$ in $I\otimes_R I$ is a torsion element. Show in fact that $2\otimes x - x\otimes 2$ is annihilated by both $2$ and $x$ and that the submodule of $I\otimes_R I$ generated by $2\otimes x - x\otimes 2$ is isomorphic to $R/I$. 
\end{problem}
\begin{solution}TODO

\end{solution}\oneperpage



\begin{problem}\label{ex:10.4.20}
Let $I=(2,x)$ be the ideal generated by $2$ and $x$ in the ring $R=\Z[x]$. Show that the element $2\otimes 2 + x\otimes x$ in $I\otimes_RI$ is not a simple tensor, i.e. cannot be written as $a\otimes b$ for some $a,b\in I$. 
\end{problem}
\begin{solution}TODO

\end{solution}\oneperpage



\begin{problem}\label{ex:10.4.21}
Suppose $R$ is commutative and let $I$ and $J$ be ideals of $R$. 
\begin{enumerate}
\item[(a)] Show that there is a surjective $R$-module homomorphism from $I\otimes_R J$ to the product ideal $IJ$ mapping $I\otimes J$ to the element $ij$.
\item[(b)] Give an example to show that the map in (a) need not be injective (cf. Exercise 17). 
\end{enumerate}
\end{problem}
\begin{solution}TODO

\end{solution}\oneperpage



\begin{problem}\label{ex:10.4.22}
Suppose that $m$ is a left and a right $R$-module such that $rm=mr$ for all $r\in R$ and $m\in M$. Show that the elements $_1r_2$ and $r_2r_1$ act the same on $M$ for every $r_1,r_2\in R$. (This explains why the assumption that $R$ is commutative in the definition of an $R$-algebra is a fairly natural one.) 
\end{problem}
\begin{solution}TODO

\end{solution}\oneperpage



\begin{problem}\label{ex:10.4.23}
Verify the details that the multiplication in Proposition 19 makes $A\otimes_R B$ into an $R$-algebra.
\end{problem}
\begin{solution}TODO

\end{solution}\oneperpage



\begin{problem}\label{ex:10.4.24}
Prove that the extension of scalars from $\Z$ to the Gaussian integers $\Z[i]$ of the ring $\R$ is isomorphic to $\C$ as a ring:$\Z[i]\otimes_\Z\R\cong\C$ as rings. 
\end{problem}
\begin{solution}TODO

\end{solution}\oneperpage



\begin{problem}\label{ex:10.4.25}
Let $R$ be a subring of the commutative ring $S$ and let $x$ be an indeterminate over $S$. Prove that $S[x]$ and $S\otimes_RR[x]$ are isomorphic as $S$-algebras.
\end{problem}
\begin{solution}TODO

\end{solution}\oneperpage



\begin{problem}\label{ex:10.4.26}
Let $S$ be a commutative ring containing $R$ (with $1_s = 1_R$) and let $x_1,\ldots,x_n$ be independent indeterminates over the ring $S$. Show that for every ideal $I$ in the polynomial ring $R[x_1,\ldots,x_n]$ that $S\otimes_R(R[x_1,\ldots,x_n]/I)\cong S[x_1,\ldots,x_n]/IS[x_1,\ldots,x_n]$. 
\end{problem}
\begin{solution}TODO

\end{solution}\oneperpage



\begin{problem}\label{ex:10.4.27}
The next exercise shows the ring $C\otimes _R\C$ introduced at the end of this section is isomorphic to $\C\times \C$. One may also prove this via Exercise 26 and Proposition 16 in Section 9.5, since $\C\cong \R[x]/(x^2+1)$. The ring $C\times \C$ is also discussed in Exercise 23 of Section 1.\begin{enumerate}
\item[(a)] Write down a formula for the multiplication of two elements $a\cdot 1 + b \cdot e_2+c\cdot e_3+d\cdot e_4$ and $a'\cdot 1 + b' \cdot e_2+c'\cdot e_3+d.\cdot e_4$ in the example $A=\C\otimes\R\C$ following Proposition 21 (where $1=1\otimes 1$ is the identity of $A$). 
\item[(b)] Let $\epsilon_1=\frac{1}{2}(1\otimes 1+i\otimes i)$ and let $\epsilon_2=\frac{1}{2}(1\otimes 1-i\otimes i)$. Show that $\epsilon_1\epsilon_2 = 0$, $\epsilon_1+\epsilon_2=1$ and $\epsilon_j^2=\epsilon_j$ for $j=1,2$ ($\epsilon_1$ and $\epsilon_2$ are called \emph{orthogonal idempotents} in $A$). Deduce that $A$ is isomorphic as a ring to the direct product of two principal ideals:$ A\cong A\epsilon_1\times A\epsilon_2$ (cf. Exercise 1, Section 7.6). 
\item[(c)] Prove that the map $\phi:\C\times \C \to \C\times\C$ by $\phi(z_1,z_2) = (z_1z_2,z_1\overline{z_2})$, where $\overline{z_2}$ denotes the complex conjugate of $z_2$, is an $\R$-bilinear map.
\item[(d)] Let $\Phi$ be the $\R$-module homomorphism from $A$ to $\C\times \C$ obtained from $\phi$ in (c). Show that $\Phi(\epsilon_1) = (0,1)$ and $\Phi(\epsilon_2)=(1,0)$. Show also that $\Phi$ is $\C$-linear, where the action of $\C$ is on the left tensor factor in $A$ and on both factors in $\C\times\C$. Deduce that $\Phi$ is surjective. Show that $\Phi$ is a $\C$-algebra isomorphism. 
 \end{enumerate} 
\end{problem}
\begin{solution}TODO

\end{solution}\oneperpage


\end{section}









\begin{section}{Exact Sequences---Projective, Injective, and Flat Modules}

\begin{problem}\label{ex:10.5.1}
Suppose that \[\begin{CD}
A @>\psi>> B @>\phi>>C\\
@V\alpha VV @V\beta VV @V\gamma VV\\
A' @>\psi'>> B' @>\phi'>>C'
\end{CD}\]
is a commutative diagram of groups and that the rows are exact. Prove that \begin{enumerate}
\item[(a)] If $\phi$ and $\alpha$ are surjective, and $\beta$ is injective then $\gamma$ is injective. [If $c\in\ker \gamma$, show there is a $b\in B$ with $\phi(b)=c.$ Show that $\phi'(\beta(b)) = 0$ and deduce that $\beta(b)=\phi'(a')$ for some $a'\in A'$. Show that there is an $a\in A$ with $\alpha(a) = a'$ and that $\beta(\psi(a)) = \beta(b).$ Conclude that $b=\psi(a)$ and hence $c=\phi(b) = 0$.]
\item[(b)] If $\phi',\alpha$ and $\gamma$ are injective, then $\beta$ is injective.
\item[(c)] If $\phi,\alpha$ and $\gamma$ are surjective, then $\beta$ is surjective.
\item[(d)] If $\beta$ is injective, $\alpha$ and $\phi$ are surjective, then $\gamma$ is injective.
\item[(e)] If $\beta$ is surjective, $\gamma$ and $\psi'$ are injective, then $\alpha$ is surjective.
\end{enumerate}
\end{problem}
\begin{solution}TODO

\end{solution}\oneperpage



\begin{problem}\label{ex:10.5.2}
Suppose that \[\begin{CD}
A @>>> B @>>>C@>>>D\\
@V\alpha VV @V\beta VV @V\gamma VV@V\delta  VV\\
A' @>'>> B' @>>>C'@>>>D'
\end{CD}\]
is a commutative diagram of groups and that the rows are exact. Prove that \begin{enumerate}
\item[(a)] If $\alpha$ is surjective and $\beta,\delta$ are injective, then $\gamma$ is injective.
\item[(b)] If $\delta$ is injective, and $\alpha,\gamma$ are surjective, then $\beta$ is surjective.
\end{enumerate}
\end{problem}
\begin{solution}TODO

\end{solution}\oneperpage



\begin{problem}\label{ex:10.5.3}
Let $P_1$ and $P_2$ be $R$-modules. Prove that $P_1\oplus P_2$ is a projective $R$-module if and only if both $P_1$ and $P_2$ are projective.
\end{problem}
\begin{solution}TODO

\end{solution}\oneperpage



\begin{problem}\label{ex:10.5.4}
Let $Q_1$ and $Q_2$ be $R$-modules. Prove that $Q_1\oplus Q_2$ is an injective $R$-modules if and only if both $Q_1$ and $Q_2$ are injective.
\end{problem}
\begin{solution}TODO

\end{solution}\oneperpage



\begin{problem}\label{ex:10.5.5}
Let $A_1$ and $A_2$ be $R$-modules. Prove that $A_1\oplus A_2$is a flat $R$-modules if and only if both $A_1$ and $A_2$ are flat. More generally, prove that an arbitrary direct sum $\sum A_i$ of $R$-modules is flat if and only if each $A_i$ is flat. [Use the fact that tensor product sommutes with arbitrary direct sums.]
\end{problem}
\begin{solution}TODO

\end{solution}\oneperpage



\begin{problem}\label{ex:10.5.6}
Prove that the following are equivalent for a ring $R$:\begin{itemize}
\item[(i)] Every $R$-module is projective.
\item[(ii)] Every $R$-module is injective.
\end{itemize}
\end{problem}
\begin{solution}TODO

\end{solution}\oneperpage



\begin{problem}\label{ex:10.5.7}
Let $A$ be a nonzero finite abelian group.\begin{enumerate}
\item[(a)] Prove that $A$ is not a projective $\Z$-module.
\item[(b)] Prove that $A$ is not an injective $\Z$-module.
\end{enumerate}
\end{problem}
\begin{solution}TODO

\end{solution}\oneperpage



\begin{problem}\label{ex:10.5.8}
Let $Q$ be a nonzero divisible $\Z$-module. Prove that $Q$ is not a projective $\Z$-module. Deduce that the rational numbers $\Q$ is not a projective $\Z$-module. [Show first that if $F$ is any free module then $\cap_{n=1}^\infty nF = 0$ (use a basis of $F$ to prove this). Now suppose to the contrary that $Q$ is projective and derive a contradiction from Proposition 30(4).]
\end{problem}
\begin{solution}TODO

\end{solution}\oneperpage



\begin{problem}\label{ex:10.5.9}
Assume $R$ is commutative with $1$. \begin{enumerate}
\item[(a)] Prove that the tensor product of two free $R$-modules is free. [Use the fact that tensor products commute with arbitrary direct sums.]
\item[(b)] Use (a) to prove that the tensor product of two projective $R$-modules is projective.
\end{enumerate}
\end{problem}
\begin{solution}TODO

\end{solution}\oneperpage



\begin{problem}\label{ex:10.5.10}
Let $R$ and $W$ be rings with $1$ and let $M$ and $N$ be left $R$-modules. Assume also that $M$ is an $(R,S)$-bimodule.
\begin{enumerate}
\item[(a)] For $s\in S$ and for $\phi\in\Hom_R(M,N)$ define $(s\phi):M\to N$ by $(s\phi)(m) = \phi(ms)$. Prove that $s\phi$ is a homomorphism of left $R$-modules, and that this action of $S$ on $\Hom_R(M,N)$ makes it into a \emph{left} $S$-module.
\item[(b)] Let $S=R$ and let $M=R$ (considered as an $(R,R)$-bimodule by left and right ring multiplication on itself). For each $n\in N$ define $\phi_n:R\to N$ by $\phi_n(r) = rn$, i.e. $\phi_n$ is the unique $R$-module homomorphism mapping $1_R$ to $n$. Show that $\phi_n\in \Hom_R(R,N)$. Use part (a) to show that the map $n\mapsto \phi_n$ is an isomorphism of left $R$-modules:$N\cong\Hom_R(R,N)$. 
\item[(c)] Deduce that if $N$ is a free (respective, projective, injective, flat) left $R$-module, then $\Hom_R(R,N)$ is also a free (respective, projective, injective, flat) left $R$-module.
\end{enumerate}
\end{problem}
\begin{solution}TODO

\end{solution}\oneperpage



\begin{problem}\label{ex:10.5.11}
Let $R$ and $W$ be rings with $1$ and let $M$ and $N$ be left $R$-modules. Assume also that $M$ is an $(R,S)$-bimodule.
\begin{enumerate}
\item[(a)] For $s\in S$ and for $\phi\in\Hom_R(M,N)$ define $(\phi s):M\to N$ by $(\phi s)(m) = \phi(m)s$. Prove that $s\phi$ is a homomorphism of left $R$-modules, and that this action of $S$ on $\Hom_R(M,N)$ makes it into a \emph{right} $S$-module. Deduce that $\Hom_R(M,R)$ is a right $R$-module, for any $R$-module $M$ --- called the \emph{dual module} to $M$.
\item[(b)] Let $N=R$ be considered as an $(R,R)$ bimodule as usual. Under the action defined in part (a) show that the map $r\mapsto \phi_r$ is an isomorphism of right $R$-modules: $\Hom_R(R,R)\cong R$, where $\phi_r$ is the homomorphism that maps $1_R$ to $r$. Deduce that if $M$ is a finitely generated free left $R$-module, then $\Hom_R(M,R)$ is a free right $R$-module of the same rank. (cf. also Exercise 13).
\item[(c)] Show that if $M$ is a finitely generated projective $R$-module then its dual module $\Hom_R(M,R)$ is also projective.
\end{enumerate}
\end{problem}
\begin{solution}TODO

\end{solution}\oneperpage



\begin{problem}\label{ex:10.5.12}
Let $A$ be an $R$-module, let $I$ be any nonempty index set and for each $i\in I$ let $B_i$ be an $R$-module. Prove the following isomorphisms of abelian groups; when $R$ is commutative prove also that these are $R$-module isomorphisms. (Arbitrary direct sums and direct products of modules are introduced in Exercise 20 of Section 3.)
\begin{enumerate}
\item[(a)] $\Hom_R(\bigoplus_{i\in I}B_i,A)\cong\prod_{i\in I}\Hom_R(B_i,A)$
\item[(b)] $\Hom_R(A,\prod_{i\in I}B_i)\cong\prod_{i\in I}\Hom_R(A,B_i).$
\end{enumerate}
\end{problem}
\begin{solution}TODO

\end{solution}\oneperpage



\begin{problem}\label{ex:10.5.13}
\begin{enumerate}
\item[(a)] Show that the dual of the free $\Z$-module with countable basis is not free. [Use the preceding exercise and Exercise 24, Section 3.] (See also Exercise 5 in Section 11.3.)
\item[(b)] Show that the dual of the free $\Z$-module with countable basis is not projective. [You may use the fact that any submodule of a free $\Z$-module is free.]
\end{enumerate}
\end{problem}
\begin{solution}TODO

\end{solution}\oneperpage



\begin{problem}\label{ex:10.5.14}
Let $\begin{CD}0@>>>L@>\psi>>M@>\phi>>N@>>>0\end{CD}$ be a sequence of $R$-modules.
\begin{enumerate}
\item[(a)] Prove that the associated sequence \[
\begin{CD}0@>>>\Hom_R(D,L)@>\psi'>>\Hom_R(D,M)@>\phi'>>\Hom_R(D,N)@>>>0\end{CD}
\]
is a short exact sequence of abelian groups for all $R$-modules $D$ if and only if the original sequence is a split short exact sequence. [To show the sequence splits, take $D=N$ and show the lift of the identity map in $\Hom_R(N,N)$ to $\Hom_R(N,M)$ is a splitting homomorphism for $\phi$.]
\item[(b)] Prove that the associated sequence \[
\begin{CD}0@>>>\Hom_R(N,D)@>\phi'>>\Hom_R(M,D)@>\psi'>>\Hom_R(L,D)@>>>0\end{CD}
\]
is a short exact sequence of abelian groups for all $R$-modules $D$ if and only if the original sequence is a split short exact sequence. 
\end{enumerate}
\end{problem}
\begin{solution}TODO

\end{solution}\oneperpage



\begin{problem}\label{ex:10.5.15}
Let $M$ be a left $\Z$-module and let $R$ be a ring with 1. 
\begin{enumerate}
\item[(a)] Show that $\Hom_\Z(R,M)$ is a left $R$-module under the action $(r\phi)(r') = \phi(r'r)$ (see Exercise 10). 
\item[(b)] Suppose that $\begin{CD}0@>>>A@>\psi>>B\end{CD}$ is an exact sequence of $R$-modules. Prove that if every $\Z$-module homomorphism $f$ from $A$ to $M$ lifts to a $\Z$-module homomorphism $F$ from $B$ to $M$ with $f=F\circ\psi$, then every $R$-module homomorphism $f'$ from $A$ to $\Hom_Z(R,M)$ lifts to an $R$-module homomorphism $F'$ from $B$ to $\Hom_\Z(R,M)$ with $f' = F'\circ \psi$. [Given $f'$, show that $f(a) = f'(a)(1_R)$ defines a $\Z$-module homomorphism of $A$ to $M$. If $F$ is the associated lift of $f$ to $B$, show that $F'(b)(r) = F(rb)$ defines an $R$-modules homomorphism from $B$ to $\Hom_Z(R,M)$ that lifts $f'$.]
\item[(c)] Prove that if $Q$ is an injective $\Z$-module then $\Hom_\Z(R,Q)$ is an injective $R$-module. 
\end{enumerate}
\end{problem}
\begin{solution}TODO

\end{solution}\oneperpage



\begin{problem}\label{ex:10.5.16}
This exercise proves Theorem 38 that every left $R$-module $M$ is contained in an injective left $R$-module.
\begin{enumerate}
\item[(a)] Show that $M$ is contained in an injective $\Z$-module $Q$. [$M$ is a $\Z$-module---use Corollary 37.]
\item[(b)] Show that $\Hom_R(R,M)\subseteq\Hom_\Z(R,M)\subseteq \Hom_\Z(R,Q)$.
\item[(c)]  Use the $R$-module isomorphism $M\cong\Hom_R(R,M)$ (Exercise 10) and the previous exercise to conclude that $M$ is contained in an injective $R$-module.
\end{enumerate}
\end{problem}
\begin{solution}TODO

\end{solution}\oneperpage



\begin{problem}\label{ex:10.5.17}
This exercise completes the proof of Proposition 34. Suppose that $Q$ is an $R$-module with the property that every short exact sequence $\begin{CD}0@>>>Q@>>>M_1@>>>N@>>>0\end{CD}$ splits and suppose that the sequence $0@>>>L@>\psi>>M$ is exact. Prove that every $R$-module homomorphism $f$ from $L$ to $Q$ can be lifted to an $R$-module homomorphism $F$ from $M$ to $Q$ with $f=F\circ\psi$. [By the previous exercise, $Q$ is contained in an injective $R$-module. Use the splitting property together with Exercise 4 (noting that Exercise 4 can be proved using (2) in Proposition 34 as the definition of an injective module).] 
\end{problem}
\begin{solution}TODO

\end{solution}\oneperpage



\begin{problem}\label{ex:10.5.18}
Prove that the injective hull of the $\Z$-module $\Z$ is $\Q$ [Let $H$ be the injective hull of $\Z$ and argue that $\Q$ contains an isomorphic copy of $H$. Use the divisibility of $H$ to show that $1/n\in H$ for all nonzero integers $n$, and deduce that $H=\Q$.]
\end{problem}
\begin{solution}TODO

\end{solution}\oneperpage



\begin{problem}\label{ex:10.5.19}
If $F$ is a field, prove that the injective hull of $F$ is $F$.
\end{problem}
\begin{solution}TODO

\end{solution}\oneperpage



\begin{problem}\label{ex:10.5.20}
Prove that the polynomial ring $R[x]$ with indeterminate $x$ over the commutative ring $R$ is a flat $R$-module.
\end{problem}
\begin{solution}TODO

\end{solution}\oneperpage



\begin{problem}\label{ex:10.5.21}
Let $R$ and $S$ be rings with 1 and suppose $M$ is a right $R$-module, and $N$ is an $(R,S)$-bimodule. If $M$ is flat over $R$ and $N$ is flat as an $S$-module prove that $M\otimes_R N$ is flat as a right $S$-module. 
\end{problem}
\begin{solution}TODO

\end{solution}\oneperpage



\begin{problem}\label{ex:10.5.22}
Suppose that $R$ is a commutative ring and that $M$ and $N$ are flat $R$-modules. Prove that $M\otimes_R N$ is a flat $R$-module. [Use the previous exercise.]
\end{problem}
\begin{solution}TODO

\end{solution}\oneperpage



\begin{problem}\label{ex:10.5.23}
Prove that the (right) module $M\otimes_R S$ obtained by changing the base from the ring $R$ to the ring $S$ (by some homomorphism $f:R\to S$ with $f(1_R) = 1_S$ cf. Example 6 following Corollary 12 in Section 4) of the flat (right) $R$-module $M$ is a flat $S$-module.
\end{problem}
\begin{solution}TODO

\end{solution}\oneperpage



\begin{problem}\label{ex:10.5.24}
Prove that $A$ is a flat $R$-module if and only if for any left $R$-modules $L$ and $M$ where $L$ is \emph{finitely egenerated}, then $\psi:L\to M$ is injective implies that laso $1\otimes \psi:A\otimes_RL\to A\otimes_RM$ is injective. [Use the techniques if the proof of corollary 42.]
\end{problem}
\begin{solution}TODO

\end{solution}\oneperpage



\begin{problem}\label{ex:10.5.25}
\emph(A Flatness Criterion) Parts (a)-(c) of this exercise prove that $A$ is a flat $R$-module if and only if for every finitely generated ideal $I$ of $R$, the map from $A\otimes_RI\to A\otimes_RR\cong A$ induced by the inclusion $I\subseteq R$ is again injective (or equivalently, $A\otimes_RI\cong AI\subseteq A$). \begin{enumerate}
\item[(a)] Prove that if $A$ is flat then $A\otimes_RI\to A\otimes_RR$ is injective.
\item[(b)] If $A\otimes_RI\to A\otimes_RR$ is injective for every finitely generated ideal $I$, prove that $A\otimes_RI\to A\otimes_RR$ is injective for every ideal $I$. Show that if $K$ is any submodule of a finitely generated free module $F$ then $A\otimes_RK\to A\otimes_RF$ is injective. Show that the same is true for any free module $F$. [Cf. the proof of Corollary 42.]
\item[(c)]Under the assumption in (b), suppose $L$ and $M$ are $R$-modules and $\begin{CD}L@>\psi>>M\end{CD}$ is injective. Prove that $\begin{CD}A\otimes_RL@>1\otimes\psi>>A\otimes_RM\end{CD}$ is injective and conclude that $A$ is flat. [Write $M$ as a quotient of the free module $F$, giving a short exact sequence\[
\begin{CD}0@>>>K@>>>F@>f>>M@>>>0.\end{CD}
\]
Show that if $J=f^{-1}(\psi(L))$ and $\iota:J\to F$ is the natural injection, then the diagram\[
\begin{CD}
0@>>>K@>>>J@>>>L@>>>0\\
@.@VidVV@V\iota VV @V\psi VV\\
0@>>>K@>>>F@>>>M@>>>0
\end{CD}
\]
is commutative with exact rows. Show that the induced diagram\[
\begin{CD}
A\otimes_R K@>>>A\otimes_R J@>>>A\otimes_R L@>>>0\\
@VidVV@V1\otimes \iota VV @V1\otimes\psi VV\\
A\otimes_R K@>>>A\otimes_R F@>>>A\otimes_R M@>>>0
\end{CD}
\]
is commutative with exact rows. Use (b) to show that $1\otimes \iota$ is injective, then use Exercise 1 to conclude that $1\otimes \psi $ is injective.]
\item[(d)]  \emph(A Flatness Criterion for quotients) Suppose $A=F/K$ where $F$ is flat (e.g., if $F$ is free) and $K$ is an $R$-submodule of $F$. Prove that $A$ is flat if and only if $FI\cap K = KI$ for every finitely generated ideal $I$ of $R$. [Use (a) to prove $F\otimes_R I\cong FI$ and observe the image of $K\otimes_RI$ is $KI$; tensor the exact sequence $0\to K\to F\to A\to0$ with $I$ to prove that $A\otimes_RI\cong FI/KI$, and apply the flatness criterion.]
\end{enumerate}
\end{problem}
\begin{solution}TODO

\end{solution}\oneperpage



\begin{problem}\label{ex:10.5.26}
Suppose $R$ is a PID. This exercise proves that $A$ is a flat $R$-module if and only if $A$ is a torsion free $R$-module (i.e., if $a\in A$ is nonzero and $r\in R$, then $ra=0$ implies $r=0$). \begin{enumerate}
\item[(a)] Suppose that $A$ is flat and for fixed $r\in R$ consider the map $\psi_r:R\to R$ defined by multiplication by $r$: $\psi_r(x)=rx$. If $r$ is nonzero show that $\psi_r$ is an injection. Conclude from the flatness of $A$ that the map from $A$ to $A$ defined by mapping $a$ to $ra$ is injective and that $A$ is torsion free.
\item[(b)] Suppose that $A$ is torsion free. If $I$ is a nonzero ideal of $R$, then $I = rR$ for some nonzero $r\in R$. Show that the map $\psi_r$ in (a) induces an isomorphism $R\cong I$ of $R$-modules and that the composite $\begin{CD}R@>\psi>>I@>\iota>>R\end{CD}$ of $\psi_r$ with the inclusion $\iota:I\subseteq R$ is multiplication by $r$. Prove that the composite $\begin{CD} A\otimes_R R@>1\otimes\psi_r>>A\otimes_RI@>1\otimes\iota>>A\otimes_RR\end{CD}$ corresponds to the map $a\mapsto ra$ under the identification $A\otimes_RR = A$ and that this composite is injective since $A$ is torsion free. Show that $1\otimes \psi_r$ is an isomorphism and deduce that $1\otimes i$ is injective. Use the previous exercise to conclude that $A$ is flat.
\end{enumerate}
\end{problem}
\begin{solution}TODO

\end{solution}\oneperpage



\begin{problem}\label{ex:10.5.27}
Let $M,A$ and $B$ be $R$-modules.\begin{enumerate}
\item[(a)] Suppose $f:A\to M$ and $g:B\to M$ are $R$-module homomorphisms. Prove that $X=\{(a,b)\mid a\in A,b\in B\text{ with } f(a) = g(b)\}$ is an $R$-submodule of the direct sum $A\oplus B$ (called the \emph{pullback} or \emph{fiber product} of $f$ and $g$) and that there is a commutative diagram \[
\begin{CD}
X@>\pi_2>>B\\
@V\pi_1VV @VgVV\\
A@>f>>M
\end{CD}
\]
where $\pi_1$ and $\pi_2$ are the natural projections onto the first and second components.

\item[(b)] Suppose $f':M\to A$ and $g':M\to B$ are $R$-module homomorphisms. Prove that the quotient $Y$ of $A\oplus B$ by $ \{(f'(m), -g'(m))\mid m\in M\}$ is an $R$-module (called the \emph{pushout} or \emph{fiber sum} of $f'$ and $g'$) and that there is a commutative diagram \[
\begin{CD}
M@>g'>>B\\
@Vf'VV @V\pi_2'VV\\
A@>\pi_1'>>X
\end{CD}
\]
where $\phi_1'$ and $\phi_2'$ are the natural maps to the quotient induced by the maps into the first and second components.
\end{enumerate}
\end{problem}
\begin{solution}TODO

\end{solution}\oneperpage



\begin{problem}\label{ex:10.5.28}
\begin{enumerate}
\item[(a)]\emph{(Schanuel's Lemma)} If $\begin{CD}0@>>>K@>>>P@>\phi>>M@>>>0\end{CD}$ and $\begin{CD}0@>>>K'@>>>P'@>\phi'>>M@>>>0\end{CD}$ are exact sequences of $R$-modules where $P$ and $p'$ are projective, prove that $P\oplus K'\cong P'\oplus K$ as $R$-modules. [Show that there is an exact sequence $\begin{CD}0@>>>\ker \phi @>>>X@>\pi>>P@>>>0\end{CD}$ with $\ker\pi \cong K'$, where $X$ is the fiber product of $\phi$ and $\phi'$ as in the previous exercise. Deduce that $X\cong P\oplus K'$. Show similarly that $X\cong P'\oplus K$.]

\item[(b)] If $\begin{CD}0@>>>M@>>>Q@>\psi>>L@>>>0\end{CD}$ and $\begin{CD}0@>>>M@>>>Q'@>\psi'>>L'@>>>0\end{CD}$ are exact sequences of $R$-modules where $Q$ and $Q'$ are injective, prove that $Q\oplus L' \cong Q'\oplus L$ as $R$-modules.
\end{enumerate}

The $R$ modules $M$ and $N$ are said to be \emph{projectively equivalent} if $M\oplus P\cong N\oplus P'$ for some projective modules $P,P'$. Simiarly, $M$ and $N$ are injective equivalent if $M\oplus Q\cong N\oplus Q'$ for some injective modules $Q,Q'$. The previous exercise shows $K$ and $K'$ are projectively equivalent and $L$ and $L'$ are injectively equivalent. 
\end{problem}
\begin{solution}TODO

\end{solution}\oneperpage



\end{section}




















\end{chapter}
